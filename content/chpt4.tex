
\chapter{异地作战}

多年前,我曾与一所高校(非哈佛大学)的教师合作,探讨如何促进学生的理解性学习。活动进展顺利,教授们表现出浓厚兴趣并积极联系自身实践。然而,最令我难忘的是一位物理学教授在休息期间愁眉苦脸地对我说:“我的学生根本不理解。”

“怎么讲?”我问。

“举个例子,”他继续说,“我教并非特别困难的基础物理,但学生们就是有很多地方搞不明白。比如,我们研究物体下落的规律。我会出这样一个题:有一座100米高的塔,一个10千克重的物体放在塔顶,有人将其推落,物体需要多长时间落地?”

我知道这是典型的习题,有标准的解题公式。如果知道公式,甚至无需推导,只需将数值代入公式即可。物体的质量是刻意设置的干扰项。忽略空气阻力的情况下,落体的加速度与其质量无关。“好的,”我说,“到目前为止我明白。”

教授继续说道:“终于到了期末考试那天,我0出了这样一道题:有一个100米深的洞,一个10千克重的物体放在洞口边缘,有人将其推入洞中。物体需要多长时间落到洞底?信不信由你,竟然有好几个学生没能解出这道题。甚至还有个学生考完试后跑来跟我抱怨,说:‘教授,我觉得这道题不公平,整个学期我们从来没做过关于洞的题。’”

我给这位教授一些建议,告诉他如何避免掉进类似的“坑”里。我私下里也纳闷,这门课的教学过程究竟是怎样的,学生们竟然无法建立如此显而易见的联系。但这段经历给我留下的最主要的回忆,还是关于“塔”和“洞”的问题——这个我之后津津乐道的故事。不久前,我将这个故事放在一个更大的报告中讲述。会议结束后,当听众陆续离场时,有几个人走过来对我说:“这个故事是你为了说明某个观点编的吧?”不,我没有!这是真事!

“塔”和“洞”的故事很好地说明了学习迁移是如何出乎意料地失败的。“迁移”的概念源于学习理论,其基本思想很简单:人们在一种情境中学到的东西,会影响他们在另一种情境中的学习和表现。例如,学生们学习了关于“塔”的问题,理想情况下,这应该能让他们应对“洞”的问题。你学会了开车,过段时间当你因搬家租了一辆小卡车时,你会发现自己也开得不错。我们都掌握了读写能力,并将其轻松应用于阅读各种材料,从日报到所得税申报表。你了解了一些关于法国大革命或美国内战的知识,当你看报纸或听新闻广播时,你会将它们与伊拉克或阿富汗正在发生的事情进行比较和对照。一位老人听到关于感冒和流感主要通过接触传播的说法,并将其转化为实际行动:在每年的高发季节,他会谨慎地握手并勤洗手。

换句话说,“迁移”就是“异地作战”,即将我们学到的“游戏”及其零散组件,不仅应用于其原始情境,也应用于其他可能有所帮助的环境。

“异地作战”听起来容易,实际并非如此。回想一下引言部分关于波士顿红袜队的例子。棒球场馆差异很大,红袜队客场作战时,会遇到布局不同的球场,这些球场有着各自独特的优劣势,外加不那么友好的观众。“异地”因素的重要性在不同体育运动和“游戏”中差异很大。比较一下高度标准化的室内运动乒乓球和每座山都不同的登山运动,可见一斑。

同样,关于学习迁移的研究也警示我们,对某些主题而言,“异地作战”是一个非常严重的问题。学习者往往无法表现出我们期望的学习迁移。教育的使命因此未能达成。作为教育工作者,我们需要努力促成我们期望的那种联系建立,这就是为什么“异地作战”是整体学习的第四条原则。

\section*{迁移的意义}

你可能会问:“‘塔’和‘洞’的问题——这不是学习的问题,而是迁移的问题?学生们并没有真正理解‘塔’的问题的本质。” 这个问题提得很好,“迁移”的含义确实有些模糊,它究竟与学习和理解有何关系?

从某种意义上说,任何学习都包含某种程度的迁移。这里有必要引入文献中粗略区分的“近迁移”和“远迁移”。近迁移指的是将所学知识与非常相似的情境建立联系。阅读报纸与阅读历史教科书并无太大差别,驾驶小卡车与驾驶轿车也相差无几。至于远迁移,将美国内战与伊拉克的紧张局势联系起来就差异性较大了。我想起一位朋友,在一次奶酪和葡萄酒的野餐中,他发现没有刀子切奶酪,于是就用信用卡代替。我的朋友将他对信用卡主要特征(坚硬、薄)的认识迁移到解决一个截然不同情境中的问题。

回到“塔”和“洞”的问题,将一些学生在“洞”的问题上遇到的困难视为近迁移失败是合理的,他们错失了一个本该轻易建立的联系。但可以肯定的是,他们进行了一定的学习。那位考后向教授抱怨课程中没有“洞”的问题的学生,显然对她解决“塔”的问题的能力充满信心,她能将该模式应用于不同高度的塔、不同重量的物体。非常近的迁移做得很好。

近迁移和远迁移的对比很方便,但也相当粗略。对于“近”或“远”并没有官方的衡量标准,“近”与“远”的概念依赖于我们对不同情境的直觉判断,以及构建两者之间桥梁的难易程度。

理解是“近”与“远”故事的一部分。一些学生在“洞”的问题上遇到的困难意味着教授没有获得他所期望的理解。对至少是近迁移的期望,是“理解”这一概念的内在要求。回顾第一章关于理解的表现模型的讨论:理解某事物意味着能够灵活地运用你对它的已知信息进行思考和行动,相关的思考和行动需要有一定的扩展,超越公式化的套用。当学生们发现自己被“洞”的问题难住时,这种扩展就显得远远不够。

在“洞”的问题上遇到的障碍,揭示了人类学习中一个常见的模式:固着于表面特征。这种模式在学习科学中经常出现。研究表明,在学习一个主题的早期阶段,学生们通常根据表面特征而不是潜在的原理来对问题进行编码。例如,关于单摆的问题和关于斜面的问题都涉及能量守恒,但起初学生们根据表面特征对它们进行分类:这里是单摆问题,那边是斜面问题,它们是不同环境中的不同事物。将“塔”的问题和“洞”的问题视为截然不同的类型,是表面编码的一个极端例子,既是迁移的失败,也是理解的失败。
有必要补充说明“迁移”的另一种含义。当我们说“迁移”时,我们通常指的是积极的影响。然而,从技术角度来看,“迁移”可以是积极的,也可以是消极的。当某人在一种情境中学到的东西损害了其在另一种情境中的表现或学习时,就会发生负迁移。

负迁移很常见。例如,来自美国或其他靠道路右侧行驶的国家的人们,在英国这样靠左侧行驶的国家开车时,通常会遇到负迁移。这些“异地”驾驶员需要时刻保持警惕,以克服他们原有的驾驶习惯。

一定程度的负迁移经常发生在第二语言学习中。一些发音相同甚至词源相似的词语意思却不相同,从而导致错误。“Actual”在西班牙语中并非“真实的”意思,而是“现在的、目前的、当代的”意思。第二语言的句子结构总是有些不同,学习者经常会将母语的句子模式套用到新语言中,从而扭曲了新的语言习惯用法。

在日益全球化的世界中,人们不断跨越国界和文化界限,并带着他们对谈话模式、商业惯例、餐饮礼仪,甚至关于如何以及何时说善意谎言的期望。正如大量关于如何在巴黎、北京或东京表现得体的指南书籍的涌现所表明的那样,旅行者的行为可能会令人困惑甚至感到冒犯。

虽然负迁移是一个重要的现象,但本文不再赘述。它主要是一个初始阶段的问题,通过一定的坚持和指导很容易克服。人们一直在进行从右侧驾驶到左侧驾驶(或反之)的转换。在语言学习中,新手也很快就能克服早期对句子结构的错误运用。更严峻的挑战是积极的迁移,尤其是远迁移。

我力图以一种直截了当的方式来描述迁移。然而,关于如何最好地概念化迁移的争议至今仍然存在。一些学习理论家认为,我们应该完全停止谈论迁移。它与学习的含义过于纠缠,以至于无法拥有一个单独的标签。另一些人则对将迁移描绘成“在此处获得一揽子知识并在彼处应用”感到不满。他们提出了更广义的概念化迁移方式。我完全认同这种片面性的问题。从整体学习的角度来看,我更倾向于将迁移视为构建丰富的、可扩展的行动库——即我们为了其他场合而混合搭配的“游戏”。这里不适合对仍在激烈的争论进行回顾,我只想简单地承认它们的存在。

为什么要如此关注迁移?除了词典中的释义,除了像近迁移和远迁移、正迁移和负迁移这样的区分,迁移对于教学和学习都具有核心意义。如果不以这种或那种方式、用这个或那个标签来理解迁移,那么通识教育就毫无意义。教育的全部意义在于培养人们的技能、知识和理解力,以供在其他地方使用,而且常常是非常不同的地方。正如“塔”和“洞”的故事所表明的那样,我们并非总能到达我们想要到达的“他处”。无论称之为迁移的失败还是其他什么,我们都面临着严峻的挑战。

想象一下,如果除了非常近的迁移之外,任何其他形式的迁移都难以实现,教育会是什么样子。学校将成为与外部世界几乎毫无关联的学习的修道院。这些“修道院”的参与者学习阅读只是为了继续阅读更高级的文本。他们学习长除法或牛顿定律只是为了在那些沿着走廊排列的教室里继续学习更高级的主题。他们对军事政变和意大利十四行诗形式的日益精深的掌握,将只能通过他们彼此之间的演讲和对细微之处的辩论来表达。仅此而已,别无其他可能。他们将成为“塔”的囚徒,学习着精美复杂却又极其脆弱的知识。

问题在于,这种修道院式的学校并非完全是虚构的。在某些方面,典型的教育机构也表现出这种封闭性。它们像封闭系统一样运作,其教学和测试的内容,在很大程度上,仅以非常狭隘和有限的方式触及人们在外部世界的行为。学习迁移有可能跨越这种修道院式学校的围墙,但这只有在我们能够弄清楚如何出色地“异地作战”的前提下才能实现。

\section*{迁移的困境}

从香港到开普敦,从墨尔本到波士顿,大量的教学都秉持着我喜欢称之为“牧羊女波比” (Bo Peep theory)的迁移理论。还记得牧羊女波比丢了羊的故事吗?根据那首童谣,她无需担心,因为“放着不管,它们自己会回家/跟在后头摇着尾巴”。本着同样的精神,世界各地的教育工作者都认为,我们想要的迁移几乎会自动发生。当人们原则性地学习了一些普遍适用的知识时,他们就会以一种普遍的方式应用它。羊群自然会回家。不存在迁移的问题。

这种乐观态度很好,因为有时确实无需担心。一些知识和技能很容易迁移,阅读就是一个很好的例子,眼前的文字会强烈地引发阅读反应。对于一个流利的读者来说,很难不读上一两行易懂的文字。不妨试着看下一段而不去读它!

有时我们受益于直接的提示,即被要求去做某事:给我写个便条,递一下盐,你能核对一下这些数字吗?另一种提示可能来自某种情境的“可供性”。“可供性”是指物体或情境中强烈倾向于某种用途的特征。椅子可供于坐,树桩或齐腰高的栅栏也可供于坐,人们坐在上面并不会觉得他们正在建立令人惊讶的联系。

即使是远迁移,当有强烈的提示时,也不一定是问题。隐喻和类比在文学、科学和其他领域的频繁使用就说明了这一点。比如,爱尔兰诗人Yeats将一位老人比作“插在棍子上的帽子”。在讨论蚂蚁或蜜蜂等群居昆虫时,我们被引导将整个蚁群或蜂群视为一个单一的有机体。我们都能很容易地理解——当然,我们已被告知要在“异地”寻找什么。

总而言之,某些条件有助于迁移。强烈的提示有所帮助。一旦我们心中有了建立联系的可能性,如果我们能够轻松地阐述这种联系,而不是将其视为一个复杂的谜题来解决,那将会有所帮助。使用隐喻和类比的作者自然会考虑到受众,并选择易于阐述的例子。

然而,这些促进迁移的条件也同样是迁移的风险因素。如果提示不够强烈,或者联系不容易阐述,情况会怎样?那么羊就不会回家。

这种情况很常见。前一章讨论了作为一种棘手知识的惰性知识,它就是一个迁移问题。回想一下那枚刻有公元前153年的硬币的例证性谜题。许多人没有意识到这肯定是伪造的,因为公元前/公元后的纪年系统是在基督诞生后才制定的。再回想一下那些访谈,哈佛大学的学生在被问及为什么夏天比冬天暖和时,认为地球在夏天离太阳更近。无论他们记住了多少科学知识,他们肯定知道北半球是夏天的时候,南半球是冬天,反之亦然。这两个例子都说明了提示的风险因素。联系不够突出,无法触发对关键信息的检索。但一旦有人提及,人们就能很容易地阐述这种联系。

有时阐述联系的过程会成为薄弱环节。例如,当柬埔寨红色高棉被称为恐怖分子时,人们会表示赞同;但当二战时期的法国抵抗运动被称为恐怖分子时,人们则会抵触。我们倾向于将反纳粹的法国抵抗运动视为英雄,但这可能会妨碍我们认识到他们某些策略的本质。正如谚语所说,“一个人的恐怖分子是另一个人的自由战士。”

人们在正规教育中接触到许多思想和实践,如果能够广泛应用,它们将具有重要价值——关于公民身份的理念、自我反思的实践、理解政治局势的方式、各种类型的思维技能、像杠杆这样的实用科学相关概念、宽容的态度,以及统计学和概率论的基础知识。但风险因素很可能占据主导地位。回报取决于在与课堂和测验截然不同的各种情境中注意到它们的相关性,在这些情境中,没有人会大声喊道:“现在是回忆第五修正案的好时机”,或者“现在是思考问题另一面的好时机”,或者“现在是看看概率的好时机”。

对这一挑战最悲观的看法是,“异地作战”大体是一场在开始之前就已注定失败的游戏。一些人认为,面对这些风险因素,人类的大脑并不具备实现大量迁移的能力,对于这个问题我们无能为力。我们只能接受它,并非像前面描述的那种极端封闭的修道院式学校那样,而是以一种仍然令人困扰的方式,即大多数学习需要逐个案例、逐个情境地进行。与“牧羊女波比”理论相反,我喜欢称这种观点为“失羊理论”。“失羊理论”认为,我们只需接受许多羊不会回来的事实。当风险因素显著时,我们不能对大多数学习者的迁移抱有太多期望。

是什么导致人们得出如此悲观的结论?一个世纪以来的系列实验记录了远迁移的脆弱性。故事始于二十世纪初一位名叫Thorndike的著名教育研究者,他进行了关于迁移的研究。当时,人们认为像拉丁语这样的困难学科可以训练思维,为其他类型的学习做好准备。Thorndike决定一探究竟,比较了学习过拉丁语的学生和没有学习过拉丁语的学生的学业成就,结果发现学习拉丁语的学生没有任何优势。

在早期的研究中,Thorndike和Woodworth考察了各种可能的迁移。他们的结论是:迁移很难实现,而且,当迁移发生时,它取决于两种表现中具体的相同要素。再次考虑学习驾驶汽车然后发现自己也能驾驶卡车的例子。这不会让桑代克感到惊讶,因为其中有许多相同的要素。方向盘、刹车、油门的功能或多或少以相同的方式运作,即使在更大车辆的操作上肯定存在一些差异。但大多数情境都缺乏相同要素这一关键因素。例如,拉丁语学习与数学没有明显的相同要素,那么为什么前者会促进后者呢?

在过去的几十年里,随着学生开始学习计算机编程,一种更为现代的“拉丁语训练思维”的观点出现了。一些教育工作者认为,编程的严谨性和逻辑性可以训练思维,从而普遍提高思维和学习技能。本着Thorndike的精神,教育研究者开始调查情况是否如此,并且像Thorndike一样,发现通常并非如此。

另一个训练思维的有力候选者是读写能力本身。当然,学会读写的人不仅仅是熟练掌握了文本交流,而且还掌握了能够提高他们一般认知表现的思维模式。伟大的俄国心理学家Luria在1930年代调查读写能力对西伯利亚人群的影响时,发现了一些似乎支持这一观点的结果。那些接受过读写课程的人在某些似乎与读写能力没有直接关系的认知测量中得分更高。

然而,后来的研究对Luria的结果是否完全是他所认为的那样提出了质疑。Sylvia Scribner和Michael Cole报告了一项对一个名为Vai的非洲部落的调查。Vai人发展了他们自己的书写形式,但他们没有类似学校的机构。Sylvia Scribner和Michael Cole使用一系列工具,发现掌握这种文字的瓦伊人和其他没有掌握这种文字的人在认知能力上没有差异。作者指出,Vai人只使用他们的文字来交换某些信息,与读写能力在城市化文化中的普遍渗透完全不同。他们认为,读写本身并不是Luria所认为的那种认知“增强剂”。David Olson认为,与其说是普遍的推理能力,不如说是读写能力所培养的一种强大的语言运用姿态更加重要。无论如何,广泛产生影响的是读写能力和学校教育共同构成的完整文化,而不是作为一项孤立技能的读写能力。

许多迁移失败案例促使一些人提出,我们应该放弃在简单案例之外的迁移。偶尔会有人做出令人惊讶的联系,但这不是我们可以系统地进行教育的内容。怀疑论者认为,学习本质上高度依赖于特定的情境,这种情境性有力地支持了在这些情境中的学习。那些支持这种思想的人通常批评传统的课堂教学过于脱离实际情境,远远缺乏学科探究的丰富特征以及学科的社会维度,因而无法提供良好的理解性学习。对于这一点,人们当然可以赞同,但情境学习观最强烈的另一个信条是,不能期望获得普遍的理解和技能。

\section*{引导迁移}

异地作战!我们对迁移的良好引导就如同它本身一样,既简单又复杂。学习最基本、甚至太基本以至于少被提及的一个原则是:人们通过做学习做。为了促进迁移,最初的学习必须包含我们希望学习者以后要用的某些联系。

再考虑一下物理学教授的学生在“洞”的问题上遇到的困难。如果教授在最初的教学中融入一些“异地作战”,很少会有学生在考试中的“洞”问题上感到困惑。假设教授花一些时间围绕“塔”问题的本质特征进行反思性的抽象讨论,也将帮助学生获得有助于解决“洞”问题的更抽象的表征。假设最初的学习涉及了多个例子,比如“塔”问题、“悬崖”问题和“小鸟掉落树枝”问题,同时将“洞”问题留到考试时再考。各种各样的例子和更广泛的练习也将为学生解决“洞”问题做好准备。

为了更详细地说明,我来说一下我个人的经历。我在哈佛大学教育研究生院教过多年的“认知与教学艺术”课程,向学生介绍有原则的学习设计“整体”,重点介绍认知科学中关于教和学的一系列思想。这些思想包括本书讨论的许多思想。

迁移是“认知与教学艺术”课程的一个重要目标。我希望学生们不仅要学习内容,还要积极地使用它;不仅是在撰写相关文章的意义上积极使用它,还要将其应用于设计真正的学习;不仅完成他们的课堂设计作业,还要应用于对他们的职业和个人生活都重要的设计。因此,我一直努力组织这门课程,使其包含大量的“异地作战”。以下是一些方法:
\begin{enumerate}
    \item 课堂在理论和实例之间不断切换。例子具有刻意的多样性,它们在人文和科学之间取得了平衡。学习者包括许多不同年龄段的学习者,其中一些关心成人学习,一些关注动物学习。课堂活动包括小组反思性讨论和对原则及例子的评估。
    \item 在讨论了理论和实践的许多细微之处的时候,也将最重要的设计原则以图表和总结图的形式呈现。
    \item 学生们做的是设计项目而不是论文。每个学生都需要制作某种教育干预的原型,一些人编写教师手册,一些人准备研讨会,一些人创建网站,一些人组装博物馆展览的模型。只要它构成教育设计的具体表达,而不仅仅是高层次的描述,几乎任何形式都可以。原型需要展示课程内容中几个原则的良好应用,学生需要在报告中解释并证明这些联系的合理性。
    \item 学生可以自由选择他们的设计项目。他们被鼓励选择一些对个人有意义的东西。许多学生在攻读学位的同时还在工作,他们被鼓励选择服务于其职业环境的项目。有时,学生在前一个学期或在职业环境中已经开始了一个他们想要继续的学习设计。没问题,只要他们以后的工作清楚地融入了课程中的设计原则。
    \item 学生从预提案到最终项目的几个阶段都会收到大量的反馈,查看其他学生的项目,识别与其他课程的联系,识别与先前和当前经验的联系,并参与许多“快速设计”活动,将设计概念应用于小问题。
\end{enumerate}

在所有这些活动中,“异地作战”发生在何处?系统地突出它的一个方法是查看迁移的“什么”、“到哪里”和“如何”——什么是应该迁移的,应该迁移到哪里,以及如何完成迁移。

“什么”指的是学习内容。我希望学生们能够迁移从认知科学中提取的基本设计原则。这就是为什么课程反复强调这些设计原则,并在一系列图表中进行总结。关于“哪里”,我希望学生将这些思想迁移到他们自己生活中的各种实际情境中,无论他们以何种身份参与教育,无论是作为教师、设计师、企业培训师、课程编写者、教育管理者或其他身份。因此,课程内容包括大量的实际例子,理论和实践之间不断循环;并且鼓励学生开发与其当前职业实践和志向相关的设计项目。

“如何”直接来自前面提到的框架,即Salomon和Perkins的“高路/低路”迁移模型。“认知与教学艺术”课程中的学习体验旨在锻炼反思性抽象的“高路”和自动触发良好实践反应的“低路”。有时用“架桥”和“拥抱”来表达更容易理解。“架桥”意味着以学习者进行各种有意的周密思考的方式进行“异地作战”。“架桥”在课程中体现在学生不断反思原则和例子,他们进行以原则为导向的快速设计任务,以及他们根据课程中的关键概念阐明其设计原型的基本原理。

“拥抱”意味着以做接近最终设想的应用的方式进行“异地作战”。“拥抱”通过使用各种各样的例子在课程中体现出来,希望每个学生都能找到一些反映其特殊兴趣领域的例子。此外,“拥抱”也体现在项目的选择上,鼓励学生选择与其自身情况和兴趣相关的项目,甚至是将立即投入实际使用的项目,以及继续他们在职业环境中已经进行的工作的项目。

不可避免的问题是:这一切的效果如何?与任何教育活动一样,结果并不完美。一方面,尽管课程不断强调,但总有一些学生制作的项目对课程重点强调的设计原则的运用很薄弱。一些学生似乎总是没有投入足够的努力,只是制作了肤浅的设计原型。另一方面,许多项目都非常出色,通常远远超出了课程的实际要求,学生们制作出他们在各种职业环境中继续使用的设计也很常见。总而言之,我感到满意,学生们似乎也很满意。我认为我们获得的相当程度的满意感要归功于我们几种“异地作战”的方式。

\section*{作为“导入”的迁移}

到目前为止,所有的例子都是作为“导出”的迁移,它们探讨的是今天的学习如何为以后的更广泛应用做好知识准备。然而,我们也可以从相反的方向思考迁移,即“导入”而非“导出”的迁移,旨在增强当前主题的学习的迁移。

一个常见的应用此原则的例子是简单地提醒学习者他们已经知道和不知道的东西,或者让他们自己提醒自己。几乎每位教师都曾或多或少地使用过某种版本的“我已经知道什么,我认为我知道什么,我需要知道什么”的程序。

这里有一个更令人惊讶的应用。前面我提到,我的同事Gavriel Salomon和我共同开发了本文分享的一些关于迁移的思想,他还研究和平教育。几年前,他向我介绍了一项由以色列海法大学的I. Lustig进行的一项有趣的研究(第3章简要提到过这项研究)。要教导年轻人对他们自己社区中敌对的民族群体采取更开明、更尊重和更敏感的态度并不容易。回想“阈值经验”的概念,在面对如此多的情感障碍时,创造一种开明的“阈值经验”并不容易。

Lustig的方法是迁移从另一个情境中获得的概念和技能——先进行“异地作战”,以便以后在“本地”进行。“本地”指的就是以色列和巴勒斯坦的冲突。以下是它的运作方式:

Lustig安排了一些以色列十二年级的学生用四个月的时间学习北爱尔兰冲突,对冲突双方的不同观点进行透彻阐述。不过,该项目一次也没有触及以色列-巴勒斯坦冲突。项目结束后,研究人员要求学生写两篇论文,一篇论文要描述以色列-犹太复国主义的观点,另一篇要描述巴勒斯坦的观点。研究人员将结果与未参加任何特殊项目的以色列学生对照组进行了比较,研究北爱尔兰冲突的学生被证明更能写出条理清晰的论文,阐述巴勒斯坦的观点。他们在这样做时更频繁地使用第一人称,这表明他们的视角转换能力有所增强。此外,他们的论文中包含了更多与冲突解决的可能性相关的术语。

总的来说,容易涉及防御性的思维和行为模式可能很难通过贴近自身情况的案例直接学习,而更容易通过表面上较为遥远的案例学习,然后再将其应用到自身。我想起了一位阿根廷同事Ernesto Gore告诉我的一个类似例子。作为一名组织发展方面的专家,他解释了他如何经常帮助商业客户洞察他们自身组织内部的问题。他会讲述另一个组织的故事,一个为了达到目的而虚构的故事。他会描述那里的困境、适得其反的做法以及改进的失败尝试。听着这个故事,他的客户会开始建立他们自己的联系,看到他们听到的一些内容是如何应用于他们自己适得其反的行为的。Ernesto向我保证,这比直接告诉他的客户哪里出了问题以及该怎么做要有效得多。

\section*{充分利用迁移}

我正在想象一次围绕美国宪法艰难制定过程的学校学习经历。我正在想象一次涉及厨房和作坊化学的家庭教育经历。我正在想象一次伦敦塔之旅。这些不一定是最好的主题,但它们都是机会。原则上,它们都提供了学习迁移的潜力。问题是这些机会是否被抓住了?

没有必要像天真的牧羊女波比那样,期望迁移的“羊群”会自发地找到回家的路。良好的引导——高路和低路、架桥和拥抱、为学习做准备的“发明”、作为“导出”和“导入”的迁移——的艺术和技巧使教育工作者能够利用迁移现象进行更有意义的学习,但这只有在我们利用这些机会时才能实现。

这里最重要的事情是一旦说出来也许就很显然了:充分利用迁移首先意味着教授完整的“游戏”。例如,这意味着看待宪法的制定,不仅要将其视为一个特殊的历史故事,还要将其视为一次复杂的谈判,其中包含对其他国家建设背景和其他一般性谈判的启示。这意味着将厨房和作坊化学视为一个人一生中可能会多次使用的一门手艺。这意味着看待伦敦塔,不仅要将其视为一座特殊的监狱,还要将其视为一种建筑模式和一种实践模式,这种模式在人类历史乃至今天都令人沮丧地普遍存在。

为了获得强大的迁移,学习者需要广泛学习一些重要的东西。几乎任何丰富的主题,如宪法、厨房和作坊化学或伦敦塔,都具有这种潜力,但这种潜力必须得到开发。不幸的是,学生在典型课程中学习的大部分内容,由于其框架和细节过于狭窄而意义不大。

第2章对“让游戏值得玩”的探索提到了范围广泛的理解,即映射到生活许多方面的概念和例子系统。那里的例子包括统计和概率、正义的本质、生物的本质、种族仇恨的根源、由谁来决定什么算作历史,以及人类的弱点和错误等等。一个好的做法是选择范围广泛的理解来突出,无论正在考虑的具体主题是宪法的诞生、厨房和作坊化学、还是伦敦塔。民主的概念和实践、酸的处理和操作、监禁和酷刑的效力和伦理,所有这些都可以在我们复杂的世界上找到其相关性。但只有当以能够突出主要主题的方式来对待宪法、厨房和作坊化学以及伦敦塔时,这些宏大的思想才有机会发挥作用。

列出范围广泛的理解可能提供的标准并不难,如果你愿意,可以将其视为“导出”和“导入”的标准,即在“理解性教学”的术语中,什么使主题尤其“具有生成性”的标准。以下是一些值得牢记的标准:
\begin{enumerate}
    \item 学科意义。这些思想在其自身学科背景内外是否具有广泛的意义?它们是否帮助我们以不同的方式看待世界?
    \item 社会意义。这些思想是否关系到整个社会的关注?
    \item 个人意义。这些思想是否与学习者和教师或导师或家长的希望、愿望、好奇心和需求产生共鸣?
    \item 魅力。这些思想是否具有吸引力、诱惑力、引人注目?范围广泛的理解在没有太多魅力的情况下也可能在技术层面是有用的,但魅力会增强这个作用。
\end{enumerate}

无论是“导入”还是“导出”、“高路”还是“低路”、“近迁移”还是“远迁移”,以及几乎任何特定主题的潜力——它们都构成了一种愿景,即教育如何能够更广泛、更有效地影响学习者的生活。让我们通过“异地作战”来充分利用我们所教和所学的内容。

\section*{学习的奇迹:异地作战}

我在思考如何组织今天的学习,使其能够广泛地影响和赋能学习者的生活。首先,我可以问自己:今天的学习在其他什么地方会有用?我该如何帮助学习者建立联系?

我在思考如何识别迁移可能成为问题的情况。在这里,我想记住,当在其他地方使用的线索很强,并且在其他地方应用的细节很清晰时,迁移就很容易发生,就像基本的阅读技能一样。不幸的是,对于许多主题来说,线索很弱或者细节很微妙,今天的学习需要预见到这一点。

我在思考如何组织教学以促进迁移。思想和技能通过以更普遍的方式编码知识的反思性抽象的“高路”和提供各种实践的多样化应用的“低路”传播到其他时间和地点。如果我可以设计包含“架桥”(反思性抽象,“高路”)和“拥抱”(模拟各种应用的刻意努力,“低路”)的学习活动,迁移将会得到改善。

我在思考如何以更积极的方式学习内容,通过更多的“架桥”和“拥抱”来促进更好的迁移。我想起了可以提供帮助的具体实践,其中包括基于问题的学习和“发明以为学习做准备”。

我在思考如何通过“导入”先前知识的迁移来增强内容学习。我可以提醒学习者回忆他们在相同或相关领域的先前知识。对于敏感话题,学生可以通过首先研究一个与自身情况相去甚远的情境,然后再将其应用到自身情况来更好地学习。

我在思考如何最大限度地利用迁移。我可以在这里更有抱负。让我找到看似具体的主题和技能中范围广泛的理解,并突出它们。

