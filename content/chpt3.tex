\chapter{攻克难点}

我小时候也上过钢琴课。但与多数人不同的是,我对钢琴有着持久的热情,父母不需催我练习。一年后,他们问我是否想继续学习时,我回答“是”。当教了我几年的老师说她已经倾囊相授后,我又找了另一位老师。一直到了大学期间的暑假,我仍以半认真的态度继续学习。

练钢琴很有趣;当然,也不全都是乐趣。我喜欢计划A:把曲子完整地演奏几遍。对于练习困难部分的计划B,我从没喜欢过:集中精力在一首特定的曲子上,挑出几个最难的乐段,分析这些难点是什么,调整手型和指法,然后小心地多次演奏这些乐段,以解决难点。

仅有计划A不行。有一个常见的说法来形容这种情况:练习你的错误。总体而言,仅通过多次演奏整首曲子,难点并不会得到改善。即使其他部分的流畅度和表现力有所提高,但困难部分依然一塌糊涂。尽管我对困难部分开始稍加关注,但从没喜欢上这种练习。我明白这是过程中的必要部分,但对我来说它始终是一个勉强的部分。我的心始终属于计划A。

“重点关注难点”不仅是大众智慧。几年前,认知心理学家K. Anders Ericsson及其同事进行了多项有关专业技能的研究,尤其专注于音乐和体育等注重表现的精英技能的发展的研究。Ericsson的一个问题涉及天赋与正确练习的相对贡献。他们发现,天赋在他所称的“刻意练习”中所起的作用微不足道——这与一般的看法相反。

举例来说,一位精英高尔夫球手的练习日程可能包括重复的“沙坑”训练,旨在改善沙滩击球。虽然水平较低的高尔夫球手也可能希望改善自己的沙滩击球,但他们的练习内容很可能缺少这种集中而乏味的训练。即使精英与普通高尔夫球手在练习时间上相当,表现的提高也仅通过对任务的深入理解和对旧表现的重构所投入的有意识努力而实现。注意,这不仅仅是指重复练习困难部分的问题。这涉及到对其进行解构和重构,以便以新的更好的方式执行它们。

回到钢琴,除了专门练习困难部分(但按照Ericsson的标准,这种练习并不够频繁),我还学到,处理难点不一定就要“守株待兔”,还可以提前准备。一种标准方法是练习音阶和琶音,它可以为许多完整作品中出现的片段做准备,同时增强手指的力量和灵活性。无论如何,要在某个领域变得优秀,就必须练习那些困难的部分。

\section*{对难点的忽视}

难点如何才能引起注意呢?大多数时候它们并没有得到关注,或者被关注得远远不够且方式也不太正确。根据我的学习经验——我敢打赌你的经历也差不多——这似乎反映了我们所说的“情感认知理论”:用心去感受,牢记在心,下次做得更好。

“情感认知理论”是全球普遍默认的做法。学生完成一组关于线性方程的问题并提交。几天后,学生收到批改的作业,上面有分数和一些标记和评论,如“假设”、“忘变号了”。老师希望学生能认真对待这些反馈并牢记在心,不再犯这些错误。学生提交一篇关于Martin Luther King的论文,几天后收到批改,附带分数和一些评论,如“引人注目的开头”、“这里失去动力”以及“需要更多证据”。老师希望学生能认真对待这些反馈并牢记在心,在未来的论文中带来更多的动力和证据。

然而,“情感认知理论”存在诸多问题,我几乎不知道从何说起。一方面,这种理论假设学习者内心有热情,关心提高表现,即使课程已经转到下一个主题,仍然关注反馈而不仅仅是分数,会努力记住这些反馈并在未来的类似任务中加以应用。另一方面,这种理论假设学习者对主题有足够的理解,能够理解相对稀疏的反馈并有效利用。还有,这种理论假设学习者很快就会有机会再次尝试,但实际上,他们往往没有这样的机会。即使是最投入的学习者和灵活的思维,也不太可能在几周后充分记住这些反馈,以便在期末考试或下次论文截止时再次尝试。

这里有一个故事,说明即使是“情感认知理论”的强版本也不够强。几年前,我的一位聪明的博士生Bill William Kendall开展了一项关于学生学习标准高中代数的研究。他的假设对我们俩来说都很合理:如果学生更加关注他们所犯的错误,记录在错误日志中,诊断出问题所在,并在完成作业和测验前回顾他们的典型不足,那么他们的表现将会提高。这就是我所说的“情感认知理论”的强版本。它包含了鼓励学生积极处理反馈并加以应用的要素。

Bill与五位教师合作,在一年级代数的第二学期中引入了错误日志。每个班中都有一些学生都要求定期查看老师反馈给他们并做了错误标记的作业和测验。这些学生在日志中记录下具体的错误,并尝试写下出错原因,为自己制定改进的规则。在下次作业和测验前,他们要回顾自己的日志。

第二学期开始和结束时进行标准代数掌握水平测试,Bill据此来评估干预的影响。他比较了保持错误日志的学生与同这三个班中不记录错误日志的学生的表现。令人失望的是,他发现错误日志并没有任何帮助,表现没有显著差异。

问题出在哪里?在此过程中,Bill对一些记录日志的学生进行了访谈。正如一位教师所说,学生们“似乎不知道如何找到自己的错误”。大多数学生对代数的理解不足,无法以有意义的方式解释他们的错误。从老师那里收到的作业反馈中,问题的解答旁边有个“X”,他们会对此感到困惑,却通常无法得出在下次能用的有用解释。老师在课堂上讨论各种问题集和测验时提供的解释,显然不足以帮助学生克服这个障碍。

回到“情感认知理论”的问题。Bill聪明而坚定的努力使这一理论更有效,这让我领悟到一个非常重要的道理。它揭示了可能存在的两个弱点:反馈信息不够丰富,主要只是对错之分;而大多数学生对代数的熟悉程度不足,无法有效地从这种相对稀疏的反馈中学习。有时情况可能好些:学习者得到更丰富的反馈;有时相对稀疏的反馈也足够。但这并不是一个罕见的问题,它是一个常见问题。不论我们如何处理难点,看来我们必须超越幼稚的“情感与认知”方法。

\section*{拥抱难点}

时机至关重要!我们深知,对于妙语运用、股市投资、“银行抢劫”、体操表演,乃至选择合适通勤以避开高峰,时机的把握都至关重要。同样,有效应对学习中的难点也高度依赖于时机的选择:何时介入、以何种方式介入、投入多少精力。妥善处理学习难点是教学过程中一项根本的结构性挑战。我们需要构建类似Ericsson提出的对于深度学习至关重要的“刻意练习”模式。我们不应回避难点,应积极拥抱它们。

我们回忆起上一章“理解性教学”的一个核心概念——“持续评估”。“持续评估”的核心理念是尽早并频繁地进行评估,而非仅在单元或章节结束后。此处的评估并非为了评定分数或衡量教学成效,而是为了直接驱动学习过程,强化学习效果。那么,富有经验的教师、教练、导师以及其他教育工作者如何通过持续性评估来应对学习难点呢?以下是一些值得关注的模式。

\subsection*{可操作的评估}

正如Bill Kendall的研究表明,简单的对错反馈通常无法为学习者提供足够的信息。诸如“需要更多证据”之类的宽泛评论也无济于事。

第一章中的水球蹦极示例提供了一个鲜明的对比。回想Kenna Barger如何引导她的学生进行线性方程建模:学生们以小组为单位,预测弹性绳的长度,使其水球从学校屋顶蹦极下落后,能达到尽可能低的高度但不触及地面而摔破。在此过程中,Kenna Barger不断提醒学生们他们拟采取行动的具体困难。她通常通过提出问题而不是直接告知具体做法来实现这一点。这样做更好,因为这样她的学生就必须自己解决一部分问题。她提出的问题足够具体,能够指明正确的方向。

或者以之前的Martin Luther King文章为例,与其给出“需要更多证据”的评论,不如这样写:“需要更多证据来证明不仅是黑人社区,其他重要的社群也受到了他的信息的影响;你能找到引文来支持这一点吗?”这样的评论足够具体,可以付诸行动。

\subsection*{面向理解的评估}

Kenna Barger在蹦极活动中进行的持续性评估并不仅仅是为了纠正错误,它还增强了学生的理解。同样,对证据的需求也拓展了写作者对其论证案例以及写作技巧的理解。与整体学习和理解之间的深刻联系相一致,良好的反馈不仅涉及正确性问题——计算的准确性、拼写、语法等——还涉及理解上的优势和不足。

\subsection*{同伴评估和自我评估}

持续性评估面临的实际困境之一是教师的时间投入。大多数情况下,教师没有足够的时间每天为有效的学习提供所有必要的反馈。Kenna Barger的“蹦极”课就让她忙得不可开交。因此,需要其他反馈来源,比如学生互评作业,甚至借助评价标准(一套关于应关注内容的具体指南)进行自评。

诚然,学生们常常不确定该说些什么。他们也常常不愿纠正同伴的错误。然而,简单的评价标准能极大地提供帮助,使大多数学生能够给出较为周全的反馈。此外,学生从评估他人作业中学到的东西和被评估同样多。提供反馈需要一种反思性的立场和对问题的具体阐述,因此学生也将培养出可以应用于自身作业的评估技能。有关此方面的更多想法,请参阅第六章。

\subsection*{互动式反馈}

即使反馈很具体,也并非总是我们期望的。除了学校,我和我的同事们还花了一些时间研究组织环境中的学习。我记得有一次,一名员工提出了一个方案,却受到了老板非常怀疑的反馈:

员工问:“你觉得这个计划怎么样?”

“谢谢你提出这件事。原则上似乎不错,但我看不到它在实践中如何真正运作。可能会有围绕知识产权的法律问题,然后是成本因素;我估计会在现有预算的基础上再增加10\%。而且还有谁会真正支持呢?”

这种回应不仅没有肯定员工的努力,而且我们看到,老板根本没有真正理解这个提案。老板的批评要点有些偏离主题,而且他也没有确认自己是否理解了。在沟通失误如此频繁发生的情况下,这真是一种遗憾。

一种思考这个问题的方式是认识三种不同反馈风格:纠正式、安抚式和互动式。它们都与课堂和工作场所相关。通常,反馈只是纠正式的。反馈的提供者宣布哪里出了问题,没有对想法、文章或其他评估对象是否真正被理解进行核对。积极的方面要么没有要么只有一带而过的关注。纠正式反馈的基本模式是“是的,但是……”或“不错,但是……”,然后迅速转向难点。

另一种常见的模式是安抚式反馈。这种反馈经常出现在同伴互评以及人们想要表现友好的社交和组织环境中。因此,他们会给出一些完全没有信息量的模糊的正面评价。“嗯,我基本喜欢它!我觉得它提出了一些不错的观点。你上周末都做了什么?”

相比之下,机敏的老板、同伴或老师可能会提供互动式反馈。顾名思义,这种反馈的结构旨在确保良好的沟通,它大致按以下顺序包含三个关键要素:澄清、赞赏以及关注和建议。

1.	澄清。为了避免对讨论内容产生误解,互动式反馈允许进行某种预先检查,例如提出澄清问题。虽然这在单独对话中最容易实现,但该原则也适用于书面作业。我记得我曾给学生写过简短的便条,或在课堂上叫住他们以澄清他们的意图。电子邮件的普及使这一切变得更加容易。

2.	赞赏。互动式反馈包括清楚地指出评估者眼中的积极特征。这可能不如随后的批评性评论那样详尽,但它是一个清晰且充分的表达。反馈的接收者知道在你看来哪些方面做得好,哪些方面应该保持,哪些方面应该继续做。

3.	关注和建议。然后,互动式反馈会分享关注和建议。这些关注和建议侧重于积极的未来:如何改进这项工作或下次做得更好。它们避免批评个人的能力或性格,而是针对具体情况提出建议。

澄清和赞赏的目的不仅仅是为了表现友好,更是为了提供信息。澄清和积极的观点也是信息,而且通常与任何关注点一样重要。此外,澄清和赞赏通过建立一个积极的平台,从而转向关注和建议,有助于避免陷入安抚式反馈的陷阱。例如,当同伴通过寻求澄清和明确指出积极方面来尊重他人的工作时,他们就会更自由地分享他们对关注点的最佳判断。当系统地使用互动式反馈模式时,同伴反馈通常会效果好得多。

\subsection*{隐式评估}

考虑以下情境:
\begin{enumerate}
    \item 三名学生在黑板上展示他们针对一道开放式数学题的不同解法,全班同学就各种解法的优点展开辩论。
    \item 一次关于二战期间法国抵抗运动的讨论引发了一场关于何为恐怖主义的辩论。
    \item 班级成员以小组为单位工作,需要就一项基于社区的生态改善计划达成共识性建议。
    \item 在这种情况下,教师可能几乎不需要提供任何反馈,只需进行引导。但评估在这些交流中大量存在,隐含在学生彼此之间的对话中。一些体育活动会自动产生反馈。当你在钢琴上弹错音时,你通常会知道。当你挥棒落空时,你肯定知道!
\end{enumerate}

学习者可以通过许多方式获得反馈,而无需他人给予。通过以对话、对比和并置的方式进行设置,教师、导师、教练以及其他教育职位的人可以产生丰富的隐含反馈。无缝嵌入事件流程中的评估通常感觉更真实,并且较少引起防御心理。

\subsection*{基于评估结果的可操作机会}

“情感导向理论 ”的一个主要问题是评估过于笼统,难以有效指导教学实践;另一个主要问题是缺乏基于评估结果进行即时调整的机会。相比之下,Kenna Barger则在她的小组学生中巡回指导,提供即时反馈,学生们可以将这些反馈融入到他们正在构建的模型和预测中,然后学生们进行了他们的水球蹦极试验。

对于传统的代数作业,教师可能会先布置奇数题,提供反馈,然后要求学生运用所学知识完成偶数题,当然之后还会提供更多反馈。对于Martin Luther King的文章,学生们可能会根据反馈进行修改,最终的评估和评分将推迟到第二稿或第三稿。

\subsection*{拆分与重组}

是的,有时快速的粗略反馈是学习者所需要的全部,这时“情感认知理论”理论便取得了胜利,建议被铭记于心,以备下次使用。但是我们能对此抱有多大期望呢?如果一个学生收到关于who's和whose的以下反馈,情况会怎样?

Who's意思是“who is”。它是一个缩略形式;例如“Who's giving the party tonight?” Whose是一个所有格代词,例如,“Whose party is it?”明白区别了吗?

学生真的会在下次写作时记住这些吗?类似地,关于击打球的技巧可能无法让球员为下一场真正的比赛中做好准备。即使使用击球机进行练习,也可能仍然存在应对差距。同样,在学习乐器时,通常会将难点单独挑出来进行集中练习,改进它们,但会发现在演奏整首乐曲时仍然会在这些地方出错。

总的来说,通过建议和单独练习得到改进的难点通常会在实际情境中复发。将改进后的技能或理解融入到整体情境中需要成为刻意练习过程中刻意的一部分。当我们认真对待学习难点时,“拆分与重组”的节奏至关重要。

再次回到Kenna Barger的例子,她可能会发现她的学生在进行蹦极项目时存在一个持续性的误解。她可能会决定需要单独上一节课来解决这个困惑。但是在单独授课之后,应该再次安排某种整体性的活动,让学生们可以将他们的新理解融入其中。例如,水球弹射?

再来看Martin Luther King的文章,我们可能会发现班上许多学生在引人注目的论证结构方面存在困难,这需要单独进行练习。当他们有机会立即修改他们的文章,或者通过更加关注论证结构来撰写新文章时,他们所学到的东西更有可能被记住。

总而言之,“情感认知理论”理论通常会导致肤浅的学习,因为它轻视而非重视刻意练习。围绕可操作的、以理解为中心的评估、同伴互评和自评、互动式反馈、隐式评估、基于评估结果的可操作机会以及“拆分与重组”模式来组织学习节奏,可以帮助确保以有效的方式进行难点攻克。

\subsection*{预见难点}

有时好的建议非常精辟,以至于同一个基本思想会被以几种不同的方式表达,例如,“及时一针省九针”,或如哲学家George Santayana的名言,“不能记住过去的人注定要重蹈覆辙”,或者简单一句“未雨绸缪!” 我们不必等到学习者在相同的难点上第二次跌倒;我们可以预见难点并尝试提前处理它们。

看待这个问题的方式有多种,我个人最喜欢的是关注“困难的知识”。尽管任何学习领域都有其独特的难点,但某些类的难题会反复出现。这些类型的难题值得我们铭记于心,因为它们预示了我们可以在一定程度上防范的潜在挑战。根据Santayana的看法,以下是学习任何学科时都需要注意的几个“难点”:仪式性知识、惰性知识、外来知识、缄默知识、技能性知识和概念性难题。

\subsection*{仪式性知识}

掌握处理数字无疑是基础算术的核心和灵魂。然而,“掌握如何处理”可能意味着截然不同的事情。我一直对多年前一位小学生报告的例子印象深刻:

我看例题就知道该怎么做。如果只有两个数字,我就做减法。如果有好多数字,我就做加法。如果只有两个数字,其中一个比另一个小,那就是一道难题。我用除法看看能不能除尽,如果除不尽,我就用乘法。

这是应对策略的极致,清晰、明确,并且在处理那些令人讨厌的算术题时非常有效。它是聪明的,但方向并非我们所期望的。这种知识具有相当无意义的表面特征,就像一种仪式:机械地做某个动作以获得某种解决方案,就像当有人问某种类型的问题时我们应该如何回答一样。所有学习者在面对复杂课程和作业时,最基本的应对策略之一就是找到可执行的仪式,以满足眼前的需求。

所有教育工作者都希望获得比这更深层次的东西,但大量的教学实践都在迎合仪式性知识并认为这就足够了。任何可以用上述策略解决的习题集,任何学生只需知道名称和日期就能做得很好的测试,任何只需组装信息并将其写下来的作文作业,都在助长仪式性知识。有时,仪式性知识会成为师生之间隐含的契约。双方都接受这一约定,以保持一切简单明了。

仪式性知识并非总是坏事。有些信息,例如电话号码,就是纯粹地知道就好。从学习者的角度来看,有时仪式化可能是最佳策略,因为学习情境没有为理解提供充分的支持和指导。也有一些领域,快速的仪式和其背后的理解都很有用。乘法表就是一个很好的例子,既需要熟记数字,也需要理解其背后的原理。

教学如何才能对抗仪式化呢?任何拥抱整体学习理念的教学都有助于弱化仪式性知识策略:开放的学习活动使仪式化无法有效地进行。这样的教学过程会避免迎合仪式化的因素,例如只进行基于事实的测试,或者可以通过像开头例子中那样的算术技巧来应付的习题集。相反,学习过程包含理解性的实践,例如,相似项目,或关于某事物含义或有效性的辩论,或对其他解释的探索。

\subsection*{惰性知识}

有一类有趣的谜题被称为顿悟问题,其中许多都是令人非常沮丧的惰性知识。这有一个例子:

一个人来到稀有硬币博物馆馆长面前,提出出售一枚外观破旧但看起来很正宗的早期罗马时期的硬币,上面标明的日期是公元前153年。馆长立即下令逮捕了这个人,理由是欺诈。为什么?

请允许我暂时不给出答案,以便你可以思考一下这个谜题。我们先来定义一下惰性知识,它是储存在大脑阁楼里的知识,只有通过刻意努力去提取并拂去灰尘才能访问。如果你被要求提供这些信息,或者你自己需要这些信息,你会找到它们。然而,如果没有直接的提示,惰性知识很少被主动使用。惰性知识一个的例子是被动词汇,我们词汇库中都有成千上万个我们认识和理解的单词,但我们很少在自己的口语和写作中使用它们。

回到硬币谜题:馆长知道这枚硬币是假的,因为上面的日期。从公元前0年向前和向后计数的纪年系统在公元前153年还没有被发明。你可能很容易就想出了这个问题,但很多人却没有。这并不是因为这个问题在复杂程度上有多难。相反,解决这个问题只需要将已知的东西正确地联系起来——通常这种联系并没有建立起来。知识是惰性的;它就在那里,但没有被使用。

硬币这个惰性知识例子看起来像是一个过于巧妙的把戏。大量的研究证据表明,在我们学习的大部分内容中,惰性知识是一个严酷的现实。例如,学生们会学习基础统计学,并在课堂上将这些原理应用于实际问题,但不会将其与在课堂外遇到的问题联系起来。学习过计算机编程的学生在面对新的情况时,可能直到被建议这样做才会应用他们知道如何使用的命令。

在一项著名的研究中,研究人员采访了哈佛大学的毕业生,他们正在制作一部名为《A Private Universe》的电影。学生们被问到一个关于世界如何运作的非常基本的问题:为什么夏天热,冬天冷?许多人回答说:“因为夏天地球离太阳更近。”这完全不是正确的解释。几乎所有学生都曾在某个时候学习过冬夏是如何形成的。显然,他们学到的东西是惰性的,或者他们完全忘记了。即使他们忘记了,他们肯定至少掌握了一项相关的知识:北半球是夏天的时候,南半球是冬天,反之亦然。因此,夏天不可能用地球离太阳更近来解释,因为那样的话,两个半球就应该同时是夏天。

惰性知识可以被视为学习迁移中的一个问题,这一主题将在第四章中进一步探讨。人们无法激活他们在其他情境中获得的知识。但至少在许多情况下,这与其说是远距离迁移的问题,不如说是其他问题。前面提到的统计学或计算机命令的应用就相当直接。

John Bransford及其同事报告的一项实验揭示了使知识变得惰性或活跃的一些相关因素。该实验要求两组学生以略微不同的方式学习相同的内容,内容涉及营养学、太阳能飞机、水作为密度标准以及其他事。一组学生的目标是记住内容,另一组学生的目标是思考一次穿越南美丛林的旅程所面临的挑战。实验人员使用直接的信息性问题测试了记忆力,发现两组学生的表现一样好。知识的积极运用需要通过开放式任务来测试,这些任务不会向学生暗示具体需要什么。在这里,实验人员要求两组学生都计划一次沙漠地形的探险。第二组的学生比第一组的学生更积极地运用了他们所学到的知识。

为什么第二组的学生获得了更活跃的知识?似乎有两个因素以某种方式共同起作用:首先,学习过程本身更积极,更以解决问题为导向。其次,具体的学习任务——穿越南美丛林的旅程——更类似于沙漠探险的应用。

第二组学生所做的听起来有点像整体学习,事实也如此。与死记硬背信息的“基础炎”不同,这里存在一个整体性的挑战——计划一次探险。第一次,探险的目标是沙漠地形,学生们掌握了其中的诀窍。然后,针对另一个环境——南美丛林——又进行了一轮游戏。通过整体学习和异地演练的结合,知识就不太可能变得惰性。

\subsection*{外来知识}

许多学童都曾在某个时候讨论过美国总统Harry Truman在二战末期做出的臭名昭著的决定——向广岛和长崎投掷原子弹。学生们很容易走向两个极端,有的认为这是一个极其残酷和幼稚的决定,不仅给无数日本人带来了巨大的伤害,而且开启了充满风险和焦虑的原子时代;有的认为这是完全正确的,它迅速结束了原本可能是一场旷日持久且代价高昂的战争——这也是教科书中常见的解释。

无论如何,这些讨论以及对历史复杂问题的许多其他探索都突出地展示了一个被称为“现代主义”的问题。也就是说,人们倾向于通过今天的态度以及他们对事物实际发展情况的了解来看待历史事件。投掷原子弹确实开启了原子时代,也确实结束了战争。我们很难将自己的思想投射到当时复杂的偏见、考虑和现有知识中。

“现代主义”问题很好地例证了一系列更大的挑战,这些挑战可以用“外来知识”这个词来概括。这种知识很难接受,因为它与我们今天的处境、我们自己的环境、我们的朋友、我们当前的信仰、偏见和惯例不太一致。其他文化的习俗和实践也构成了这种难题。大洋彼岸的人们做事的方式就是让人无法理解,而且实际上常常显得完全错误。哈佛大学教育研究生院的Robert Kegan用“中心主义”——民族中心主义、自我中心主义或你能想到的任何其他中心主义——对这种困境进行了简洁的描述。Kegan指出,各种中心主义都有一个共同的缺陷,它们都将陌生和不舒服的事物视为错误的甚至是邪恶的。

文化根源上的外来知识很容易找到例子。但是,技术知识也可能成为外来知识。我最喜欢的一个例子来自科学教育研究员Marcia Linn。她饶有兴致地讲述了一个学生对著名的牛顿定律的看法,该定律指出,除非受到外力干扰,运动中的物体将不变的速度沿直线运动。这位学生说:“运动中的物体在教室里保持运动,但在操场上会静止下来。”这位学生的理念似乎是:是,我可以学习书本上的概念,学习如何解决书中的问题,但世界并非如此运作。看看就知道了!

外来知识不仅提出了学习新知识的问题,也提出了摒弃旧知识的问题。有时这意味着完全取代旧的知识,但它总是意味着不要牢牢抓住旧知识,以便新的知识能够与之并存。

什么样的学习模式能够抓住摒弃旧知的关键?一种方法是直接对抗,例如,与历史系学生一起,帮助他们识别可能存在的“现代主义”倾向,并将自己更多地投射到当时的思维模式中。另一种方法是间接的推迟对抗。下一章的一个例子,以色列学生通过研究北爱尔兰冲突,能够更好地看待以色列-巴勒斯坦冲突。矛盾的是,有时外来知识还不够“外来”。对他们来说,将北爱尔兰冲突作为遥远的事情来审视,并在之后建立联系,比将巴勒斯坦人的态度视为离家太近的事情来处理要容易得多。

科学教学中一个常用的策略是通过推理和实验来揭示人们普遍接受的信念中的不一致性,从而引导学习者走向外来但更复杂的理论。有时这是成功的,尽管其效果好坏参半。学习者并不总是按照人们期望的推理路径进行,不能最终得到期望的结论。
这种策略的一个有效版本是使用锚定直觉。这个过程从一个学生的直觉是正确的案例开始,鼓励他们扩展这个直觉。科学教育研究员John Clement及其同事探索了常识有时正确有时错误的情况,并敦促学习者调和这些情况。例如,学生们倾向于认为停在桌子上的苍蝇会向下压桌子,但桌子不会向上推苍蝇。这违反了牛顿第三定律,该定律指出,作用力和反作用力相等且相反。然而,同样的学生很容易接受桌子向上推保龄球。这就是锚定直觉。现在让我们想象一下保龄球缩小到苍蝇的大小和重量。当保龄球缩小时,桌子会在哪一点突然停止向上推?像这样的推理模式帮助学生们看到了牛顿定律的普遍逻辑。

我一直认为另一种有吸引力的方法是显式括号法(bracketing),要求学习者跳出他们直观的信念,学习另一种思考问题的方式。花一段时间学习这个新方法,对它进行一次阈限体验!之后,人们会回到那些直观的信念,并仔细审视它们与思考问题新方式之间的平衡。括号法认识到,知识之所以显得陌生,部分原因是学习者没有机会以开放的心态从内部看待它。刻意地将新旧观念之间的冲突括起来,可以争取时间来适应新的观念。

\subsection*{缄默知识}

几乎没有人能告诉你他们是如何走路的,几乎没有人能告诉你他们是如何流利地说出符合语法的句子并持续谈话的,几乎没有人能确切地告诉你为什么他们对一个人、一首诗或一幅画的第一印象是积极的或消极的。他们或许能够指出一两个特征,但很可能这远非完整的故事。

我们的大量知识是缄默的。我们在不确切知道原因的情况下形成了感知和判断,并采取了流畅的行动。缄默知识在许多方面都非常有用,是对复杂情况的一种自然的直觉反应,会以多种方式给正式学习带来麻烦。

让我们从教师自身开始。教师对学生的期望通常部分是缄默的,学生可能会发现自己对究竟期望他们做什么感到困惑,这就是为什么前面讨论的“理解性教学”框架鼓励明确的共享理解目标以及频繁的持续性评估。这些要素使隐藏的游戏浮出水面,以便学习者能够掌握它。

我们真正希望学生学习的部分内容通常也是缄默的,这使得它既难以解释,也难被解释。

最重要的是,某些类型的缄默信念和承诺——例如偏见和成见——会阻碍更广泛和更开放的学习,有时是学习者自己、甚至教师都难以察觉的微妙障碍。

缄默知识的隐藏性使其成为即将到来的第五章的自然主题,更多相关内容将在那里介绍。

\subsection*{技能性知识}

我再次想起了我与钢琴的战斗。很多时候你的手指并不听使唤,它们就是不肯服从指令。同样,试图发出好球的网球运动员或努力避免切球的高尔夫球手可能已经从教练那里听说过或在书中读到过该做什么动作,但通常说起来容易做起来难,动作模式无法自然而然地流畅进行。

换句话说,有时问题不在于知道该做什么,而在于如何去做,需要通过勤奋的刻意练习才能找到感觉。

音乐和体育等表演领域充斥着这类棘手的知识,在更传统的学术领域中也不少见。技能性知识的挑战渗透在第二语言学习中:你不可能通过每时每刻费力地回忆语法来像一个流利的说话者或写作者那样进行交流;它们需要像在你脑海中、笔尖上和舌尖上一样随时待命。阅读能力的发展从根本上取决于高度自动化的反应模式。再来看数学,当组成运算很费力时,就很难将算术或代数作为对世界进行建模的方式来使用。

在技能性知识的挑战成为焦点的情况下,再次提醒人们注意“基础炎”的风险是很有价值的——人们很容易长时间地培养技能,而从不将它们组合在一起。一个好得多的解决方案是回顾前面提到的“拆分-重组”模式。学习者以某种适当的初级版本进行完整的游戏,无论这个完整的游戏是钢琴、运动、某种阅读活动,还是数学建模。与此同时,学习者练习组合性技能,将它们相应的重组到整体游戏之中。

\subsection*{概念性难题}

是什么让相对论如此难以理解?这里很容易回答:“一切! ”

事实证明,“一切”有些言过其实。Paul Feltovich, Rand Spiro和Richard Coulson对概念难度进行了一项分析,提出了一种衡量标准。他们讨论了所谓“高级知识习得”的学习,这种学习远远超出了获取事实和套路的范畴。他们尤其对高风险的医学学习领域感兴趣。

他们给出了使概念和概念系统对学习者更具挑战性的因素列表。以下是他们的评分卡(示例是我添加的):

1.	抽象而非具体:是原理而非例子,是规则而非应用。

2.	连续而非离散:是实数而非整数,是舞者或运动员流畅的动作而非棋盘上离散的走法。

3.	动态而非静态:是月球的轨道而非月球的位置。

4.	同时性而非序列性:联立线性方程组,或者一行诗同时表达多种含义。

5.	有机性而非机械性:是丛林的复杂性而不是瑞士手表的复杂性。

6.	互动性而非可分离性:是月球拉动地球的同时地球也在拉动月球的方式,或者William Butler Yeats在他的诗作《在学童中间》结尾处的名句,“我们如何区分舞者和舞蹈?”

7.	条件性而非普遍性:是诸如“只有当”、“假设”、“除非”等限定条件。

8.	非线性而非线性:是二次方程而不是线性方程,是当人们试图强行推行某个议题时可能引发的政治反弹。

那么再次回到这个问题,是什么让相对论如此难以理解?按照这个评分卡来看,它肯定是抽象而非具体的,处理的是时间和空间中的连续运动,并且是动态而非静态的,其约束条件是同时而非按顺序应用的,并且以互动和非线性的方式进行。但也有一些好消息:它是机械性的而非有机性的,是普遍的而非条件性的。因此,相对论符合列表中的八个难度因素中的六个,并非“一切”。

让我们使用相同的评分卡来审视历史因果关系,即是什么导致历史朝着某个方向发展。这当然也是抽象而非具体的,连续而非离散的,动态而非静态的,同时而非按顺序的,并且当然是有机而非机械的,互动而非可分离的,并且是高度条件性的——关于历史事件如何展开而制定有效的普遍规则是出了名的困难——当然也是非线性的。八个因素全部符合!

换句话说,历史因果关系比相对论更具概念性挑战。表面上看,这似乎难以置信,但也许这是有道理的。历史因果关系起初看起来比相对论更容易理解,因为它入门更容易,任何人都可以理解其中一些基本因素。

然而,历史因果关系复杂性的全面爆发可能很容易比相对论思维的整洁数学科学更令人震惊。Feltovich, Spiro和Coulson当然只是想把他们的列表作为一个粗略的指南。这是一种有用的方式,可以提醒我们概念和概念系统会变得多么混乱,并在我们参与教学过程时为我们提供需要预见的难题和需要解决的目标。

那么我们该如何应对呢?我一次又一次看到的主要事情是:正视概念上的困难,而不是退回到仪式性知识的回避策略中,让学习者参与到能够逐步消除复杂性的初级版本中。通过一系列生动的例子使抽象概念具体化。展示连续性和动态特征,而不是两三个静止的画面。通过例子引出各种因素相互作用的同时性和有机方式。进行完整的游戏。

这还不够,概念难的问题还有另一层。除了Feltovich等人列出的清单之外,概念上难以掌握的知识通常也指向隐藏游戏的挑战。我们之所以难以理解新思想,不仅仅是因为它们非常真实的表面复杂性,还因为它们预设了不明显的概念、框架和思维方式。

\subsection*{构建难度理论}

假设你正在打棒球,你站在击球区,准备挥棒击出本垒打,或者至少安全上到一垒……是什么让这件事变得困难?

投手的工作就是制造困难,优秀的投手有很多方法可以做到这一点,他们投不同类型的球:快速球、曲线球、慢速球。他们会变换投球方式,让击球手不知道会投什么球。他们会瞄准好球区的边角,希望击球手认为这个球会被判为坏球而不是好球。他们甚至可能会投出故意偏离好球区的“坏球”,希望引诱击球手进行一次徒劳的挥棒。这些都是危险,但了解这些也是一种力量,它为击球手和他们的教练提供了关注和努力的方向。他们拥有一套难度理论。

作为教育工作者,我们可以提出的最重要的一个问题是:“是什么让这件事变得困难?”当我们对这个问题有一个很好的答案时,我们就预见到了特定主题或活动中存在的难点。也许通过正确的方法,可以防止这些难点造成严重的损害。

任何有经验的教师、家长、教练、牧师或导师总是会对“是什么让这件事变得困难”这个问题做出某种回应。此外,前面提到的六种“困难的知识”提供了一个非常广泛的备选答案。Feltovich, Spiro和Coulson提出的八个复杂性因素为具有挑战性的概念学习提供了更具体的答案。诸如历史学中的“现代主义”、道德判断中的视角转换,或者人们对物理学的日常理解(例如“运动中的物体在教室里保持运动,但在操场上会静止下来”)等难题,都是针对常见主题提出的“是什么让这件事变得困难”的回应。

所有这些,无论是普遍的还是具体的,正式的还是非正式的,来自研究人员的还是教师的,都是难度理论。这些理论通常不是很学术化,它们也不需要非常学术化才能发挥作用。它们会提醒教师和学习者学习道路上的坑洼,告诉我们我们在教育的道路上哪里需要格外小心。

难度理论并不总是像它们需要的那样具体或有针对性,我告诉你我在这方面的一些经验。多年来,我一直在哈佛大学教育研究生院为学生开设一门名为“认知与教学艺术”的课程,该课程要求参与者针对他们选择的学习议程开发单独的设计项目。这些设计项目的学习目标和环境差异很大,从诸如分数算术等标准的校内主题到诸如养猪、理解膝关节置换术等校外主题。我经常要求学生为他们选择的主题详细阐述简单的难度理论。是什么会让学习者感到困难?因此,你将在你的学习设计中采取什么措施?

答案不必非常详尽就能有所帮助。一名为管理职位人员设计决策制定程序的学生可能会注意到以下困难:在时间压力下,经常会出现忽视长期后果的问题。或者,在等级森严的管理环境中,一种尊重且真正有益的咨询可能会被忽视。一名设计干预措施以培养生态责任感的学生可能会注意到诸如此类的困难:从概念上理解一些问题甚至写出关于这些问题的文章是一回事,而辨别你可以在社区中采取的实际行动(除了像回收利用这样非常简单的行动之外)则是另一回事,而真正去采取这些实际行动又是另一回事!通过诸如此类的描述,学生们为自己提供了更明确的设计目标。

一切都很好,但我多年来发现,一些学生提出了非常单薄的难度理论;一个主题或概念之所以困难,是因为“它很复杂”,或者因为“学习者通常觉得它很无聊”,或者因为“它非常陌生”,或者因为“有太多东西要记住”。因此,我或多或少会这样回复:
请再仔细思考一下,并为你的主题提供一个更具体的难度理论,给我们一个听起来不像可以用于其他一百个主题的理论。有很多主题都很复杂,或者通常很无聊,或者起初很陌生,或者需要记住很多要点。请具体说明!你看,如果难度理论针对特定事物的特定学习挑战,它们就能提供更大的杠杆作用。

此外,学生们经常只是孤立地指出困难而不加解释。例如,有人可能会写道:“是分数给孩子们带来了麻烦”,或者“比起解方程,更难的是把文字题翻译成方程”。这样的难度理论提供了一些帮助,它告诉我们应该在哪里投入更多注意力,但没有说明注意力应该采取什么形式。这里的关键是推动更深入的解释。究竟在处理分数时哪里出了问题?在将文字题翻译成方程时会出现哪些类型的错误?

以下是我通常没有从学生那里听到但有时会在其他场合听到的回应:“要做什么很清楚,但只是没有足够的资源和时间。”这是一种难度理论。它将困难归咎于环境而不是内容本身。是的,我几乎想不到有哪个场合是严肃的教育工作者不希望有更多资源和时间的,但这也可能是一种避免认真对待内容挑战的方式。

也许最没有帮助的难度理论(几乎是一种变态的理论)就是责怪学习者。是什么让这件事变得困难?“嗯,是这些孩子,他们就是不学习,他们就是不在乎,他们之前的学习真的没有做好充分的准备。”这种伪难度理论如此阴险的原因在于,它成为了不采取任何不同行动的借口。

公平地说,任何这些不太理想的难度理论都可能包含一定的真实性。一些学习者确实很懒惰或准备不足,有时时间和资源确实严重不足,有时仅仅知道哪些子主题最有可能引起麻烦就足够了。承认所有这些之后,理想的目标仍然值得坚持:一个针对所教内容具体内容、解释其难点的难度理论。

也许将所有这些想法打包成一个关于教师学习的图表是有用的。
 
当我们一遍又一遍地教授相同的主题时,我们会注意到持续存在的难点。问题是,我们如何回应这些难点?图表上标记为“责怪”的最简单回应是责怪其他事物或其他人——时间不够、资源不足、学习者没有做好准备或缺乏智慧或意愿。这会导致以相同的方式进行教学,甚至可能投入更少的精力。图表上标记为“关注”的更复杂的的回应是关注导致困难的主题的特定部分,这会导致更努力地教学,将更多的时间和精力投入到这些部分。

图表上标记为“解释”的最复杂的回应不仅指出特定的困难,而且解释其原因。有时,记住诸如仪式性知识、惰性知识、外来知识、缄默知识、技能性知识和概念性难题等类型,可以帮助理解一般的难度理论;有时,来自第五章的想法会有所帮助。其他想法可以来自任何地方:同事、导师、书籍等等。目标始终是提出一个相对特定于相关内容的解释性难度理论,所有这些都是为了更聪明地进行教学。

还有一个想法。当你第一次尝试教授任何东西时,仅仅依靠更聪明的教学方法几乎永远不行。第一个难度理论就像今年的新款汽车或微软的新操作系统一样,容易出现漏洞。人们最初对难点会是什么样子了解得不够多。我们这些教师、童子军领队、培训师和课程设计师也需要学习。我们需要攻克我们自己的难点。而最值得努力攻克的难点之一就是为我们的学习者提出一个真正好的难度理论。

\subsection*{从练习到练习曲}

作为对这些关于难点的变体的补充,请记住我们一开始提到的那位努力理顺音乐技巧的年轻钢琴演奏者。他所需的刻意练习的一部分是为了流畅性、速度和音色均匀而演奏音阶和琶音。这是强有力的准备,但它不是很令人兴奋。

大约在十九世纪初,音乐家们意识到必须找到一种方法使这一切更有趣一些,“练习曲”的概念由此诞生。“Étude”只是法语中“学习”的意思。在音乐中,练习曲是特意创作的乐曲,旨在加强特定的技术要素,例如音阶和琶音,从而不用简单地一遍又一遍地演奏。我记得我曾练习过Carl Czerny的一些著名练习曲。

许多练习曲仍然过于接近技术练习,无法完全满足音乐上的享受。Czerny的一些作品集的标题就暗示了这个问题:《手指灵巧的艺术》、《速度练习曲》。这听起来更像是枯燥的训练,而不是激动人心的演奏。但是,十九世纪作曲家Frederic Chopin的二十七首练习曲展示了真正能做到的事情,Chopin的练习曲是极具表现力的精湛作品,旨在突出系统性的技术挑战。

我从来没有好到可以认真演奏它们的程度。我能勉强弹奏一两首 Chopin叙事曲,甚至曾在高中音乐会上比较像样地演奏过一首波兰舞曲。这些练习曲完全是另一个层次的。尽管如此,知道它们的存在仍然令人欣慰,也许有一天我会达到那个水平。
练习曲的思想几乎适用于任何学习领域。许多具有完整游戏特征的学校活动本质上就是练习曲。再次回想一下水球蹦极活动,它是一个练习曲:它是一个带有技术议程的整体性任务,旨在解决学习者对线性方程和数学建模的理解。在工作室艺术中,教师通常布置相当于画布上的练习曲的作业,要求创作探索色彩、透视或纹理等各个方面的完整作品。当要求学习者深入研究一首诗,努力理解它,深入了解其表达策略,对它的意义得出一些有见地且站得住脚的解释,探索它的个人意义,并写下所有这些时,这也是一个练习曲。这是一项完整的任务,而不是一张关于诗歌比喻的练习题,就像Czerny和Chopin的练习曲一样,它也有一个技术议程,培养某些技能和敏感性。

这并不是说直接的练习毫无用处。任何学科都需要某种形式的音阶和琶音练习。但是,当我们可以的时候,值得尝试通过练习曲而不是仅仅通过练习来提高对完整游戏的技术掌握,这一切都本着整体学习的第三个基本原则:攻克难点。

\section*{学习的奇迹:攻克难点}

我在思考如何才能有效地帮助学习者攻克难点。首先,我可以安排定期的刻意练习,并将其反馈到完整的游戏中。

我在思考如何才能避免“情感导向理论”理论,以及那种肤浅的“用心记住,牢记在心,下次做得更好”的评估形式。我需要弄清楚学习者如何才能从我或其他学习者那里获得频繁且丰富的反馈,并有机会尽快运用这些反馈。

我在思考如何围绕难点建立强大的学习节奏。我记得强大的学习节奏是拥抱而不是轻视刻意练习,并伴随着持续的、以理解为中心的可操作评估、同伴互评和自评、互动式反馈、隐含评估、立即应用的机会,以及“拆分与重组”到完整游戏中的模式。

我在思考如何预见难点的到来。我可以尝试预见我的主题最有可能出现的“困难的知识”(仪式性知识、惰性知识、外来知识、缄默知识、技能性知识和概念性难题)模式,并以对抗这些模式的方式组织学习。

我在思考如何为我所教授的内容建立一个好的难度理论。在这里,我自己的经验是一个关键资源。我可以尝试不仅识别而且向自己解释学习者的具体困难,这将帮助我更聪明地教学,而不仅仅是更努力地教学。

我在思考如何不仅使用练习,而且使用“练习曲”。隔离难点的练习很重要,但我正在考虑设计“练习曲”,即选择完整的游戏来提供针对特定目标难点的练习。