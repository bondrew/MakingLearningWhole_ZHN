\chapter*{引言:一种全新的球赛}
我的棒球,是平庸的胜利。我并不特别擅长,但也不算烂。因为我并没有表现出运动方面的天赋,这才能算是一项成就。我打球并不差。击球调动着这个胖孩子奔跑于垒间,有时会被接杀,但有时也会得分。由于我接球无能而总被分到外野,我会稳定地丢掉飞向我方的球。

也许这种平庸听起来很糟糕,但我满意于自己能做的事。我喜欢棒球,它是我在夏日午后度过几个小时的十几种方式之一。在那之后的几年里,我得出了关于早期学习经历的一个奇怪结论:结果虽马马虎虎,但过程相当不错。

那这个过程是什么呢?我记得我父亲在我家后院教我打球。他向我展示如何站脚、如何握棒、如何挥棒。他说,盯准了球——熟悉的咒语!他用轻柔的下手投球,我努力掌握这一切。

一个夏天,我参加了小联盟棒球比赛。我不喜欢它的正式和繁琐。大多数人都像对待一场军事行动一样认真地对待整件事。我再次做到了:接球,击球,跑垒,站在球场,错过飞球。更让我怀念的是,我还记得在邻家后院的休闲比赛,七八个孩子,只有一两个垒,不需九局,有时甚至不需记分,只是单纯地玩。

为什么我说这个过程非常好呢?从直觉上来说,因为我喜欢玩耍和学习,故它是非常好的。从更分析的角度来看,因为从一开始我就对整个游戏有感觉,故它是非常好的。我知道击球或漏球会让你得到什么,我知道如何得分和记分,我知道我必须做什么才能做好——即使我只有部分时间做到了。我看到了它是如何组合在一起的。

这些听起来都很普通,但当想到正规学习很少让我们有机会从一开始就学习整体时,我简直惊呆了。当我和伙伴们学基础算术时,并不知道整个数学是怎么回事\footnote{也许你在想:好吧,你怎么能知道呢?你们只是孩子,而数学是一门精细的技术性学科。但我不确定做数学的基本训练是否需要微积分或代数甚至分数。}。或者,我在了解内战\footnote{指美国南北战争。}的事实后,却不知道人们是如何发现这些事实的,也不知道这些事实可以怎么用——比如,将之与其他时代和国家的内战比较\footnote{也许你在想:好吧,对于那些一开始对历史了解不多的年轻人来说,还能用什么其他方法开始呢。但我不确定人们是否必须以如此零碎的方式开始。}。

换句话说:我打棒球时,尽管多数时候我并没有打满四垒九局,但我正在玩一个非常适合的初级版游戏,这个初级版正适合我的体型、耐力以及邻里孩子的数量。但当我学习数学和历史时,并没有初级版本。这有点像在不知道整个比赛的情况下练习击球:为什么要做这个?

当然,我学习棒球的方法也有很多问题。一方面,棒球对我来说不是一场战役而只是一种消遣,但是真正认真地学习几乎任何东西都必须像一场战役。即便如此,那些阳光明媚的午后,青草和汗水的味道与手上廉价的皮手套仍萦绕在我脑海。今天我想:也许学习大多数东西都应该更像学习如何打棒球。

\section*{处理复杂}

有些学习很容易。你走进一个新的购物中心,很快就几乎自动地找到主要的地标:书店、百货店、电子产品商店、美食广场。我们自发地学习第一语言。完成这些任务所需时间巨大,但这一过程是如此程序化地融进人类天性,如此得到社会的支持,如此与日常生活交织在一起,以至于几乎不需刻意就完成。

然而,我们需要学习的很多东西都存在巨大挑战。棒球是一项复杂的运动,完全不像走进购物中心。基础算术或代数、阅读、理解文学、科学探索和科学世界观、历史理解及其与当今时代的相关性,也是如此。学术性较低的领域同样具有挑战性,如管理和领导力、与他人保持良好关系以及社会责任。

在所有这些情况下,正式或非正式教育都面临着一个最根本、最普遍的问题:处理复杂。教育的目的是帮助人们学习他们无法简单掌握的知识。教育总是要问,怎样才能使具有挑战性的知识和实践变得通俗易懂。

在学校和大学等正式学习环境中,这个问题变得尤为突出,因为那里有大量的人和大量的内容。以下是关于如何处理复杂性的两种最流行的答案:
\begin{enumerate}
\item 先基础(element first)。现在学习基础,以后再将它们组合在一起,逐步提高复杂性。
\item 先了解(learning about)。先了解一些事情以开始,而不是学习如何去做。
\end{enumerate}
我们依次来审视一下。从基础入手处理复杂性具有巨大的吸引力。它用于汽车生产效果相当不错,他们在流水线上装配传动系统、发动机和轮胎。它用于建造预制房屋的墙壁、窗户和屋顶,也非常有效。装配的逻辑是如此自然,以至于从幼儿园到企业培训,几乎所有的学习领域都能发现基础优先。学生们学习加、减、乘、除等算术基础,是为了最终有机会将它们组合在一起解决有意义的问题。学生们学习语法基础,是希望这些知识日后能用于凝聚成全面、令人信服,当然也是正确的书面和口头交流。

问题是,语法基础脱离整体没有多大意义,并且整体要到很久以后才出现。例如,学校要求青少年做的算术练习很少能为算术如何在日常生活中应用提供参考,并且早期几乎没有东西可以称为数学思维。再比如写作。我惊恐地发现,我的小儿子已经学会了写作的所有要素,但他的老师却很少要求他或班上的其他学生进行大量的扩展写作。这种在无视整体、或存在感微乎其微的情况下,靠要素基础来处理问题的趋势非常令人担忧,我喜欢把它命名为一种病:基础炎\footnote{原文是elementitis,-itis 是英文构成各类炎症的后缀。} 。

我记得有次在一个小组里分享这些想法时,一位女士举手提出了一个有趣的问题:“我有两个女儿,她们性格迥异。一个喜欢一头扎进去,另一个却喜欢一件一件来,做好充分准备后再尝试完整游戏。这难道不好吗?”

当然不好。基础炎并不意味着学习了一些基础,然后就能马上将它们融入完整游戏中。基础优先是一个很好的短期策略,但基础炎意味着周复一周、甚至年复一年地专注于基础却非整体。

基础炎并不是一种罕见的疾病。经验和确凿的证据都证明了它的普遍性。斯坦福大学的教育家Linda Darling Hammond在《The Right to Learn》一书中指出,狭隘的课程标准、臃肿的教科书以及覆盖面的压力导致了碎片式课程,每一个可以想到的话题都有15分钟的露脸机会。2007年,Wayne Au综合多方资料,在Educational Researcher杂志上报告了美国“不让一个孩子掉队”政策的影响是如何窄化和碎片化课程的。与考试无关的内容被丢弃,与考试相关的内容则被切成适合测试的小块。其实,大可不必。有些学校能更好地应对“不让一个孩子掉队”的挑战,有些州的考试更有意义。大可不必,但是趋势。

哈佛大学心理学家Ellen Langer把这样的教育称为无意识教育。数十年来,Langer一直坚持对“有意识”(mindfulness)和“无意识”(mindlessness)进行了丰富的研究,证明在许多普通情况下,人们会陷入盲目而狭隘的思维和行为模式,使原本可以更深思熟虑的事情变得一团糟。然而,人们可以培养一种更有意识的灵活姿态,对新信息持开放并注意多视角。在《The Power of Mindful Learning》一书中,Langer对教育界普遍倾向于无意识的学习模式提出了警告,并说明了如何才能避免这种情况。类似于“基础炎”的一个特别危害是认为必须将基础知识掌握得非常好以至于成为第二天性。另一个是延迟奖励文化,真正“玩完整游戏”的奖励总是在以后。

现在,让我们来考虑另一种几乎普遍采用的处理复杂性的策略,即通过了解某件事情来学习做这件事。阅读和数学一般都能避免这种情况,因为学生当然要学会做。但在历史和科学等学科的早期学习中,“了解”占据了主导地位。典型的历史教学被描述为学习“别人的事实”。这只是获取历史特定版本的信息,几乎没有经过深思的解释或批判性的视角。同样地,将对科学的代表性研究描述为学习“别人的理论”。学生们熟悉牛顿定律或有丝分裂的步骤,他们可以在测验或章节末尾的问题中表现出色。然而,大量关于科学理解的研究表明,学习者的理解其实非常有限,对这些想法的真正含义存在一系列误解。

一定量的了解是好的,就像一定量的基础一样;问题在于过度、无休止地学习,却永远无法提高“做”方面的能力。因此,与“基础炎”对应,我喜欢把“无休止地学习”称为“了解炎”\footnote{原文为aboutitis}。它能让学习者获得一些关于法国大革命和美国大革命、有丝分裂和减数分裂、行星位置、大陆漂移以及《奥赛罗》中种族和地位紧张关系的信息,但只是提供了一种信息背景,而不是增强和启发理解的主体。

“了解炎”的问题也不局限于早期教育。专业教育也深受“了解炎”之苦。例如,教师教育,教师们要上无数有关学习理论和课堂动态的课程,而用于在学校中扮演各种实践角色进行教学的时间却少得惊人。

如果说“基础炎”和“了解炎”对正式学习模式的定性似乎过于苛刻,那么我承认,即使是“基础炎”和“了解炎”在一定程度上也是有益的。在几乎一无所有的欠发达国家,传统的直截了当的教学可以产生相当大的影响。受过一定师范教育的教师、教室里的教科书、基本的识字和算术能力以及学科常识——所有这些都很重要。我们对“基础炎”和“了解炎”的抱怨并不是因为它们没有取得任何成果,而是因为我们本可以取得更多的成果。

自然我们要问:怎么办?处理复杂的问题非常现实。除了各个击破基础然后将其组合在一起或仅在相当长一段时间内学习它之外,还有什么选择可以处理复杂问题?还有什么可以做的吗?

一个诱人的答案已在手边,这就是初级版本\footnote{junior version}的概念。还记得我和伙伴们夏天下午在后院玩的那些更简单的棒球游戏吗:不是一个基础接着基础,也不是了解接着了解,而是参与其中的初级版本。从根本上说,这是一种不同的处理复杂性的方法,并且是一种本质上更强大的方法。它让学习者了解全局,从而使前进道路上的挑战变得更有意义。而且它还让学习者有机会发展积极参与其中的隐性知识,也就是我们所说的“感觉”或“诀窍”。

它提出了一种不同的教育和学习的思维方式。关于初级版本,稍后会有更多介绍,但让我们先进入正题。让我们以总结的形式来看看整个概念。

\section*{学习的七项原则}

如果学习大多数事物都能更像学习如何打棒球,或其他我们通常会整体学习的活动会怎样?大多数体育运动都是作为整体学习的。大多数游戏,如桥牌、跳棋、国际象棋或西洋双陆棋,也是如此。艺术也是如此:从一开始,学习者就花很多时间来创作整幅图画、绘画或诗歌。音乐表演也是如此:从一开始,就唱整首歌,演奏整首乐曲。因此,让我尝试按照学习如何打棒球、演奏乐器或描绘风景的精神,概括出一种良好学习的一般思维方式。

“一般”指的是几乎可以在任何地方、对任何人都有用的东西。我不只是在谈论课堂或教会团体或在职学习,它几乎可以应用于你所能想象到的任何事物的学习——相对论、滑冰、微积分、交朋友、商业管理、艾略特的诗歌、说中文、铺床或制作被子,因为大原则都是相同的。

我们可以将这种广阔的视野称为整体学习,并将其分为七个原则。我将在此列出这些原则并简要介绍它们,然后在后面的章节中更全面地探讨它们。
\begin{enumerate}
\item 玩完整的游戏(Play the whole game.)
\item 让游戏值得玩(Make the game worth playing.)
\item 攻克困难的部分(Work on the hard parts.)
\item 异地作战(Play out of town.)
\item 揭开隐藏的游戏( Uncover the hidden game.)
\item 向团队以及其他团队学习(Learn from the team . . . and the other teams.)
\item 学习学习的游戏(Learn the game of learning.)
\end{enumerate}

\subsection*{玩完整的游戏}
除了棒球,父亲还教了我跳棋。我们从整盘棋开始,我赢了第一盘。他简单地解释了规则,一边走一边提醒我,让我慢慢来。令人惊讶的是,我吃掉了他所有的棋子!

对我这个年轻天真的孩子来说有些太令人惊讶了。“你让我了吗?”父亲一如既往诚实的回答:“是的。”

“别啊!”我抱怨道。“好吧。”父亲回应道。他是一个相当自尊的人,也能理解我的自尊。从那以后的两三年里,在这种习惯逐渐消失之前,我俩时不时会玩一玩,但我再也没有赢过他!不过,我的棋艺还是有了长足的进步,而且也玩得很开心。我很享受学习整个游戏的过程,无关输赢。

我们可以问问自己,当我们开始学习什么东西时,尽早并经常参与到整个游戏的某个可接受版本中了吗?如果这样做了,我们就会获得所谓的“阈值体验 ”,这种学习体验会让我们摆脱最初的迷茫,会带我们进入游戏。从那时起,以一种有意义的方式向前迈进就更容易了。

很多正式教育都缺乏“阈值体验”\footnote{threshold experience}。这种感觉就像学习一幅拼图的碎片,但却永远无法拼在一起;或像学习了拼图的知识,但却无法触摸到拼图的碎片。与此相反,在开始的时候就接触整个游戏的某个版本是有意义的,因为它赋予了这项事业更多的意义。你或许不会做得很好,但至少你知道自己在做什么,以及为什么要这么做。

\subsection*{让游戏值得玩}
学校和其他学习场所要求我们做许多事情并不那么令人着迷。我们觉得自己恍若在玩学校游戏,但而不是真正游戏。我们学习倒数和乘法除分数的严格步骤,而这种数字翻筋斗的神秘动机却几乎无人理解......你照做就行了。我们背诵总统的日期或亨利八世的妻子时,我们练习用好的主题句来编写段落时,莫不如此。

时不时就会有咄咄逼人的学生提出这样的问题:“我们为啥要学这个?”从老师或者课本得到的答案几乎都是这样的:“后面你才需要知道。”“你考试需要”“这是本单元的目标”。

是什么会让一个游戏值得一玩呢?事实上,我们已经看到了一个最简单的因素:玩完整的游戏。倒数和乘法、记住名字和日期、练习段落结构,这些零碎只有在整个游戏中才有意义。若非在初级阶段就经常玩数学思维、历史理解或议论性和表达性写作的完整游戏,它们将毫无意义。玩完整游戏可以阐明游戏的价值,因为你可以立即看到事物是如何组合在一起的。

可以肯定的是,对大多数学习者而言,有些完整游戏也不是那么有趣,而且没有人会对所有事情都感兴趣。即便如此,完整游戏仍然有帮助,而巧妙的老师会使用许多其他方法将学习者与某个主题的有趣之处联系起来。一个主题的全部重要性并不总是显而易见的。即便如此,还有许多诚实的方法可以预览某件事的重要性,而不仅仅是说“你后面才需要知道它”。

\subsection*{攻克难点}
多年来,我的父母经常和另一对夫妇打桥牌。最后,我也学会了,有时也会加入其中;有时和妻子与父母玩。直到那时我才意识到,我父母并没有变得更擅长:他们尽管一直打,却没有在打中学。

想想你做了多年的事情,你会发现自己并没有变得更擅长。缺失的因素通常就是我们的第三条原则:攻克难点。在学习的最初阶段,这并不重要,此时重要的是熟悉整体。然而,随着学习者逐渐适应活动模式,难点开始出现。

难点有一个恼人的特点:并不是玩完整个游戏就一定会有进步。真正的进步源于对游戏的分解,找出难点部分给予特别注意并练习他们,制定能更好处理它们的策略并将它们快速重新融入整个游戏中。击球练习!

正常的学校教育包括对难点的重要练习。这很好。但这种作业通常不够多,而且没有个体针对性。回想我从幼儿园到大学的求学经历,我竟然很少有机会复习巩固重难点。交卷后我会收到批改,上面写着 "95\%"、"70\%"、"不错的观点"、"需要进一步的证据"——没有足够的信息来有效地诊断出难点到底在哪里,也没有机会进行调整,因为我们已经在转到下一个话题了。

\subsection*{异地作战}
回到棒球。棒球比赛存在主场优势现象。Boston Red Sox在波士顿Fenway Park进行比赛时,他们不仅能得到热情观众的支持,还熟悉球场的一些特殊情况。任何运动都有主场优势,但棒球运动的主场优势尤为显著,因为全国各地的球场都有自己独特的布局。

主场优势的反面是客场劣势。Boston Red Sox在外地比赛时,它是一个问题,但也是一个学习的机会。新的环境对球员们提出了挑战,要求他们拓展和调整自己的技能和洞察。他们能找出更好利用不同环境的方法,或许还能泛化学到的东西,让之后的客场比赛变得不再那么不利。

不同的环境有那么重要吗?这一点在不同的项目中差别很大。对于在高度标准化的室内球场进行的运动来说,这一点最不重要。相反,足球比赛通常会给客队带来他们不习惯天气,比如暴风雪。在网球比赛中,草地球场、红土球场和硬地球场之间的差异会极大地影响谁最有机会胜出。探险这项极限团队运动则会故意将小团队安排在他们不熟悉的荒野,他们需要在指定站点之间自行确定路线,以尽可能快的速度在危险的地形上完成跋涉。我的同事Daniel Wilson对探险比赛进行了系统的研究,揭示了团队成员在比赛中处置和学习时的异常复杂和多变的互动。探险者总是在城外比赛!

体育运动的异地现象在不同程度上适用于学习任何东西。正式教育的全部意义在于为其他时间和地点做准备,而不仅是为了在课堂上取得好成绩。今天所学不是为了今天,而是为了后天。有时候后天和今天差不多,但更多时候并不同。

但问题是在正式教育中,通常没人会让我们出城,去玩、去拓展我们的阅历。数学的思想和算法是非常普适的,但实践中学生们只关注火车、帆船或买苹果的一些刻板练习。良好公民的理念是非常普适的,但实际上学生们只关注关于投票或社区服务的几个故事。甚至,连走廊对面的教室都可能离得太远。有一句来自于一位高中科学教师的关于学习的调侃我最喜欢,多年来一直铭记在心。他在抱怨他的学生在将数学应用于科学时遇到的麻烦时说:“似乎学生们从数学教室走到科学教室,就把数学给忘了”。

研究人员称之为学习迁移问题。出城也能玩好并不会自动发生,它像学习的其他方面一样,需要我们的努力。

\subsection*{揭开隐藏的游戏}
在网上搜“棒球的隐藏游戏”,最先出现的结果之一是1984年John Thorn和Pete Palmer合著的《The Hidden Game of Baseball》。在很多人眼里,棒球和数学可能并不属于同一范畴,但《The Hidden Game of Baseball》却将两者放在了一起。这本书从统计学的角度阐述了棒球:棒球比赛和整个赛季为什么会以这样的方式进行,以及聪明的策略是什么样的。

棒球如此,几乎任何事情都如此——文学评论、新交并维系朋友、数学建模、炒股、缔造和平、发动战争、创造艺术——都有隐藏的游戏。事实上,“该”隐藏游戏令事情被低估。任何复杂而有挑战性的活动,都在显而易见的表象下有着很多层次。棒球和物理学都有多个侧面:统计、战略,甚至政治。棒球中也有非常有趣的物理学,尽管我不确定物理学中是否有棒球。

隐藏游戏是有趣轭,且往往对做好表面游戏非常重要。教练和经理必须关注打击和投球方面的统计趋势并把握机会。在下棋时,必须注意广泛的战略考虑,如控制中心。在学习科学概念时,对各种科学理论所涉及的因果关系的基本原理有一定的了解也很重要,它们往往与日常的因果关系概念大相径庭,如果不了解其中的隐藏游戏,很可能会产生误解。

很多学习都是在仿佛没有隐藏游戏的情况下进行的,但它就在那。学习者要关注隐藏游戏,否则将永远只是在表面上滑行。

\subsection*{向团队以及其他团队学习}
己的事情自己做!如果有学生行为十诫的话,这句话可以排在第一位。从一般做法来看,这句话不错,但从社会运作的角度来看,这句话就很怪了。我们几乎很少有事情是独自完成的,无论你是运动员、商人、科学家、垃圾归集员还是文员,你几乎总是以一种复杂的方式与其他人协作。人类的事业是深度且本质的集体性的,但学校除外。

这就是为什么在学习的七项原则中,我们会凝练出“向团队的人以及其他的团队学习”。实际上,要想从单一来源、被动文本以及除了你还需要关注其他许多人的老师那里学好是非常困难的。人练要好得多,但多数人负担不起,大多数社会也负担不起为大规模学习过程提供私教!即使是私教,也只能告诉你在团队中与他人协调所需的艺术和技巧,而不能替你做。

可以肯定的是,有些活动比其他活动更适合独自进行。把阅读变成集体活动容易,但把写作变成集体活动却很难——尽管也能做到。向团队成员和其他团队学习这一原则应做广义的解读,该原则不仅涉及自然具有群体特征的活动,还涉及向从事同样追求的其他人学习,如朋友、同伴、同事、对手、敌人、模范、导师,甚至不如己者。

\subsection*{学习学习的游戏}
许多人学习第二语言,有些人则学习第三语言。学习第三语言与学习第二语言的经历截然不同,这很有意思。学习母语之外的任一语言都是非常有挑战性的,但学习第三语言通常不会像学习第二语言那样令人生畏。在学习第二语言的过程中,你对语法的组织方式有了更好的理解,更容易理解第三语言的语法。记忆词汇和句法结构的节奏也熟悉了:除了第二语言本身,你还学到了一些关于如何学习语言的知识。

学习学习是一种比学会学习语言更为普遍的现象。即使是非人类的哺乳动物也会以一种基本的方式学习学习,习惯训练过程的节奏,并经常参与其中。学会学习与很多方面有关:引导注意力、选择时间和地点、将新的想法和技能与已有知识联系起来。事实上,它与前六项原则有很大关系。自我管理能力强的学习者即使在没有教练或老师强制要求的情况下,也会坚持练习难学的部分。自我管理能力强学习者意识到异地作战的意义——把想法和技能与其他情境联系起来——尽管没有教练或导师送他一程。

我想不出还有什么比学习学习更值得学习的了,它就像存在银行里以复利计息的钱。遗憾的是,大多数学习环境很少直接关注对学习的学习。

\subsection*{顺序的问题}
七项原则的顺序有什么特殊意义吗?前面的原则并不比后面的更重要,您也不需要关注原则的序号。例如,有时一个主题倾向于在早期发现隐藏游戏的某些特征(\#5),或向团队学习(\#6)。

把整体学习排在最前,因为它是中心思想。把学会学习放在最后并不是因为它是最后要解决的问题,而是因为"学会学习"是一个跨越特定主题的上位主题。在这两者之间的其他主题的顺序只是为了便于叙述。如果你想以不同的顺序来思考这些原则,完全没问题。

\section*{是的,但是……}

我希望这一切都有意义,我希望这与人们记忆中许多好的、坏的和中等的校内外学习经历相吻合。我希望你们也能回忆起学习打棒球或其他自己喜欢的运动或游戏时的感受。我希望你们也能回忆起发展某项虽不一定擅长、但也能掌握诀窍的艺术或手艺的过程。我希望你们也能回忆起那种只学棋子而不了解整个棋局的空洞的“基础炎”,以及那种无休止地学习某样东西却从未动手去做的“了解炎”。

即便如此,整体学习似乎仍是一项理想主义的事业,就像Matterhorn峰的山顶一样遥不可及。下面我们简单地反驳一些保留意见。

由于数学、历史和科学的结构比棒球、桥牌和羽毛球(baseball,bridge和badminton,统称3B)要松散得多,自然会有“是的,但是……”的说法。3B都是有规则的游戏,但数学、历史和科学的"整体游戏"是什么呢?在其中做特定的游戏,比如寻求数学证明、收集和评估历史证据,或者设计和运行一个实验,又是什么呢?我在本书的后文的部分任务就是让你相信,整体游戏隐喻都指向有用的方向。尽管学术学科少有严格规则,但还是有一些经验法则、指导原则、常规做法、典型形式、惯用策略等来辅助定义。

另一种自然的反对意见是认为有些学科像金字塔,例如数学,不可能在底部建好前就建造顶部。只有在建立起一些基本事实和例程的基础上才能上升到理解和创造性解决问题的高度。反对者说,这里没有初级版本。本书后文中我的部分任务就是要说明,总有一个初级版本。我们不否认金字塔的现实性,但很多学科中都有合理的、生动的初级版本供初学者使用。

此外,我们不该只对学科学习感兴趣,还有许多其他类型的学习同样重要,例如,领导技能和态度、人际关系、道德决策和公民意识。与学科学习一样,尽管在这些领域没有严格的规则,但都有一些准则、惯例、策略等有助于框定何为“玩游戏”。

已有许多关于学习理论及其与教育的联系的好资料,例如,Bransford、Brown和Cocking合著的《How People Learn》。有人可能会问:“我们真的需要另一种学习理论吗?我们已经有了行为主义、建构主义和人类毕生发展等关于学习的学术观点”。

问得好!一些好消息:整体学习并不是一种与其他理论竞争的学习理论。整体学习是一种教学理论,或者更广泛地说,是一种教育理论。学习是一个比教育更广泛的范畴:在闲聊中、超市里、大街上、玩射击电子游戏时、研究股票市场投资时,学习都自然发生;而教育是学习的特定编排,是为了更及时、更集中、更有效和更高效地组织学习。这就是“整体学习”的作用。

整体学习融合了多种学习理论,提供了一个设计框架。整体学习是一种整合性方法,它将为实现好的教育而提出的学习理论的许多关键特征牢记于心并付诸行动,有时也可称为行动理论。本书后文中我的任务之一就是展示"整体学习"七项原则背后的学习科学。

现在开始。在不对行为主义、建构主义或任何其他学习流派作过多回顾的情况下,让我来大致勾勒一下整体学习与它们的关系。整体学习的基调不是行为主义的,尤其原旨行为主义——否认存在想法和意图。整体学习将学习者视为觉醒且积极并会变得更好的人。

不过,整体学习确实与行为主义有相同的观点:当反馈是即时且有信息量时,以及当围绕某一努力的激励结构是积极而非深层威胁时,事情就会进展得更好。

整体学习是非常建构主义的,它认为学习者总是在某种意义上从学习经验中建构自己的意义。事实上,整体学习是在一般性建构主义的骨头上添上血肉的一种方式。发现式学习和探究式学习可以理解为建构主义的特殊表现形式。下文中的一些例子就具有发现式学习和探究式学习的味道。

然而,整体学习并非是说所有的学习都该以激进的发现为导向。什么适合某个特定的主题,是需要判断的。在多数体育运动和游戏等为代表的很多情况中,当开始学习的最佳方式是清楚地解释和演示时,我们会让学习者尝试和再尝试,并指导他们完成一个改进的过程。这与偶尔给点提示,让他们自己摸索的做法大相径庭。

一般来说,从发展的角度看,学习强调的是人们对学习变化的准备 。长期以来,儿童和成人都发展了广泛的认知能力、知识观和理解方式,这些能促使更有力的思考和学习。在发展轨迹上走得更远的学习者,其理解地学习某个特定想法或主题的"发展准备程度"可能要高得多。此外,生理年龄相同的学习者,其发展年龄也不一定相同。精心设计的学习方式可以照顾到同一群体中不同的准备程度。

那么,整体学习是如何匹配的呢?整体学习当然把对发展准备情况保持敏感视为一般事项。整体学习不强调一种特定的发展模式,因为人类发展领域是如此复杂,其本身就是一个完整的故事。下一章和第5章末尾会出现一些关于发展的进一步观点。

最后,说两句技术。如果运用得当,当代信息技术将为学习提供强有力的方法。技术可以为学生带来他们无法接触到的整个游戏。例如,计算机模拟、在线研究工具和电子邮件交流可以帮助学习者进行合作探究或对难题的有思考的批判性讨论。下面将举例说明。

然而,整体学习不是必须的要用这些技术。许多社会模拟根本不需要计算机,只需面对面的角色扮演。具有整体游戏特征的正式的面对面辩论要比互联网和论坛早数千年。

总之,整体学习与其说是提供了一种新的学习理论,不如说是与当代关于学习和教学的许多观点相吻合的一种综合行动理论;当然还有其他组织学习的行动理论。你可以根据自己的喜好来选择,或者选出对你最有帮助的部分。

为了帮助你思考这个问题,请记住这一点:学习的设计框架的趋势是在不抱怨主题本身如何原子化(例如分数除法、总统日期、牛顿第三定律)的条件下处理任何主题。与此相反,整体学习不仅强调如何进行学习,而且强调学习的正确单位是什么,即有意义的整体。整体学习反对基础学习,反对大量堆积的了解学习,因为最终的目的是学会做事情。在本书的其余部分中,我的任务之一就是反复论证这种强调整体性、始终适当关注"难点"的方法才是最有效的方法。

还有一种截然不同的“是的,但是……”值得关注:有时人们会对游戏这个隐喻感到不安。

有人担心,“游戏”用于莎士比亚的戏剧、国家的建立或人类的生物起源等严肃的问题来说过于轻率。另一种担心是竞争这个词。大多数体育运动和游戏都涉及个人或团队之间的竞争,而成绩和考试的竞争可能会带来的坏处大于好处。

我部分同意上述两点。我希望学习整个游戏这一隐喻听起来不轻率,因为我有时认为我们对待整个教育事业的态度过于严肃了,应该轻松一些。我也希望竞争的内涵能够柔和一些,因为我认为在精心选择的情况下,某些温和的竞争是有助于学习的。

没有隐喻是完美的。无论我们属于哪个民族,当说到“我们民族的父亲”(或母亲)时,并不是在任何时候都恰当。当我们像17世纪英国诗人兼传教士John Donne一样说“\textit{没有人是一座孤岛,可以自全。每个人都是大陆的一片,整体的一部分。}”时,我们承认了一个生动而重要的事实,同时也将人类自主性的一些复杂性推到了幕后。隐喻就像地毯:它揭示了一种引人注目的模式,而复杂的绒毛则被扫到了地毯下面。

在权衡后,我认为对隐喻本身的担忧并不能完全抵消它的整合力量,我们也能够提防它的负面影响。如果你愿意,可以从字面上理解这七条原则,而不必在意游戏隐喻。它们听起来可能是这样的:
\begin{enumerate}
\item 参与某种形式的整体活动,而不仅仅是零碎的活动。
\item 让活动值得追求。
\item 克服难点(至少这一点听起来是一样的)。
\item 探索活动的不同版本和设置。
\end{enumerate}

如此如此。我希望你能读下去以了解其余的故事。在阅读的过程中,如果你扮演着教育者的角色——教师、导师、教练、家长,甚至是正管理自己学习的学生——我希望你能尝试一些事情。你可能想构建自己的初级版"整体学习",而不是一次尝试所有的东西!只需关注基本原则,而不必拘于细节。只需掌握其中的两三条原则,并以简单的方式将其付诸实施。

事实上,你甚至无需深入研究一个原则的细节,就能用它做很多事情。我发现,在我说出这七条原则的名称后,无需我做过多的提示,它们就激发了人们根据自己的经验的阐述。每一章的末尾,都有对主要观点的方框式总结的"学习的思虑"。它以第一人称写成,就像你在思考问题一样,通过一系列“我考虑……”的问题,引出本章的简短答案。请将这些问题和其他类似问题牢记在心,并在实际教学中提出和回答这些问题。

在经过一些路试之后,如果再翻开这本书,您可能会发现许多细节更加有意义。为了将所有内容归纳到一起,后记中提供了一些关于以整体学习和教学的经验,哪些原则要尽早强调以及为什么,整体学习和教学的技巧是如何随着时间的推移而形成的,在一个复杂的全球化的且不断变化的世界中的教育所面临的挑战等。请记住,我们也是学习者,精心选择的初级版本的力量既适用于我们自己的教学学习,也适用于他人学习我们希望教给他们的东西。

\section*{论硕果累累的平庸}
还有一个疑问值得在此讨论:既然整体学习如此强大,那我为什么不擅长棒球呢?事实上,既然人们通常以整体学习的方式学习体育、游戏、艺术和手艺,那么为什么大多数人并不擅长这些呢?

当然,还有天赋因素。我并不是特别擅长体育运动,但这并不是问题的核心。除了玩整个游戏外,还有六项整体学习原则。这些原则并不总是对我有利。下面是一张记分卡,附带一点解释。
\begin{itemize}
\item[√] Play the whole game.
\item[√] Make the game worth playing.
\item[×] Work on the hard parts.
\item[×] Play out of town.
\item[×] Uncover the hidden game.
\item[×] Learn from the team . . . and the other teams. 
\item[×] Learn the game of learning.
\end{itemize}

我玩了整个棒球游戏,觉得整个游戏都值得一玩。我玩的也不仅是初级版本,在小联盟的暑期比赛以及学校课间休息和体育课上的许多比赛几乎都是完整版的。不过,除了小联盟的暑假和父亲早期给我的一些提示外,没有人让我练习困难的部分。我对打棒球也不够认真,没有自己练习过困难的部分。至于出城打球,我们没有出过,只是在学校和邻居家和孩子们打打闹闹。直到我长大了很多,才有人告诉我关于隐藏比赛的事情。向球队学习?只是偶然。我们当然没有互相学习或互相指导。学习关于学习的游戏,则根本没有出现。

如果早知道这神奇七原则,我就能比现在更神奇地学会打棒球。好的整体学习远不止经常玩整个游戏。就像"基础炎"和"了解炎"提供了一种过度还原的方法,只玩整场比赛的表面版本是一种过度全面的方法。人们之所以在许多运动和游戏、艺术和手艺以及专业工作中表现平平,就是因为他们花了太多时间玩整个游戏,而没有将其他六项原则付诸实践。

也许我们应该承认,即使这种平庸也有一定的价值,至少它实现了对整个游戏的普遍认识和参与。当然,我虽没有成为棒球高手,但至少我学会了做一些有意义的事情,并在做这件事时变得更好了一些。我对自己有限的技能感到相当满意,而且有能力偶尔打打球,了解棒球的话题,关注电视上的比赛,几十年后还能和自己的孩子一起玩后院棒球。这很有价值!

有关教育的许多言论都强调卓越,卓越也的确是我们追求的目标。想象一个世界,在这个世界中,几乎每个成年人都有一种充满活力但简单的公民参与感、生态责任感或避免偏见的意识。从今天的冷漠和无视出发,这些"游戏"不一定要玩得非常复杂才能产生实质性的益处!如果在这些领域的大多数人都能主动的平庸,而不是被动的精明,那么这个世界将会变得更加美好。

在接下来的章节中我们将更详细地探讨如何进行整个游戏,接着逐个讨论其他六个原则,以更好地理解学习的作用机制,并使学习效果更好。