\chapter{向团队学习}

我从未打过一个人的棒球,其他人也没有,只有超人才能做到:他从投手丘上奋力投球,直奔本垒,球在空中飞行的同时,他以超音速冲回本垒,拿起球棒,挥棒击球。我们假设他击中了,毕竟他是超人。球高高飞起,超人则冲向外野——考虑到他那势大力沉的一击,球很可能飞得非常远。我们假设他赶在球落地之前到达了北极并接住了球。他把自己累坏了。

这种将棒球描绘成独角戏的场景固然离奇,但在许多群体学习环境中,例如学校、专业讲习班和宗教指导中,却经常发生类似的事情。学习者以“单人”模式运作,人们被要求独自学习,即使其他人就坐在他们旁边。他们的大部分时间都花在阅读和倾听而不是互动上,他们不被鼓励去看其他学习者的作品,不能评论或提供帮助。

在大多数非正式环境中,“游戏”的方式可能截然不同:当人们学打棒球时,他们通常一起学习,互相观察、互相帮助;当人们学习各种纸牌游戏时,他们积极地的互动、互相学习;当你养第一只宠物时,你可能会从朋友、叔叔和宠物店老板那里获得建议;当你学骑自行车时,很可能有人在你旁边跑着,帮你保持平衡。概括而言,学习将知识、技能和理解力的构建视为一项集体事业。这引出了我们整体学习七项原则中的另一项:向团队学习……以及其他团队。

需要明确的是,当你“向团队学习”时,所学内容不必一定是团队活动,也不必具有竞争性。“向团队学习”只是一个宽泛的比喻。人们可以从“团队”中学习通常是单独进行的活动,例如如何骑自行车或如何照顾宠物,“团队”可以是任何由朋友和邻居组成的组合,他们可以在学习过程中不时提供一些帮助。

而且并非只有身边的人才能提供帮助。在很多情况下,提供帮助的是“其他团队”:可以观察的陌生人,有时是可以与之竞争以挑战个人最佳表现的对手,有时甚至是那些没能很好处理好自己手头事情的人。他们不仅可以从自己的错误中学习,其他人也可以从他们的错误中学习!有一个非常有用的概念可以使这一切变得有序。这就是“参与结构”的概念。这个短语为我们提供了一个名称,用来描述活动如何通过角色和责任来组织。

让我们思考一下传统课堂的参与结构。教师和课本是信息的主要来源,学生提供的信息很少。教师和课本也是活动的主要来源。互动主要发生在每个学生和教师之间,教师评估作业并提供反馈。学生之间的交流很微弱,只是在偶尔的全班讨论中听到其他人说什么。这是一种参与结构,但它是一种稀疏的结构,主要强调学习者与教师之间的关系,而没有充分利用教室中其他潜在的关系。

相比之下,更丰富的参与结构出现在许多课堂内外。在工作室学习中,艺术专业的学生每天都能看到其他学生在做什么,并进行交流。在一种称为“结对问题解决”的策略中,学习者两人一组工作,互相帮助思考他们的问题解决过程。在实践社群中,专业人士通过互相学习技能来提升自己的技能,通常以非常非正式的方式进行。在指导关系中,导师与学习者进行一对一的互动,以进行双向的细致的信息和思想交流。所有这些都是不同的“向团队学习”的参与结构。

那么,向团队学习如何帮助提升知识、理解力和技能呢?一种看待它的方式是参与结构如何服务于整体学习的其他原则:
\begin{enumerate}
    \item 玩完整的游戏。通常刚接触一项活动或一个知识领域的学习者无法独自进行完整的游戏。但在其他更有经验的人的帮助下,他们可以参与进来。最初是在边缘参与,然后逐渐走向中心。
    \item 使游戏值得玩。群体中的社会互动和承担特定角色的责任会产生投入感。
    \item 攻克难点。其他参与者可以成为关于如何处理难点的宝贵信息来源,无论是通过学习者观察他们的行为,还是通过他们的直接建议和指导。
    \item 异地作战。仅仅与他人一起工作就提供了一点“异地作战”的机会,因为人们处理事情的方式不同,有时差异很大。
    \item 揭示隐藏的游戏。通过观察他人并在自己没有完全参与游戏的情况下倾听他人的意见,人们有时可以更容易地看到正在运作的“隐藏游戏”。与他人的讨论通常集中在策略的“隐藏游戏”上。
    \item 学习学习的游戏。本章和直接探讨这一主题的下一章都举例说明了参与结构如何促进对自身学习过程的理解和管理。
\end{enumerate}

当然,细节决定成败。如果新手不受欢迎,社会环境就无法帮助他们进行完整的游戏。如果竞争和背后诽谤压倒了友情和合作,就会使游戏不值得进行。除非明确的角色和责任将他们吸引进来并保持他们的活跃,否则这无助于倾向于偷懒的学习者。我们需要一些技巧和方法来充分利用“向团队学习……以及其他团队”。

\section*{学习的社会学视角}

学习的社会学视角有许多分支,但其源头无疑是第五章提到的俄罗斯心理学家Lev Vygotsky 。Vygotsky探讨了认知与社会、作为象征工具的语言、语言如何支持思维、社会互动如何促进认知发展等主题。他最广为人知的概念之一是我们已经赞扬过的“最近发展区”,即人们通过社会支持在自身能力范围之外进行学习的思想。这引出了Vygotsky的另一个洞见——“社会脚手架”。

思维和学习总是在社会文化背景下发展的。我们可能认为问题解决很大程度上是“在头脑中”进行的,但它通常会从人际互动(谈话、指导、评论)、群体活动(团队合作、专业组织、项目)和文化产物(语言、计算机、写作、书桌、书籍)中获得大量支持。学者们以“活动理论”之名,分析了人类事业如何在社会文化活动系统中展开。例如,回想一下围绕育儿、管道维修或打台球的高度社会化活动群。从发展角度来看,内部的思维活动最好被视为最初在谈话和身体行动中更为外显的过程的内化。而且,即使是相对内在化的思维活动,仍然依赖于社会传播的工具,例如语言。

让我们选择一个明显的抽象例子来强调这一点。数学思维和学习听起来像是完全在头脑中进行的个人活动。然而,它涉及到融入数学话语的高度社会化背景。什么才算是一个有趣的问题,很大程度上是由社群定义的,思想在社群中通过讨论、合作和评论而蓬勃发展或消亡。这一切都没有否认数学家们会花相当多的时间在角落里用铅笔和一张纸(或者现在,可能是像 Mathematica 这样的计算机计算平台)进行思考,但这些单独的活动都位于更大的社会努力网络中。

“情境认知”已成为一个关键短语,表示有效的思想和行动如何依赖于适应和利用特定的社会文化环境。同样,“情境学习”已成为一个关键短语,表示有意义的学习需要真实的社会努力背景。“情境”视角对传统学校教育提出了尖锐的批评。正如前面提到的,教师和学习者之间常规的互动模式与良好的学徒制或家庭或工作环境中的指导关系相比,并没有真正提供非常有力的支持结构。学校学习通常位于真实的社会文化实践中。学校数学与真正的数学探究关系不大。它只是一堆孤立的技能,又是“基础炎”作祟。同样,学校版本的历史通常与历史理解或历史探究关系不大。它只是一个信息袋——姓名、日期和关于事情是如何发生的官方说法。学习者没有进行完整的“游戏”。

有时,对情境学习的倡导似乎走得太远了。有时,社会文化视角的狂热拥护者似乎在强调,普遍而非情境化的知识本质上是无用的,并且抽象符号系统或一般的思考或学习技能毫无用武之地。关于这场辩论的一个有趣的焦点是Jean Lave在她的《实践中的认知》一书中最初讲述的著名的“农家奶酪”的故事。

作为情境认知的敏锐观察者,Lave讨论了一个“体重观察者”小组的成员如何解决“找出四分之三的二分之三杯农家奶酪”的问题。这个人量出三分之二杯,把它拍成一个圆形,像披萨一样用两条线交叉切开,然后拿走四分之一。这个解决方案利用了当时环境的特殊条件——手边的杯子、农家奶酪的柔软特性——而无需记住繁琐的分数算术细节。这与通用的数学方法形成对比:3/4 $times$ 2/3,分子和分母相乘得到 6/12,化简为 1/2,所以取半杯。

情境观点的支持者赞扬了该解决方案巧妙的语境转换,强调了它如何展示了适应性行为的微妙特殊性。这挑战了我们关于非常通用和抽象的符号系统的效用的固有假设。反对者则反驳说,当你恰好没有处理农家奶酪时该怎么办;分数乘法可以处理任何情况,无论物质多么顽固。

最热情的倡导者也倾向于将文化濡化理想化为一种学习模式。毫无疑问,它可以非常有效。然而,它也可能使人们陷入狭隘和恶性思维的模式。偏见和歧视的发展显然是一个高度社会化的学习过程。社会学习也可能强化平庸,例如,当劳动力成员被群体“教导”不要工作太努力,以免最终暴露人们的真正生产力有多高时。

就我个人而言,我想兼而有之。我想要利用情境的特殊条件进行巧妙适应的能力。但我也想要算术的普遍适用性。我当然想要社会和文化情境学习的力量。但我们需要承认,有时它会滋生偏见和平庸。也许最根本的是一种讽刺:最强形式的情境学习的优势……是情境化的!

关键仍然是,从社会文化的角度来看,相当多的传统学习非常贫乏。它既没有利用社群来推动学习事业,也没有促进在学科或其他社会文化事业中有意义地进行情境化的技能和理解力的发展。当然,这是最糟糕的情况。在许多环境中的许多方面,情况可能会好得多。让我们考察一些参与结构,这些结构可以为了学习的目的而利用我们的社会和文化天性。

\section*{结对问题解决}

这里有一道思考题:

Bill, Judy和Sally 的职业分别是教师、水管工和搬运工,但不一定是按这个顺序。Bill比Judy矮,但比Sally高。水管工最高,搬运工最矮。Judy的职业是什么?

这个问题是一项旨在培养系统推理及其管理的练习。然而,这个问题的情境并非是单人进行——你当然可以自己解决,但理想情况下,应该有一个解题者和一个倾听者。以下是活动可能展开的一种方式:

解题者开始使用不等号来记录一些关系。倾听者提示:“好的。所以你在这里使用不等号?”

这使得解题者对写下的不等式进行了更详细的解释。稍后,解题者沉默不语,倾听者问道:“你现在在想什么?”

解题者断言Judy是水管工,并说他打算在“搬运工”和“水管工”之间写上“教师”。

“为什么?”倾听者问道,即使解题者似乎已经解决了问题。倾听者坚持要听到详细的理由。

这是对Jack Lochhead的《回想》(Thinkback)的一个简短摘录,它展示了“大声思考结对问题解决法”(简称结对问题解决法)。这项广泛使用的社会学习技术由 Arthur Whimbey和Jack Lochhead于 1979 年在他们的著作《问题解决与理解》(Problem Solving and Comprehension)中提出。这是一种让学习者互相帮助的方式,但带有一个奇特的转折。其理念与其说是让学生提供直接帮助,不如说是帮助他人意识到他们自己的思维和学习过程。

结对问题解决策略适用于许多类型的学习。在《回想》中, Lochhead提供了来自谜题、处理图形组织者、大声写作、记忆工作、概念图和“知识即设计”的例子。Lochhead在他和Whimbey多年前首次提出的经典结对问题解决法的基础上进行了一次改进。他将“回想”描述为使用“增加了视频回放的视角”的大声思考配对问题解决策略,其理念是利用结对问题解决法来创造一种工作中的大脑的心理电影。结果是一个思维图像。人们敦促学生将这些思维图像记在脑海中,以此来学习关于思维和学习的知识并改进过程。

那么,结对问题解决法是如何运作的呢?我们开头的例子说明了基本模式:学习者配成对,一个扮演解题者的角色,另一个扮演倾听者的角色。手头有一个问题,解题者着手解决,并在过程中大声思考。倾听者倾听的目的是保持对解题者过程的清晰了解。当解题者沉默不语时,倾听者会提示提供信息。当解题者采取倾听者没有完全理解的行动时,倾听者会要求解释。解决第一个问题后,两人互换角色,解题者变成倾听者,倾听者变成解题者。

Bill-Judy-Sally问题中的互动演示了倾听者不提供建议而是提示解释的做法。提供建议的诱惑非常大;人们会随着时间的推移而变得更擅长结对问题解决,而“不提供建议”是他们从倾听者角色中学到的最重要的事情之一。在解题者方面,许多人立即就能进行大声思考,但有些人起初会觉得别扭。一个关键的告诫是避免“隐形思考”,也就是说,不要让自己沉默不语。这项技能也会随着时间的推移而发展。在这个过程中,学习者对他们的思维方式获得了更多的视角和管理能力。

但是,我们为什么需要倾听者呢?为什么不简单地让解题者坐下来,大声地自言自语地解决问题,以更好地掌握这个过程呢?
一方面,如果没有倾听者的提示,“隐形思考”是一个很大的陷阱。解题者往往会沉浸在问题中,最终很少关注他们自己的思维策略。另一方面,仅仅因为你大声思考并不意味着你在向自己解释。倾听者的工作不仅是让解题者说话,还要让他们解释:“为什么要这样做?”“你想要从中得到什么?”“你得到了你想要的吗?”“所以你改变了方向,你为什么要这样做?”

让我们从整体学习的角度来审视结对问题解决法:除了“向团队学习”之外,它如何促进整体学习的六项原则?

结对问题解决法不太适合完整地进行“玩完整的游戏”,因为完整的游戏通常会在相当长的时间内展开,并涉及许多不同类型的社会互动。然而,结对问题解决法非常适合“攻克难点”。这是一种专注于学习领域中具有挑战性的方面的方法,倾听者的认知镜像提供了一种即时反馈,否则这将非常难以设置。此外,当人们在攻克难点时,有一个同伴一起奋斗,而且这个同伴很快就会与你互换角色以保持公平,这也有助于“让游戏值得玩”。

至于“揭示隐藏的游戏”,倾听者的问题和解题者的回答创造了一种认知镜像,向双方揭示了解题者的过程。他们的伙伴关系和角色互换增加了“异地作战”的机会,因为每个人都进入了对方的头脑。

最后,整个活动都支持“学习学习的游戏”:随着学习者变得熟练于结对问题解决,他们会变得更加自觉和自我管理。因此,结对问题解决的参与结构服务于整体学习的多重目标。

\section*{工作室学习}

最近我和一位朋友谈论了工作室学习。她颇有热情地宣布,工作室学习最重要的特征之一是不要‘手形火鸡’!

你可能还记得“手形火鸡”。几乎所有美国学校的学生都曾画过它们。感恩节来临时,“火鸡”既是课堂上的主题,也是餐桌上的佳肴。

那么,如何画一只“手形火鸡”呢?一种方法是以适当的角度将你的手放在一张纸上,然后沿着它的轮廓描摹,你就能得到火鸡的大部分形状,并且可以轻松地填上头部和几只脚。这很有趣,但并不是我们对艺术的期望。这是一个很好的座右铭——“不要‘手形火鸡’!”

“工作室学习”这个名称本身听起来很小众,似乎仅适用于艺术。诚然,工作室学习在艺术领域找到了其天然的归宿,但并非仅限于此。最近,我的同事Lois Hetland, Ellen Winner, Shirley Veenema和Kimberly M. Sheridan完成了一项关于中学阶段高质量工作室学习的研究。他们的观察和发现发表为《工作室思考:艺术教育的真正益处》。我被他们揭示的教学和学习的强大模式所吸引,这不仅是为了年轻的艺术家,更是为了更广泛的领域。

一个重要的点是关于经验的组织。作者确定了三种促进复杂和精湛技艺发展的“工作室结构”:示范讲座、学生工作和评判。他们发现这些结构通常是依次进行的,从示范讲座开始,进入长时间的学生工作,最后以评判过程结束。有时会出现更复杂的交织。无论如何,给我留下深刻印象的是这种教学和学习模式的高度社会性。

在典型的示范讲座中,教师花几分钟时间介绍一些概念或技巧以及一项工作室作业。示范讲座突出了解释和示范,不仅是“这些是概念”,而且是“做起来就是这个样子”。这一环节通常相对较短,为学生工作留出了充足的时间。此外,所介绍的实践会立即应用于实践,而不是隔夜或下周,这一特点有助于集中学习者的注意力。

然后是学生工作环节。我猜想对这一环节的讽刺性描述是,教师会在学生们辛勤工作的一小时里跑到最近的星巴克;但这完全不是事实。从“向团队学习……以及其他团队”的角度来看,有两个重要的特征。首先,记住教师也是团队的一员,教师会一直在巡视,提供个性化的指导,这与“高高在上的智者”模式截然不同。《工作室思考》的作者写到了教师的双重任务,他们既要牢记示范讲座中介绍的一般概念和技巧,也要关注每个学生的个体发展。学生工作环节为教师提供了一个以细致和个性化的方式回应进行中的工作的机会,通过轻推、敦促和劝诱来帮助每个学生进步。

第二个特征是这个阶段一个简单但极其重要的特征:学生可以很容易地看到彼此的作品和工作方式。当然,他们对工作室作业采取了不同的方法,但每个人都可以从他人的路径中学习一些东西。

在第三种学习“工作室结构”——评判中,涌现出更多互相学习的机会。与示范讲座一样,学习者再次聚集在一起。然而,这次不是教师展示和解释,学生观看和倾听。随着学生评论彼此的作品,教师发表评论,以及以全面评估为重点,关注事情是如何发展的、它们可能走向何方以及进一步的需求是什么,随之而来的是更多的对话。作者认为,这一切的一个重要影响是培养了“工作室的思维习惯”,包括坚持、构想、表达、观察、反思、探索等等。

当然,工作室学习的参与结构有一些吸引人的特点。为什么我们没有在各个学科中更频繁地看到它呢?一个原因与本书的整个主题有关:工作室学习非常注重“完整游戏”,而大多数正规教育并非如此。工作室学习的节奏、风格、特点和活力都利用了创作艺术作品的“完整游戏”的特点。

但我们没有看到我们可能看到的那么多工作室学习的第二个更具体的原因是:创作艺术作品尤其适合“清晰可见”的工作室体验。毕竟,可见的展示是视觉艺术作品的重点。学习者在其他领域可能创作的许多有意义的产品——数学猜想和证明、诗歌、科学解释、历史诠释——根本不像视觉艺术那样自然而然地、内在可见。

也许解决这一困境的方法不是安于现状,而是使事物比通常情况下更可见。想象一个教室,学生们小组合作,在大型白板上进行数学运算,并鼓励四处浏览。想象一下一个场景,学生们在研究一段有争议的历史事件时,将他们的论点写在大型便利贴上,贴在墙上,以便开始形成对证据的集体认识。对于此类场景以及工作室体验而言,关键在于它们不仅使作品可见,而且使工作过程可见。
除了最终结果之外,过程中的步骤也成为进一步思考和行动的共享资源。

考虑到这一切,工作室学习及其相关方法在整体学习的各个要素中表现如何?“进行完整的游戏”是最重要的点。这一点以及工作室模式立即将新引入的概念和实践付诸实践的方式有助于“使游戏值得玩”。巡视的教师进行故障排除、看到彼此作品的机会以及精心选择的工作室任务都有助于“攻克难点”。学生正在进行的作品的可见性以及围绕它们的话语促进了“异地作战”和“揭示隐藏的游戏”。最后,教师充分了解每个学生不同的学习轨迹,并与他们进行丰富的互动,这有助于他们“学习学习的游戏”。

当然,这样的环境并不总是田园诗般的。工作室学习研究的第一作者Lois Hetland告诉我一个情况,一个学生从一位工作室教师(不是研究中的教师之一)那里得到的反馈是:唯一能帮到这件作品的是石膏底料。

石膏底料是一种类似石膏的物质,用于覆盖表面以准备绘画。换句话说:把它盖住!不用说,这种贬低和打击士气的玩笑对任何学习者都没有帮助。我们应该警惕:任何参与结构都可能被误用或浅薄地使用。话虽如此,工作室学习的总体形式鼓励可以促进技能和理解力发展的丰富的社会交流。

\section*{实践社群}

你会去施乐公司的员工名册上寻找人类学家吗?估计不会。但你确实可能会找到一些。Julian Orr是 20 世纪 80 年代在施乐公司工作的一位人类学家,他的任务是仔细观察施乐公司的技术代表们的时间实际上是如何度过的。他不仅考察了他们摆弄机器的时间,还考察了他们喝咖啡和吃零食时闲聊的时间。

Julian Orr发现了一些发人深省的事情:这些零散时间里的活动并没有涉及政治和体育,而是关注如何修理施乐机器。技术代表们会互相询问他们遇到的各种挑战,分享关于如何处理有时难以启动的机器的“经验之谈”。事实证明,这种非正式的知识交流对于机器的维修和技术代表技能的提升都非常重要。

这只是一个特别发人深省且具有启发性的参与结构的例子——实践社群。近年来,实践社群在商业领域及其他领域受到了广泛关注。在实践社群中,参与者拥有共同的使命:修理机器、演奏音乐、在互联网上讨论老电影、投资股市。他们之间的社会接触创造了交流技艺的机会。参与者自然会谈论他们最关心的事情、今天或明天需要知道的事情,以及昨天或前天发现的特别有帮助的事情。大量的学习是自发进行的,并且针对的是当下,而不是将在一年或五年后才能实现的长期目标。

实践社群为已经在特定活动中拥有经验的人们提供了一个有吸引力的合作愿景。那么新手们的命运如何呢?他们会被排斥在外吗?不会,研究此类非正式学习过程的学者们已经找到了他们进入的方式。Jean Lav和Etienne Wenger在他们的著作《情境学习》中描述了“合法边缘参与”(legitimate peripheral participation)的关键机制。

这个短语确实有点拗口,但每个词都不可或缺。在从助产术到技术工艺的许多实践社群中,新手都是从边缘开始的。他们并不试图解决难题,而是观察并帮助处理工作中较简单的方面——边缘参与。但他们的存在并没有被轻视,他们的贡献受到欢迎并作为融入完整技艺过程的一部分而受到尊重——合法边缘参与。随着时间的推移,他们的技艺变得更加精湛,并承担起更多的责任。

实践社群和合法边缘参与是弥合正规培训与特定实践之间重要差距的学习方式。虽然正规培训可以提供重要的跳板,但它通常无法捕捉到日常实践的重要方面。在这种或那种情况下会出什么问题,如何处理这种或那种特殊情况,在不寻常的情况下可以在哪里找到资源,或者应该咨询谁,这些无数的细微差别都不是培训的一部分,而且实际上也无法轻易成为其中的一部分,因为它们是在高度情境化的环境中不时出现的。

与实践社群和合法边缘参与相关的思想可以应用于在正规环境中学习的学生。例如,人们当然可以将前一节讨论的工作室学习过程视为一种实践社群。此外,实践社群与教育工作者的学习相关。一种有用的模式包括教师一起仔细观察学生作业,并思考如何改进他们的实践。Tina Blythe, David Allen和Barbara Schieffelin Powell的《一起观察学生作业》是一个很好的概述资源。

这里有个我自己的例子。一段时间以来,我和我的同事一直在开发一个名为“可见思维”的项目。该项目邀请教师将各种思维惯例和文化态度融入到学科教学中,以促进更深入的思考和学习。第5章中出现了一个例子:Debbie O’Hara 带领她的幼儿园班级围绕一件艺术品进行了解释游戏。除了学生体验到的“可见思维”之外,我们还需要弄清楚教师如何学习这种方法。我们通过建立小型、密集的实践社群并添加一些结构来应对这一挑战——可以说是有结构的实践社群。

这些小型社群被称为学习小组。它们包括大约7-8名教师,理想情况下代表不同的年级和不同的学科兴趣,因为这有助于打破典型的界限并建立教师之间的同事关系。他们定期会面,最初是每周一次,后来可能是每两周一次。他们集体承担几种任务,例如学习背景信息和学习新技术。最重要的是,参与者使用某些对话协议来指导交流,分享他们在各自课堂上如何进行特定工作的技艺。其中最重要的协议之一被称为 LAST(Looking At Student Thinking)——“观察学生思维”。

LAST 以教师向小组展示一份学生作业样本开始——例如,不同学生制作的代表特定主题的三到四张概念图,或者代表课堂讨论的白板的数码照片打印件。LAST 的目标是从这些作品中梳理出学生思维的迹象,并反思学习体验是如何进行的,以及如何才能做得更好。分发材料后,教师首先简要描述产生这些材料的活动是如何进行的。在此期间,其他参与者可以简短地提问以进行澄清。

然后,发生了一些不寻常的事情:展示教师退后一步,只是倾听。其他人继续检查学生的作品,描述样本,然后推测学生的想法,提出可能适合进一步探讨的问题,最后讨论对教学和学习的影响。展示教师在整个过程中保持沉默,最后再次站出来,评论从谈话中突出的内容。

这种有条不紊的程序与施乐技术代表们在咖啡周围的闲聊截然不同。为什么要如此构建这个过程?

首先,也有非正式交流的时间。其次,教师们通常在日常工作中彼此没有太多的接触时间,因此会议时间能够高效利用非常重要。第三,经验告诉我们,围绕学生作业的自由式谈话往往会遇到系统性的障碍。例如,如果展示教师全程充分参与会发生什么?一个后果是如潮水般涌向展示教师的进一步澄清问题,而展示教师则详细回答。因此,弄清楚情况最终会占用大部分小组时间。而且最终这是一项徒劳的追求:其他参与者永远无法理解展示教师个人教学经验的细微之处。此外,当展示教师继续作为谈话的焦点时,这会将注意力从仔细检查学生作业上移开。关于通过该作业表达的学生思维的特征和质量的见解会减少。最后,要求展示教师退到一边倾听,暂时消除了谈话中任何自然的防御冲动。

合法边缘参与如何在这些学习小组中发挥作用?并非以最直接的方式,因为通常学习小组的大多数成员都是“可见思维”的新手。最初,小组由学校内负责“可见思维”的人或之前参加过学习小组过程的资深人士指导。即便如此,随着小组的发展势头,一种自然的筛选和分类会发生,参与者会分为更积极和不那么积极的参与者。一些特别大胆的教师会提前站出来,尝试一些事情并带入学生作业。其他人则会稍微退缩一下,看看情况如何,向其他人学习,然后再尝试。

在某些情况下,我们会有意地在一个小组中安排两到三名资深人士,他们会帮助带领“可见思维”的新参与者。此外,作为一项政策,任何没有参加任何学习小组的人都可以在方便的时候参加,以了解情况,并在他们愿意的情况下在课堂上尝试一些东西。最终,他们中的一些人会组建自己的学习小组。通过这种和其他方式,该过程在密集学习小组的层面之上培养了一个更宽松的全校范围的实践社群。

这项围绕“可见思维”的工作只是一个例子,说明如何利用实践社群的动态来促进学习。它还提醒人们,不能总是依赖围绕饮水机或咖啡壶的完全自发的实践社群来完成工作,可能需要会议时间、对话协议和其他方面来支持丰富的技艺交流。

回到整体学习的一般主题,最佳状态的实践社群如何阐释其各个要素?就“玩完整的游戏”而言,这种同事间的交流最容易围绕一个正在进行的“完整游戏”进行,无论是修理施乐机器还是教学或其他事情。“使游戏值得玩”受益于参与者对其技艺的真正兴趣和群体的社会支持。“攻克难点”几乎是自动的,因为难点会在谈话中作为关注领域被提出。群体中不同的经验和观点会自动产生一定程度的“异地作战”。当人们讨论他们的方法和基本原理时,他们在一定程度上“揭示了隐藏的游戏”。最后,实践社群本身就是一个参与者可以“学习学习的游戏”的环境。

\section*{跨年龄辅导}

男孩 1:我有一个小女孩,她叫Kathy,我教她数学。她一开始学得不太好,但现在进步很大。当她完成作业后,我会带她出去玩大约五分钟,然后她就会安静下来,好好学习。

男孩 2:嗯,我也有一个小女孩,我会让她玩一会儿,但当我试图教她的时候,她就开始胡闹,然后当我一转身,她就不见了。

男孩 3:嗯,也许你可以讲些笑话。我一下子想不起来,但也许你可以编一些。我们班有个男孩对数学很感兴趣,他教别人的时候会给孩子们讲笑话,这让他们对数学产生了兴趣。

这段摘自Dennie Briggs的《他们自己的班级》(A Class of Their Own )的文字,描绘了十二岁的孩子讨论他们教六岁孩子的场景。一位语法学家可能不会完全满意他们的语法,一位教育专家可能会怀疑笑话是否是吸引年幼学习者注意力的最佳方式,但即便如此,这些十二岁的孩子仍然非常投入并关心着比他们小得多的同伴的学习。这是对跨年龄辅导世界的一个简短的观察,跨年龄辅导是另一种我们可以互相帮助学习的参与结构。

成人辅导年幼的孩子并不是什么新鲜事,它是我们所知的最有效的教学模式之一——如果方法得当的话。斯坦福大学教授ark Lepper和他的同事对专家进行的成人-儿童辅导进行了广泛的研究。对教学方法有很好把握的成年人会选择提出对孩子来说既有挑战性又可以接受的问题(再次强调最近发展区),用问题和提示而不是直接的建议和反馈来引导,鼓励自我意识和自我管理,将错误和困难视为良好学习的机会,并始终提供足够的微妙支持,使学习者在某种程度上成功地解决问题。这样做既能满足情感需求,也能满足认知需求。其结果可能非常令人印象深刻,学习者在态度和能力上都会取得显著的进步。

如果社会能够负担得起并为每个孩子找到一位一对一的专家导师,其影响将是相当惊人的!但当然,负担得起和找到导师都是乌托邦式的追求。那么就转到 B 计划,即在一定程度的培训和支持下,由孩子们负责辅导其他孩子。当然,这比不上专家成人导师……嗯,实际上,考虑到对辅导者和被辅导者的影响,也许并非如此。跨年龄辅导的基本思想与其名称一致:年龄较大的学生一对一地辅导年龄较小的学生。跨年龄辅导的基本逻辑同样清晰明了。学习可以通过个别关注而蓬勃发展。鉴于公立学校资源有限以及专家成人导师的可用性有限,跨年龄辅导是为年幼的学习者提供一定程度的个别关注的一种方式,而这种关注在其他任何方式中都很难实现。

此外,针对跨年龄辅导提出的优点远不止个别关注。在某些情况下,似乎只比被辅导者大几岁的辅导者对年幼同伴的心态和困惑有特别好的理解,并且善于建立良好的、无威胁的融洽关系。也许最自然的担忧是,跨年龄辅导不恰当地利用了辅导者。恰恰相反,这通常对辅导者有好处。在学业方面,他们必须为了完成辅导而磨炼自己的理解,印证了那句老话:最好的学习方式是教导。但对辅导者的好处不仅限于学业上的进步。他们正在学习责任感、同理心和关怀。

教师在所有这些过程中扮演着极其重要但不太传统的角色,组织和监控整个过程。细节很重要。同伴辅导在具有跨年龄特征时似乎效果最佳,而不是同一班级中较优秀的孩子帮助较差的孩子,至少在正式的辅导关系中不是这样。当最初的辅导时间相对较短时(大约二十分钟左右),效果会更好。辅导对于年龄较大的孩子来说并非完全是自然而然的事情。他们需要一些技巧和方法的工具箱,这些技巧和方法是他们在彼此的帮助和老师的帮助下逐渐建立起来的。准备和汇报很重要,就像上面的例子一样。此外,教师需要认真考虑如何配对人员。谁需要关注?哪里有自然的匹配?对特定的辅导者和被辅导者的预期好处是什么?

什么样的学生是好的辅导者?根据丹尼·布里格斯的说法,显而易见的答案“聪明的孩子”并不是特别正确。毕竟,在精心选择的配对中,辅导者和被辅导者之间的年龄差距意味着辅导者无论如何都会在内容上掌握得更好。在某个领域遇到过麻烦的辅导者可能特别有能力帮助年幼的同伴解决同样的难题。最后,辅导者理清思路的努力可能会增强他们自己的理解和信心。

另一个显而易见的好辅导者的标准“表现良好的孩子”似乎也偏离了中心。感到无聊和躁动的学生可以在他们的辅导角色中找到一个引人入胜的焦点。叛逆的学生可以在他们的责任中发现一种稳定的影响。事实上,Briggs 认为,最关键的品质是“愿意尝试”。无论聪明与否,表现好与不好,那些因为任何原因而发现自己被这个想法吸引的学生,在成功方面都有相当大的优势。

它总是运作良好吗?当然不是。有许多特殊的难题和问题需要教师进行大量的故障排除。但它基本上有效吗?这是任何此类实践模式的根本问题,答案似乎是肯定的。例如,斯坦福大学的研究人员进行了一项详尽的研究,比较了四种不同的改进教学方法:跨年龄辅导、计算机辅助教学、减少班级规模以及增加成人教师的教学时间。事实证明,跨年龄辅导是四种方法中最有效的。此外,跨年龄辅导比增加成人教学时间或减少班级规模的成本效益要高得多,接近后者的四分之一。

跨年龄辅导对整体学习有什么好处?这里重要的是要同时考虑被辅导者和辅导者,因为情况有些不同。先说被辅导者,被辅导者不一定比传统教学中遇到更多的“玩完整的游戏”。这完全取决于辅导的重点,辅导的重点可能是“基础炎”盛行的常规算术方面。然而,辅导互动可能比通常情况更能吸引年幼的学习者——“使游戏值得玩”。至于“攻克难点”,跨年龄辅导的一对一形式使其自然而然。“异地作战”和“揭示隐藏的游戏”在某种程度上是跨年龄关系的自然结果。毕竟,对一个六岁的孩子来说,一个十二岁的孩子是非常“异地”的,并且可能对游戏的来龙去脉有更好的了解。然而,这并不意味着这些学习方面非常复杂。最后,熟练的成人辅导会有意地培养坚持性、自我监控以及其他有助于“学习学习的游戏”的特征,但我不太确定这是否可以对跨年龄辅导抱有期望。不过,这也许是一个有待发展的方面。

再说辅导者。辅导者肯定在进行一个重要的完整游戏,即教与学的游戏。辅导者为了教导而需要更深入、更广泛地理解材料,这在一定程度上将他们拉向完整的学科游戏,甚至可能拉向“隐藏的游戏”的某些方面。正如十二岁的孩子对六岁的孩子来说是“异地”的一样,反之亦然,辅导者可能会从年幼同伴的误解中,以及从对年幼的人是什么样子的更广泛的理解中,对相关领域获得一些更广泛的理解。最后,辅导者对辅导的关注肯定会揭示很多关于“学习的游戏”的信息。

\section*{极限团队学习}

结对问题解决、工作室学习、实践社群和跨年龄辅导——这些参与结构只是冰山一角。我们可以将基于项目的学习添加到列表中,在这种学习方式中,学生们组成团队进行实验、创作艺术作品或调查他们当地社区或生态的各个方面。我们可以添加基于问题的学习,在这种学习方式中,团队处理有些开放式的问题,根据需要利用不同的知识来源来推进可能的解决方案。这已成为许多环境下医学教育的主要内容,与传统的强化讲座课程形成对比,医学学生通过团队合作完成一系列预先准备好的案例来掌握内容,并根据需要进行学习,以得出诊断和治疗方案。

我们可以添加辩论形式,学生们就一些重要的历史、政治或科学问题准备论点和反驳论点。另一种广为人知的参与结构是拼图法,学生们组成四人小组,并将要学习的主题进行划分。每个学生负责四分之一的内容,并将其学到足以教给小组其他成员的程度。这样的例子不胜枚举。

令人鼓舞的是,如此多的参与结构都提供了向团队学习的机会,因为有力地运用这一原则可能对教育转型至关重要。我们需要的可能不是零星出现的偶尔的集体活动,不是罕见的跨年龄辅导,不是仅针对特别适合的学科的工作室学习,而是极限团队学习——用数天、数周甚至数月的时间,将大部分学习时间用于各种形式的向团队学习。

是什么促使我们将“向团队学习”置于如此核心的位置?在本章中,衡量“向团队学习”的益处是根据其对整体学习其他原则的影响来衡量的。如果没有大量的“向团队学习”,以下两个原则在大型环境中尤其难以很好地解决:“使游戏值得玩”和“攻克难点”。
让我们首先看看“使游戏值得玩”。许多因素对此有所贡献:首先要有一个完整的游戏,可及性和挑战性的适当平衡,以及清晰的长远利益。然而,在各种因素中,重要的是要记住,我们人类是深刻的社会性生物。回想一下前面概述的关于情境学习的思想,我们日常的事业不仅从围绕它们的社会互动中获得大量信息,还从中获得大量的意义和动力。如果没有某种充满活力的社会环境,就很难塑造一项完全有动力的活动,而这种环境通常不仅仅意味着与一位老师的疏远关系以及期末的成绩。

再来看“攻克难点”,学习者不可避免地会发现自己处于不同的位置。学习数学、历史或电气工程的一个学生在这里遇到一个问题,另一个学生在那里遇到一个问题。人们以不同的速度接近学科学习中特有的障碍,并发现障碍的高度也不同。

然而,保障压力强力地推动标准教育实践朝着“一刀切”的方向发展,或者当学生被划分为不同的能力水平时,朝着“三种尺寸适合所有人”的方向发展。不少人会从这种粗略的网眼中溜走,并陷入长期的不良表现,或者完全辍学。类似的后勤压力往往使学习者在作业和测试中收到的个别反馈变得晦涩难懂。回想一下第 3 章中关于学习代数的故事,学生们从作业中学到的东西很少,因为反馈主要提供的是更正,而学生由于对代数的理解不够深入,无法理解其中的原因。

这时就需要“向团队学习……以及其他团队”。正确的参与结构摆脱了限制个别关注和丰富反馈的后勤束缚。正如前面斯坦福大学的研究结果所示,这些实践也显得极具成本效益……而且它们必须如此,因为社会无法大规模地承担像小班制这样的配置。

对学习的社会层面的审视始于超人独自打棒球。只有超人才能做到,而且无论如何这都不会很有趣!也许学习要想蓬勃发展所需要的恰恰相反,是几乎没有人长时间单独做任何事情的事业和参与模式。

\section*{学习的奇迹:向团队以及其他团队学习}

我在思考如何挖掘向团队学习……以及其他团队的潜力。我可以使用各种小组活动,将学习置于更真实、更有意义的社会文化背景中。在这里,从“参与结构”的角度进行思考可能会有所帮助,“参与结构”是组织学习的角色和责任的不同方式。

我在思考如何最大限度地利用向团队学习。作为一项通用策略,我可以组织向团队学习,以促进整体学习的所有其他原则。团队支持可以帮助初学者进行完整的游戏,社会互动和责任可以帮助使游戏值得进行,等等。

我在思考如何通过结对学习来促进学习。在这里,我有一些方法。一种参与结构是结对问题解决,学习者轮流扮演倾听者和解题者的角色。另一种是跨年龄辅导,由年龄较大、经验更丰富的学生辅导其他经验较少的学生,教师则扮演导师和组织者的角色。

我在思考较大的群体如何促进学习。“实践社群”的参与结构在这里给我提供了一些想法。此外,工作室学习及其示范讲座、学生工作和评判的节奏,使学生能够观察并向彼此以及教师学习。教师学习小组使用简单的协议,专注于学生作业,可以为我们自身作为教师的发展组成一个有结构的实践社群。

我在思考如何利用其他向团队……以及其他团队的学习策略。一旦我环顾四周,我发现有许多这样的参与结构:辩论、拼图法、基于问题的学习、基于项目的学习。我可以寻找最适合我情况的方法并尝试一下。
 
