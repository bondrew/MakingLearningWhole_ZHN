\chapter{学习的未来}
学习是如此寻常的事情,如此融入我们的生活;人们永远不知道自己会在什么时候被``伏击"而学到一些东西。

我和妻子都是博物馆爱好者。几个月前,我们在旧金山的迪扬博物馆(de Young Museum)游览。在浏览美国藏品时,我注意到19 世纪美国西部的一位著名画家Alfred Bierstadt的一幅引人注目的画作。这幅画描绘了前景中一条险峻的山路,俯瞰着远处的一个湖泊,湖泊之外的景色逐渐消失在地平线上几度之上的朦胧的太阳中。画面很漂亮,但我想知道为什么前景看起来如此荒凉。我匆匆看了一眼就继续前进了。

几分钟后,正在画廊里独自参观的妻子走过来说:``你注意到那幅Bierstadt的画了吗?有点意思。"

这幅画在我看来还算不错,但我不会用``有意思"这个词。我转回去仔细地看了一下。这一次,我做了之前没有费心去做的事情,我读了画名:《View of Donner Lake(多纳湖的景色)》。突然间,这幅 1871 年的作品在我眼前开始发生变化,``多纳湖"这个名字引发了我对多纳遇难队和食人行为的模糊回忆。

补充讲一下这个故事。1846 年,由George Donner率领的 80 多名定居者向西迁移,一路遭遇了无数的麻烦和延误。10 月底,他们到了今天被称为内华达山脉多纳山口的地方,一场暴风雪挡住了他们的去路。他们在多纳湖和附近的一个地点扎营,试图等待暴风雪过去。但严冬来临,他们的补给逐渐减少。一小群人冒险穿过雪地向西寻求帮助,其余的人则躲藏起来。一些寻求帮助的人死了,他们的同伴吃了他们,然后继续前进。一些人最终成功到达,救援队被派了回去。到最后,大部队中的许多人也死于寒冷和饥饿,幸存者为了生存而诉诸食人行为。最终,有 48 人幸存下来。这件事变得臭名昭著。

回到Bierstadt在25年后创作的《多纳湖的景色》:现在,通往湖边的荒凉前景变得更有意义了。这并不是一幅田园诗般的场景,扭曲的树木紧贴着岩石斜坡,仔细一看,可以看到在远处的一个突出部上竖立着一个简单的木制十字架,一条崎岖的道路蜿蜒而下通往湖边,这肯定不是多纳遇难队会遇到的,而是一些后来的建设,使山口更容易通行。查看墙上的文字说明揭示了更多信息。在画面右侧的中远处,一条铁路正在建设中。

事实上,这幅画既是对西迁之旅巨大困难的回顾,也是对进步的颂扬,这条致命的山口先是被崎岖的道路驯服,现在又被铁路驯服,这幅画是铁路建设者委托创作的。你可能喜欢或不喜欢这个信息,你可能喜欢或不喜欢这幅画,但毫无疑问,其中包含的内容比我最初看到或想象的要多得多。

\section*{今天学习,为了明天}

我再说一遍:人们永远不知道自己会在什么时候被``伏击"而学到一些东西。也许你在博物馆里,第一次看、第二次看或第三次看时,有什么东西让你感到惊讶。也许是我爸爸拿着棒球棒和手套对我说:``我们来学打棒球吧。"也许你正在参观一个新的购物中心,你需要找到方向——无线电小屋、西尔斯百货、美食广场在哪里?你甚至可以从购物中心提供的地图信息亭那里学到一些东西。也许你正在关注一位政治候选人,而这位候选人为了说正确的话而不断地变换策略,这变得越来越令人反感,使光环褪色。也许你的台灯开关出了问题,你需要弄清楚如何修理它。也许你即将第一次去巴塞罗那或波士顿出差,所以你为会议做准备,并做一些额外的研究来计划一个愉快的额外的一天。

当然,这些事情并非总是进展顺利。如果没有我妻子的提示,我就会错过Bierstadt画作的深层含义。然而,在这些最佳状态的非常普通的学习案例中,最突出的是它们自然的、投入的、有目的性的特点。我们通常不把这些事情看作是学习的问题。我们把它们看作是感到惊讶、找到方向、解决问题、做出决定、制定计划或理解事物的问题。我们从今天、从现在、从手头的情况及其直接的意义中学习。

那么,这与整体学习有什么关系呢?简单来说,这些场合自然的、投入的、有目的性的特点正是整体学习旨在捕捉的。让我阐述一下它的一些特质。

这项工作被体验为本身就具有直接的意义和价值,同时也代表着更大的事。例如,看Bierstadt的画作本身就很有启发性,同时也提供了关于艺术作品可以承载的社会象征意义的信息。总的来说,整体学习旨在让学习者现在就参与到一个完整的游戏中,以此作为以后参与更大、更复杂游戏的垫脚石。

知识根据需要从过去零星地融入进来,同时也通过不断发展的经验来揭示。例如,我利用了我已经了解的关于多纳遇难队的知识,以及观察Bierstadt如何处理这个主题。总的来说,整体学习邀请学习者将他们从一般经验或之前的直接教学中学到的东西带到游戏中,并通过游戏本身发现新的技能、知识和见解。

来自过去经验的相互冲突的知识会汇聚在一起,并通过思考和实验来协商解决方案。例如,我最初没有仔细看,因为我过快地将这幅画归类为传统的漂亮风景画。然而,我妻子的评论与此相悖,所以我又看了一遍。总的来说,一个完整的游戏会产生关于下一步该做什么的困境,学习者可以追求并努力解决这些困境,从而扩展他们的技能。

大量的学习会自动发生,并通过强调、反思和有针对性的排练来提取知识进行扩展。例如,除了全盘接受之外,我发现自己在反思Bierstadt的画作,以及我最初是如何盲目地看待它的。总的来说,整体学习的原则邀请在之前、期间和之后进行策略性反思,以收获对未来的意义。

学校式的学习通常感觉与此非常不同。它不太像当下自然的、投入的、有目的性的学习,而更像是仅仅为了一个模糊设想的未来而学习。基础学习、了解学习占据主导地位。学生们发现自己在一件事情上费力,不是因为它在当下有意义,而是因为它应该在明年或后年很重要。许多人愿意坚持下去,有时对于某些内容来说,这可能是我们能做到的最好的,但为未来而学习的风格会引发关于投入和知识保留的严重问题。

让我们举一个熟悉的例子,做章节末的数学题。完全公开:我经常喜欢章节末尾的数学题。我们都有自己喜欢的领域。但这并不意味着它们像它们本可以做到的那样有效地服务于我的学习。通过以上四个要点来考虑标准的问题集。

这项工作是否被体验为本身就具有直接的意义和价值,同时也代表着更大的事物?数学题可能不会被体验为具有直接的意义和价值。这些问题显然是没有任何更大意义的练习。它们可能作为谜题(对我来说有效!)很有吸引力,但仅此而已。学生们被告知它们代表着更大的事物,即最终要掌握的一系列技能和见解,但那是什么却遥不可及。

知识是否根据需要从过去零星地融入进来,同时也通过不断发展的经验来揭示?学习者需要将他们刚读过的章节以及早期学习的知识融入进来,但这些练习本身几乎完全是为了练习已经呈现的知识而设计的,在这个过程中很少有新的知识被揭示出来。章节末尾的问题通常是这样设计的,它们特意不提及任何不熟悉的内容,也不需要收集超出简短问题陈述的信息。这与基于问题的学习和基于项目的学习形成了鲜明对比。

来自过去经验的相互冲突的知识是否会汇聚在一起,并通过思考和实验来协商解决方案?当天的章节往往与之前的章节隔离开来,没有提出相互冲突的知识的困境。这在目前是有效的,但不幸的后果是学生们没有学会整合他们所知道的东西,也没有学会选择不同的方法。在以后更开放的情况下,他们经常不知道该选择哪种方法。

大量的学习是否会自动发生,并通过强调、反思和有针对性的排练来提取知识进行扩展?通过做练习,一些学习肯定会自动发生。然而,很少有提示可以进行刻意的反思性策略制定或得出面向未来的结论。

懒惰的结论是:不要使用章节末尾的问题。然而,这太过绝对了。一方面,有时章节末尾的问题要丰富得多。另一方面,在整体学习中,我们可以很好地利用传统的章节末尾的问题。记住,“攻克难点”是基本原则之一。如果一页练习被视为并体验为清晰可见的更大事业的一部分,那就很好。

这就是整体学习的全部内容。整体学习的目标是直接从生动的当下学习。它的目标是基于被体验为具有直接意义和价值的工作来构
建学习——构建面向更复杂版本的完整游戏的初级版本。它的承诺是利用良好自然学习的特征,无论我们谈论的是比尔施塔特、棒球还是巴塞罗那。它的方法是通过七项原则系统化此类学习的重要特征。它的信条是,好的学习是从一个有着对未来的展望的丰富经验的今天进行学习。

\section*{今天教学,为了明天}

我当然不认为这项使命是容易的。周到而引人入胜的教育从来都不是一件容易的事。它需要关怀、思考、精力和投入。与此同时,我们也没有必要用一个又一个的原则、一个又一个的概念来为难自己。

我想起了我早期在哈佛大学教育研究生院作为一名教师的发展经历。当然,我仍在学习,但当时的一个挫折是原则太多了。我与这个和那个学习理论、这个和那个关于习得复杂概念的视角、这个和那个关于动机的模型、这个和那个关于思考和理解的模型交上了朋友。我记得有一次,我列出了一份我所知道的关于如何促进学习的最重要的想法的长长的清单。我正在计划我的教学,我问自己,“我如何才能把所有这些都融入到我所做的事情中?”

我花了两天时间才意识到这是一个非常愚蠢的问题。我根本无法认真考虑清单中哪怕十分之一的内容。每周都费力地权衡所有原则以使它们发挥作用,就像同时耍弄一桌宴席的牛排刀一样,在这个过程中我会失去几根手指。

当时让事情变得更困难的是,我没有做足够的归纳和优先排序。但即使是七项原则也不算少,而且七项原则中的每一项都隐藏着多个概念和策略。因此,对于那些想要认真尝试整体学习的人,我的建议是:绝对没有必要一开始就全力以赴地部署所有七项原则。相反,从一个初级版本开始(我们以前在哪里听到过这个?)。

最重要的要素是第一项原则,玩完整的游戏。除非人们为相关的主题和学习者找到一个合适的完整游戏的初级版本,否则根本就不是在进行整体学习。也许是使用普通算术的简单版本的数学建模。也许是简单版本的文学分析,或者将历史教训应用于当代,或者检验一个科学假设。也许是简单版本的棒球。无论它是什么,如果没有一个完整的游戏在进行,就没有整体学习。

一开始另一个明智的选择是使游戏值得玩。如果没有围绕学习的强烈的吸引力,一切都会非常艰难。使游戏对学习者来说值得进行,也就是使它对自己来说也值得进行。我们永远无法激起每个人的热情;认为我们可以做到这一点是天真的,但对于教师来说,没有什么比不关心并且宁愿做其他事情的学习者更令人沮丧的了。

在这两个方面——玩完整的游戏和使游戏值得玩——建立一些动力,你就可以赢得时间来融入其他你认为重要的原则。很快你就会希望学习者攻克难点。你可能不会那么急于让他们进行“异地作战”或揭示隐藏的游戏,但过一段时间会的。

另一项原则从一开始就会让生活更轻松:向团队学习。在这里,我指的不是学生的学习(向团队学习可能对他们早期有好处,也可能没有),而是我们自己从环境中的其他人——其他教师、导师、顾问以及任何人那里学习。如果你能建立一个读书小组,一定要这样做。如果你能建立一个由几位同事组成的定期会议,认真地查看学生作业并进行讨论,一定要这样做(回想一下第 6 章中教师使用 LAST 来看学生思维的例子)。如果你能建立一个在彼此的课堂上进行简单观察的模式,一定要这样做。整体学习,就像任何其他教育方法一样,一起处理比单独处理要容易得多。

最后一个要点,也许是最奇怪的一个:不要过于仔细地阅读这本书。一定要浏览一下它,但如果你发现一些看起来有启发性的想法,尽快尝试一些事情。正如引言中所敦促的那样,如果你把你的第一次阅读作为尝试一些简单事情的基础,你自己的个性化初级版本,你会发现这些页面更有用。然后回头看,你会发现一些进一步的想法,这些想法可以满足你甚至不知道自己拥有的需求。

我确信你也会遇到一些实践中的挑战,而整本书中都没有一句有用的建议。我们什么时候才能把一切都做对呢?我的愿望不是把一切都做对,而是把大部分都做得有帮助,我希望你也能这样认为。

\section*{明日的知识}

从某种程度上说,教学内容的问题很简单:教授今天学习者明天需要理解和采取行动的内容。不幸的是,无论是作为过个人生活的个体,还是从更大的社会意义上来说,我们都只能根据趋势和猜测来大致了解明天会是什么样子。明天是一个移动的目标。

即便如此,我们仍然可以探索什么可能帮助我们击中目标。在《未来五种心智》一书中,Howard Gardner提出了应对新兴挑战的五种基本方式,用比喻的说法是五种``心智",并敦促教育更加关注它们的发展。``有纪律的心智"指的是学科知识和思维,``综合的心智"指的是将不同的知识组合成富有洞察力和有用的综合体,“创造的心智”指的是真正新颖的见解和产品,``尊重的心智"指的是尊重远近的他人,``伦理的心智"指的是在具有挑战性的问题和关系中采取根本性的伦理立场。Gardner认为,有了这五种心智的充分发挥,人们将能够更好地应对未来几十年错综复杂的情况。例如,认识到过度专业化的风险和复杂探究的障碍,加德纳对``激光智能"和``探照灯智能"做出了有用的区分。“激光智能”深入研究,就像在学科内进行精细的工作一样。“探照灯智能”则广泛地跨越多个学科和视角,试图将事物整合在一起。我们两者都需要!

许多作者提出了关于随着人们的生活和有时不稳定的、肯定是复杂的时代的发展,目标如何转移的观点。例如,苏塞克斯大学的Michael Eraut在他的《发展专业知识和能力》一书中强调了专业教育在为人们进入工作世界做准备时遗漏了多少。大量的基本学习发生在以后的工作中,在那里,人们获得非正式的和默会的个人知识,使他们能够应对一系列有时非常微妙的实际挑战。这种知识非常注重进行游戏,是过程性知识而不是命题性知识。Eraut认为,真正有效的专业教育应该以过程为基础,而不是以命题为基础,作为初始资格的重要组成部分。从在真实世界的行动中得到答案,而不仅是从测试得到答案,才是做好准备的正确标志。

这一切都不能说明当我们最终走进经验学校的大门时,它提供了任何类似理想的学习。当我们在工作中学习时,我们很可能是在进行完整的游戏,至少是从其中的一个位置——也许是一垒,也许是外野,也许是替补击球手。然而,在缺乏指导或其他机制的情况下,可能没有时间来进行诸如分离难点以发展有针对性的技能,或进行“异地作战”以扩展能力,或揭示隐藏的游戏之类的事情。有效地利用工作场所的经验需要所有七项原则,而不仅仅是第一项。

一些专业教育方法明显更接近于直接有意义的积极参与。即便如此,它们也必然会留下很多需要学习的东西。现实情况是,一些明天以复杂且难以预测的方式出现在地平线上。Eraut对专业教育需求的洞察是教育需要培养敏捷的学习者的众多原因之一,这些学习者已经学会了学习的游戏,并且还获得了广泛理解的知识库——强大的概念系统和范例,可以帮助我们理解人性、系统性变化、冲突的根源、科学和人文探究的模式、清晰的表达沟通、创造力和批判性思维等等。

这始终是正确的,但现在尤其如此。一个简单的论点是说变化的步伐正在加快。实际上,我对此并不那么有信心。是什么速度计告诉我们变化发生的速度有多快?当然,过去的几个世纪包括了许多关于生活方式、财富和前景的剧烈转变。今天的变化速度是否比方说工业革命期间的曼彻斯特和利物浦周边地区,或者文艺复兴时期佛罗伦萨的商人阶层更加剧烈?

不要管普遍的变化速度,而是考虑当前时代的一些具体特征。最简单的一个是人类寿命的增加,这是一个缓慢上升的指数,可能会真正起飞。医学领域的当代工作似乎可能会大大延长我们的寿命。在未来五十年中,一个相当健康的成年人的预期寿命很可能会跃升五十年。即使在今天,许多人都在从事两份职业,所以想象一下活到 130 岁或 140 岁会带来什么不同,并想象一下其中涉及的多轮学习。正规教育将不仅仅是在一个人生命的开始时上升然后消退的潮汐,而是一个反复出现的循环。

除了预期寿命之外,让我们把身体、社会和经济流动性添加到列表中。这方面的一个简单指标是口音。回想一下音乐剧《窈窕淑女》以及其所基于的经典戏剧,George Bernard Shaw的《皮格马利翁》,语音学专家Henry Higgins,教授可以根据人们的口音以惊人的准确度读出他们的出生地。这怎么可能呢?因为人们在地理上或社会上都不怎么流动,因此可以形成非常特定的口音。当代社会持续的地理和社会动荡使得今天Higgins,式的侦查壮举变得更加困难。人们不断地在做新的事情:一个新的地方,一份新的工作,一个新的社交圈,一个新的国家。

今天,在身体、社会和经济流动性之上,又增加了信息流动性。先进的通信技术使信息的单向和双向传播速度更快、成本更低。过去需要费力前往大型图书馆才能获得的背景数据,现在通常可以在几分钟内从互联网上提取。与其写信给巴黎或北京的酒店询问预订情况,不如进行一次在线视频游览,以评估你是否喜欢这个地方的外观,然后也可以在线注册。在许多情况下,寻找信息的主要挑战已经从获得访问权限转变为筛选信息。哈佛大学教育研究生院的技术教育专家克里斯·迪迪写到了这需要的新千年学习风格,包括``集体寻求、筛选和综合经验,而不是从某个单一的最佳来源单独定位和吸收信息"的技巧。

不出所料,如此巨大的社会和技术变革正在推动世界上的有用知识朝着广泛的理解方向发展。Richard Murnane和Frank Levy在他们的《Teaching the New Basic Skills》中记录了当代工作世界对认知功能的要求高于以往的趋势,这些技能包括``解决问题……在团队中工作的能力,以及进行有效的口头和书面表达的能力"。在 20 世纪的最后 20 年里,只有高中及以下学历且不具备这些技能的工人在类似的岗位上面临着以不变美元计算的收入停滞不前,尽管经济在不断扩张。相比之下,大学毕业生以不变美元计算的收入几乎翻了一番。

几年后,Levy和Murnane在《The New Division of Labor》中进一步探讨了这个主题,描绘了信息处理技术如何改变工作的性质。他们发现,由于使用计算机处理相对常规的活动,以及通信和运输技术使离岸外包成为可能,工业化国家的就业市场出现了``空心化"。秘书、文员和流水线工作的市场相对于创造性工作和需要人际交往技巧的工作的市场而言受到了影响。

另一方面,许多被认为是相对不熟练的工作——看门、街道维护、垃圾收集——由于它们需要身体在场以及不易编程的身体和模式识别技能而难以被取代和离岸外包。然而,这些角色的声望和工资都相对较低,尤其是随着一些可能占据消失的蓝领、秘书和文员职位的人下滑到更低的阶层,对这些角色的竞争日益激烈。重要的信息是:学习如何学习和广泛的理解非常重要。

人们普遍认识到在我们活跃的全球化文化中持续学习的挑战,但很少有人如此坦率地正视无知。这就是亚利桑那大学医学院的医学无知项目给人一种特别耳目一新的感觉的原因。该项目针对医学生、教师和高中生进行不同的调整,强调把问题放在答案前面。该项目由医学博士Marlys Witte在 20 世纪 80 年代中期创立,它将其基本工具之一设计为``无知地图",旨在承认和阐明我们不知道的东西,而不是我们知道的东西。``已知的未知"是你所有知道自己不知道的事情。``未知的未知"是你所有不知道自己不知道的事情。``错误"是你所有认为自己知道但实际上不知道的事情。``未知的已知"是你所有不知道自己知道的事情。``禁忌"是危险的、具有污染性的或被禁止的知识。最后,``否认"是你所有因为太过痛苦而不想知道的事情,所以你不知道。

当我们走进诊所的大门时,思考医学上的无知当然是可怕的。然而,医学上的无知以及任何领域的类似情况都是非常重要的思想——再次是广泛的理解——因为被承认和识别的无知已经朝着解决迈出了第一步。正如当代教育的许多方面一样,这更加强了我的感觉,即我们应该从不仅仅是为了已知而教育,而是为了未知而教育的角度来思考邀请或可能很快邀请关注的多个学习方面。

绝大多数情况下,我们传统的教育努力都集中在为已知、为经过检验的真理、为已签署和盖章的东西而教育。可以肯定的是,一定程度的这样做是很有道理的。作为人类,我们最显著和最强大的特质之一就是有能力将事实、思想、实践,甚至是智慧传递给下一代。与此同时,正如从医学无知的主题到当代劳工趋势再到人类寿命增长的例子所证明的那样,除了为已知而教育之外,我们还需要为未知而教育,为如何绘制其地图、如何应对它以及如何掌握可以帮助我们理解它的大量理解而教育。这是我们所能期望的最大目标,因为关于未知,有一点是肯定的,那就是它总是比我们想象的更多。

对于人们永远不知道自己会在什么时候被“伏击”而学到一些东西这一事实,让我们补充一点:我们确实知道这种情况很可能会经常发生。让我们不要像多纳遇难队那样,在季节末期到达并被暴风雪伏击。让我们在山口修建我们概念上的道路和铁路。学校越关注整体学习或类似的东西,夏令营和工作场所越尊重整体学习或类似的东西,基于技术的学习环境越尝试整体学习或类似的东西,以及这些整体越能代表主动运用的广泛理解,从而实现灵活的理解和明智的行动,我们就会过得越好。我们作为个体、我们的社会及其机构越认真地对待这一挑战并积极而有策略地做出回应,人们就越能准备好抓住明天不断变化的节奏。