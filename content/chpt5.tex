\chapter{揭示隐藏的游戏}

《棒球的隐藏游戏》——你猜一下这本 1984 年出版的书是关于什么的?你可能会说是“策略”,或者是“管理”,甚至可能是“比赛的商业方面”。然而,John Thorn和Pete Palmer的这本经典著作提供的是一个关于棒球的统计视角。最令人惊讶的是,它不是为大学数学家写的,而是为球迷写的。而这仅仅是个开始。关于棒球的统计方面存在一个虚拟产业,有大量的书籍、文章、讨论等。

你可能还记得在开篇的章节中我提到过我喜欢棒球,但我并不是一个狂热的球迷。我完全坦白地说,我过去比现在更关注棒球。尽管如此,我还是从这个隐藏的游戏中发现了一些有趣的观点。击球率是你经常看到的统计数据。当一个特定的击球手上场击球时,他击中的概率有多大?击球率就是击中次数除以击球次数(乘以1000以去掉那些可怕的小数点)。这听起来当然是用于球队计划或换人决策的一个很好的衡量标准。

然而,“隐藏游戏”的爱好者认为,它并不像表面上看起来那么直接地提供信息。一个稍微复杂一些的计算得出的更优的衡量标准是“创造得分”,就是它字面上的意思。我们不问击球手击中的百分比是多少,而是问击球手平均每次击球能创造多少分。这样的统计数据涉及到保送和其他因素,但这是基本思想,也是一个明智的想法。毕竟,我们希望击球手创造得分,而不仅仅是上垒。
“创造得分”可以使球员的排名与击球率不同。例如,想象一个经常击出短打的击球手,通常只上一垒。另一个击球手击出长打的频率稍低,但可以上二垒、三垒甚至本垒。第一个人的击球率会更高,但第二个人的“创造得分”可能会更高,而这才是最重要的。或者,某个球员可能是一个优秀的“关键时刻击球手”,尤其有可能在其他球员在垒上且有机会得分的关键时刻击中。即使他的击球率相同,“关键时刻击球”也会提高该球员的“创造得分”统计数据。

从统计的角度看待运动是一种“隐藏游戏”,是一个观察我们通常认为是技巧、精神和汗水的全新窗口。任何运动都有一个统计上的“隐藏游戏”,当然,生活的许多其他方面也是如此——军务、政治、交通管理、婚姻和离婚、以及健康决策。没有人说我们必须对我们所参与的生活片段背后的统计“隐藏游戏”感兴趣,但它们确实提供了一种不同的理解方式和一种不同的杠杆。

人们在学校和校外学到的大部分东西都有其隐藏的方面、维度、层次和视角,这些在活动的表面上并不明显。以下是一些例子:
\begin{enumerate}
    \item 几乎任何领域的策略,从体育到商业,到军事,到政治,到学术出版
    \item 世界市场对你在当地五金店或杂货店支付价格的复杂影响
    \item 经济和地理对历史的影响,我们倾向于将其主要视为政治
    \item 引爆点对诸如流行病传播或春季时尚等趋势的影响
    \item 更一般地说,系统现象(如强化循环、繁荣和萧条、存量和流量)在从供水到工业创新等各种不同环境中的作用
    \item 人们通过行为传递的隐含信息,例如坐在桌子首位意味着权威和支配地位
    \item 更一般地说,社会地位的模式和象征,人们的穿着方式及其含义中的微妙信号
    \item 来自不同国家和民族的人们如何进行对话,以及通常如何相互建立关系
    \item 无意识在人类思想和行为中的作用
\end{enumerate}

此时,你可能会想,“好吧,这些游戏并没有那么隐秘,我对它们大多数都略知一二!”我们在前进的过程中都了解了一些关于“隐藏游戏”的东西,我们对许多游戏都略知一二,而且大多数人可能对其中的几个游戏知之甚深。如果它们是完全彻底隐藏的,那么没有人会知道任何事情,我们也不必为此担心……那是幸福的无知。

但请记住两点:我们对这些“隐藏游戏”的了解可能比我们想象的要少。我们只知道一些皮毛。例如,在我读到“创造得分”之前,击球率对我来说一直是一个完全合理的统计数据。或者更一般地说,以统计学和概率论为例,我们大多数人都了解一些,但研究也表明,人们“知道”的东西往往受到许多会影响日常选择的严重误解的影响。“游戏”比我们想象的要隐藏得更深。

第二点:人们可能会偶然了解到各种“隐藏游戏”,我们这里只关注的是“隐藏游戏”在教学和学习中的地位。一些相当重要的“隐藏游戏”仍然是隐藏的。学习者的大部分时间都花在玩表面游戏上。

真正掌握整个“游戏”意味着也要学习表面之下的游戏。这些层次、维度和视角可以极大地改变理解和表现。前面我强调了整体学习如何创造“阈值经验”,即掌握“游戏”的强烈导向性经验。隐藏在表面之下的“游戏”定义了新的“阈值”,并邀请了可以从根本上改变“游戏”和“玩”的感觉的新的“阈值经验”。这里重要的不仅是技术技能,还有性格能量,即对发现新视野的好奇心的激发。这就是整体学习的第五个原则:揭示隐藏的游戏。

\section*{策略的隐藏游戏}

“隐藏游戏”一个的奇怪之处是:有时我们甚至不知道我们正在玩。我之前提到过,我的本科和研究生学位都是数学,我非常擅长解决数学问题,但直到很多年后我才发现我是如何擅长的。当我最投入数学时,我没有退后一步审视我的策略方法。

洞察力来自一本著名的经典著作,数学家Gyorgy Polya的《如何解题》。在这本通俗读物和几本更专业的书中,Polya认为,解决数学问题的秘诀是采取好的策略性步骤,他称之为“启发法” 。好的“启发法”包括:将一个问题分解成几个部分,将一个问题与你以前解决过的问题联系起来,画一个图表,刻意构建一个更简单的问题版本并尝试先解决那个问题,查看具体的例子以了解它们如何阐明一般问题,以及许多其他技巧。正如这个例子所表明的,Polya列表中的许多想法不仅适用于解决数学问题,也适用于解决任何类型的问题。甚至在棒球运动中也是如此!

我偶然看到Polya的著作时,我想:他是对的吗?我是否使用了这些不同的“启发法”?我对自己做了一个刻意的实验,找了三四个数学题,开始做题,并仔细观察我自己的思考过程,同时快速地做笔记。

有点出乎我的意料,我发现我正在玩Polya的“隐藏游戏”。在我学习数学的那些年里,我甚至在没有意识到的情况下就成了一个Polya式的策略性问题解决者,我并不是说这些“启发法”是在无意识中运作的。当我把一个问题分解成几个部分,或者画一个图表,或者查看具体的例子时,我当然知道我这样做是为了帮助我继续解决问题。然而,如果你在那些年问我,“你是如何解决问题的?”我不会滔滔不绝地说出一堆技巧。我只是在当下知道如何做,但无法给出一个全面的解释。

这段个人经历提出了一个关于揭示“隐藏游戏”的严重难题:如果我们通常会自动地玩“隐藏游戏”,为什么还要在规划教和学的背景下担心它们呢?答案是,策略性的“隐藏游戏”并不总是发展得很好。凭借一些天赋、热情和多年的经验,我获得了良好的数学问题解决的“隐藏游戏”,但大多数人根本没有。

那么为什么不直接教Polya的“启发法”列表呢?因为它没有那么简单。早期迹象表明,Polya的观点可能并不完整。Polya的著作引起了数学教育界相当大的兴趣,教师和研究人员试图通过教授这些实践来提高学生的数学问题解决能力,结果好坏参半。总的来说,使用一些Polya的方法似乎并不能可靠地转化为显著的改进。

哪里出错了?关键的洞察来自伯克利教授Alan Schoenfeld的研究。Alan Schoenfeld和他的同事们进行了仔细的调查,研究如何以真正帮助学生的方式教授“启发法”。用前面(第3章和第4章)介绍的一个概念来说,Alan Schoenfeld发现学生对“启发法”的了解往往是惰性的。学生们了解了“启发法”,但没有在实际解决问题的过程中运用它们。

Alan Schoenfeld发现,最重要的缺失要素是自我管理。学生们普遍缺乏一个宏观的框架来组织他们解决问题的方法,而这个框架本应为启发式方法的运用和指导过程提供自然的切入点。Alan Schoenfeld引入了一个五步自我管理流程,该流程始于分析问题以理解问题并找到简化问题的方法;然后进入规划总体方法以避免过早计算的阶段;并发展到探索、实施和验证的阶段。Alan Schoenfeld将此与一些启发式方法的直接教学和示范相结合,获得了巨大成功。与只做相同的练习题但没有明确关注启发式方法和自我管理的对照组相比,启发式方法和自我管理相结合的教学使学生解决的问题数量增加了一倍。

数学问题解决只是策略性方法被证明有效的众多领域之一。对大多数教育而言,另一个具有重要意义的例子是阅读。一旦学生在早期教育中掌握了阅读的基础知识——这本身就是一个相当大的挑战——他们需要在学校内外的其他领域中用阅读进行学习。研究表明,各种阅读策略可以显著提高学习者对所读内容的理解和记忆。策略性阅读有很多不同的方法,其中最著名的方法之一是由教育心理学家Annemarie Palincsar和Ann Brown开发的互惠式教学,它强调包含四个核心启发式的对话过程:提问、澄清、总结和预测。“互惠”部分意味着教师和学生轮流主导对话,这是一种确保学生承担起自己采取行动的责任的方式,人们可以将此视为促进他们按照Alan Schoenfeld的方法发展自我管理过程的一种手段。

阅读策略不仅仅适用于青少年。几十年来,哈佛大学学习辅导处一直为面临令人生畏的阅读任务的哈佛学生提供策略性阅读项目。Michelene Chi和她的同事们进行了一项关于自我解释的系统研究项目,其中大部分是与大学生一起进行的。自我解释意味着在阅读过程中停下来,试着向自己解释自己正在阅读的内容。事实上,人们经常在没有真正理解的情况下就读完了困难的段落和例题。研究表明,理解力较强的学生有很强的自我解释习惯,而接受过自我解释训练的学生会理解得更好。

调查表明,各种阅读策略都可能有效。在 1988 年的一项综合研究中,Haller, Child和Walberg回顾了20项研究,发现结果均呈强烈的积极趋势。平均而言,这些项目使学生在所使用的衡量标准上的阅读能力提高了 0.7 个标准差,这是一个相当大的提升。两种策略被证明尤其有效:在文本中前后搜索以更好地理解令人困惑的点,以及使用自我提问策略来衡量自己的进步并重新调整阅读方向。

对任何事情采取策略性方法的想法都很常见。它适用于数学问题解决、阅读、体育、办公室管理、军事任务和政治竞选。那么,从什么意义上来说策略的“隐藏游戏”仍然是隐藏的呢?尽管有这些资源,但它们对大多数学习者来说仍然是隐藏的,只有一小部分教与学的经验包括对策略维度的明确关注,“策略游戏”因忽视而被隐藏,它被教与学过程对表面游戏的专注——即正确掌握事实和程序,完成习题集和其他作业——所隐藏。

这不仅仅是教师和教学设计者的责任。许多学习者自己似乎也没有时间顾及“隐藏游戏”。我的博士生Rebecca Simmons对上述哈佛阅读项目(一项普遍成功的举措)进行了一项研究,考察了学生在完成该项目四个月后的情况,以检验他们从中获得了什么。她发现,许多学生收获颇丰,但一些学生由于各种原因最终放弃了这些策略。一些学生认为,预览文本或非正式地测试他们的理解力是浪费时间,即使研究表明这些策略有明显的益处,而且课程也教授并演示了这一点。其他人则对阅读表现出厌恶的态度,他们只是想以尽可能简单的方式完成阅读。还有一些人担心像略读这样的技巧会导致他们错过一些非常重要的东西。

考虑到这么多,以下是一些揭示策略“隐藏游戏”的策略:
1.	找到或设计一个合理版本的“隐藏游戏”,例如Polya的启发式方法或互惠式教学。
2.	包括自我管理,而不仅仅是好的方法。
3.	本着整体学习的精神,像教授其他任何东西一样教授“隐藏游戏”。找到好的初级版本,将其展示出来,让学习者参与其中,展示这样做的价值,关注迁移,练习困难的部分等等。
4.	唤起这一切的性格方面——好奇心、自主性,是什么使游戏值得玩,并且值得玩得更好。
5.	注意复杂性和节奏。一次性给学习者增加太多需要管理的事情可能会适得其反。

\section*{因果思维的隐藏游戏}

这里有一个与城市化的世界及其不断增长的人口息息相关的公共政策问题:

想象一个早晨的通勤,那种要花费令人震惊的一个半小时的通勤,那种以令人沮丧的可靠性把你置于无法摆脱的交通堵塞之中的通勤。让我们为此做些什么,让我们呼吁公共部门实施一项重大的重建计划,将通往市区的干道车道数量增加一倍。如果我们把车道增加一倍,也许可以把通勤时间缩短一半。会吗?

这个问题的主要思想来自我的一个博士生Linda Booth Sweeney,她是发展系统思维方面的专家。当人们思考这个问题时,他们会得出非常不同的答案。有些人喜欢通勤时间缩短的想法,尽管他们可能会怀疑公共投资和多年的重建中断是否值得。然而,有些人怀疑是否会有任何改善。他们说:“嗯,那些更宽的道路可能会像较窄的道路一样堵塞。又会是同样的一个半小时。”

悲观主义者是对的。原因是这样:想要以某种方式进入城市的人群总是比任何合理的道路所能容纳的要多。因此,人们会根据他们一天中最理想的时间段,权衡其他选择(比如早点或晚点来,或者使用哪种公共交通工具,比如火车),来决定他们愿意在公路通勤上投入多少时间。典型的通勤时间,比如一个半小时,代表了人们平均愿意容忍的时间。将道路宽度增加一倍并不会改变不同人愿意投入的时间的分布,因此新道路将再次拥堵到几乎相同的程度,每小时带来两倍的人,却有着相同的挫败感。你可以说我们把道路的“挫败生产力”提高了一倍。

现在另有一个完全不同的难题。一根电线从电池的一极连接到开关,开关连接到一个灯泡,灯泡连接到另一个灯泡,然后连接回电池。你合上开关。第一个灯泡会比第二个灯泡亮得稍微早一点吗,还是同时亮起?当然,这一切发生得非常快,但原则上答案是什么?

一个接一个亮起是常见的回答。人们倾向于把电池想象成一种水库,电流从头到尾充满电线,先到达第一个灯泡,然后到达第二个灯泡。然而,事实并非如此。电线中已经充满了电子,就像软管里的弹珠一样。当开关合上时,电池可以说是在软管的一端推入一颗弹珠,迫使所有其他弹珠同时向前移动。因此,“弹珠”的同时运动使两个灯泡同时亮起。

高速公路和电路的世界截然不同,但这两个例子都指向一个具有挑战性的“隐藏游戏”,它对于理解我们周围的世界以及我们自身都非常重要。我的同事Tina Grotzer称这个游戏为“复杂因果关系”。在许多领域,学习者不理解各种因果系统是如何运作的。他们接受特定案例的表面故事,并获得惯例知识来解决标准问题,而没有任何关于发生了什么的总体概念。关于因果关系的困惑出现在理解诸如电路、沉浮、生物、飞机如何飞行、进化、生态、经济、历史力量和家庭关系等主题中。

复杂因果关系的复杂性在哪里呢?最简单的因果关系是一个很好的起点。我们日常推理的大部分都依赖于可以称为“多米诺骨牌式因果关系”的东西。想象一排多米诺骨牌,每一张都推倒下一张。机制就在表面:我们可以看到多米诺骨牌倾倒。因果关系非常简单:每张多米诺骨牌都推倒下一张。这是一个确定性的故事。如果你正确地设置了这一排,多米诺骨牌就会从头到尾倒下,没有任何意外。最后,有一个单一的因果主体启动这个过程,一个主要的推动者,比如说你放在第一张多米诺骨牌上的食指。

我们关于日常世界的许多因果推理都接近于多米诺骨牌式因果关系。为什么犯罪率上升?警察预算不足,因此街上的警察不够,因此犯罪机会更多——一条小小的多米诺骨牌链。为什么石油价格上涨?中东冲突,导致石油供应减少,因此根据供求规律,石油成本更高——另一条小小的多米诺骨牌链。

这些解释就其本身而言可能是合理的,它们在日常情况下可能也足够用,但它们通常只讲述了一个更复杂故事的一小部分。而这个故事确实会变得非常复杂。Tina Grotzer和我试图描绘因果解释变得复杂的方式,提出了四个主要维度:机制、互动模式、概率和能动性。这是我们试图描绘因果思维“隐藏游戏”的方式。

机制。这个维度涉及到因果故事中的“参与者”是谁。“参与者”可以是就在表面上的多米诺骨牌。然而,因果关系通常涉及一个表面之下的故事——一个关于电子、DNA、大脑过程、细菌或无意识动机的故事。人们经常不了解隐藏的机制,或者即使了解也不去思考它们(再次是惰性知识),或者如果他们思考了它们。非专家也许会学习使用机制推理,但很少有人能评估他们是否理性。

互动模式。这个维度与超越最简单的因果关系(即一张多米诺骨牌推倒另一张)有关。影响可以向多个方向辐射,而不仅仅是形成一条链。多种原因可以共同作用产生单一结果。想象一下,需要五张小多米诺骨牌推倒一张大骨牌才能将其推倒的情况。许多因果关系具有相互作用的特点,不仅仅是 A 导致 B,而且 B 也同时影响 A。在拔河比赛中,队伍互相拉扯,万有引力也是如此,不仅仅是地球吸引月球,月球也吸引地球。事实上,根据牛顿定律,所有力都是同时双向作用的。

许多系统中存在因果循环。一个令人沮丧的例子是军事升级,一方的侵略行为会引发另一方更具侵略性的行为。家庭世仇中也会发生同样的事情。许多科学理论都采用约束的形式,例如欧姆定律,它阐明了电压、电流和电阻与电路之间的数学关系。这些理论完全没有以多米诺骨牌的方式说明什么先发生,什么后发生。相反,它们断言某种关系将始终成立。这与多米诺骨牌的故事如此不同,以至于会产生相当大的困惑。

概率。最容易想到的是像多米诺骨牌这样的确定性系统,其中的效应会确定发生。但是不确定性在我们生活的世界中无处不在——疾病的传播、股市的波动、找到一份好工作或人生伴侣、量子层面基本粒子的波动,当然还有棒球比赛。事实上,这就是我们开始讨论的“隐藏游戏”,统计和概率的游戏。我们在世界上看到的许多大规模模式,从纽约扬基队在世界大赛中的统治地位到艾滋病的传播和海滩的侵蚀,都可以最好地理解为根据其概率趋势进行的小规模事件的累积效应。

能动性。这个维度与谁或什么是最主要的推动者,以及是否存在推动者有关。手指启动了多米诺骨牌的连锁反应,但是谁发动了第二次世界大战?是希特勒的“手指”吗?在某种意义上是的,但是第一次世界大战结束后,盟军强加给德国的《凡尔赛条约》在德国引起了极大的怨恨和经济困境,为纳粹分子提供了煽风点火的燃料。德国长期的反犹太主义历史也起到了推波助澜的作用,而斯大林统治下的俄罗斯迫在眉睫的威胁也给希特勒带来了非常现实的威胁。当然,还有其他因素在起作用。

在一些因果系统中,会产生一种具有欺骗性的主要推动者的错觉。许多人倾向于认为蜂后是蜂巢的统治者,指导有条不紊的筑巢、觅食和抚育活动。然而,蜂后作为繁殖者的角色非常有限。蜂巢活动的大规模有序性并非来自集中的指挥,而是来自蜜蜂在小范围内相互作用以及与环境相互作用的本能规则。如果蜜蜂看起来离日常生活太远,那么健康的自由市场经济也以同样的方式运作。对竞争性定价和交易的小规模搜索产生了成本和价值的大规模平价。对食物等 的本地需求,通过企业家精神导致了供应商、运输商和零售商的大规模网络化结构的形成。

在前一章中,我写了什么使相对论变得复杂,并将其与理解历史因果关系进行了比较,现在结合这四个维度再次审视这个问题。相对论远非多米诺骨牌模式,它调用了一种高度违反直觉的机制,其中空间和时间在复杂的相互作用中是可互换的。它是确定性的而不是概率性的,这是唯一的简单之处,但是没有任何因素可以作为一切事物都由此产生的中心能动性发挥作用。而且,正如在前一章中一样,历史因果关系甚至更糟:从个人行为到经济和人口趋势的多个层面的机制和极其复杂的互动模式,这些模式是高度概率性的,并且承认像希特勒这样的人物的重要作用,很大程度上取决于一系列其他环境,而不是单一的关键能动性。再次,我们得出结论:历史比相对论更难理解。

在这个研究方向的早期,我和Tina一起开发了描述因果关系“隐藏游戏”的方法。但Tina已经远远超出了这一点,她正在研究如何向学习者介绍这个“隐藏游戏”。在非常普遍的层面上,答案与Schoenfeld的数学问题解决方法或本书的整体精神并没有太大不同:让学习者以适当的初级版本进入整个“游戏”。传统的教学倾向于满足于表面游戏。挑战是以合理且可控的方式将“隐藏游戏”引入实践。

Tina的方法是构建一系列揭示“隐藏游戏”的经验和对话。她称这些旨在揭示潜在的因果结构的活动为 RECAST。例如,学生们通常会对沉浮现象中重量和密度之间的区别感到困惑。起初,他们认为重物下沉,轻物漂浮。在一个简单的活动中,学生们得到了一大块蜡烛,当放入一个装有透明液体的烧杯中时,蜡烛顺从地下沉了。一小块蜡烛放入附近一个同样装有透明液体的烧杯中时,也很顺从地漂浮着。到目前为止一切顺利:重意味着下沉,轻意味着漂浮。但是,如果有人把蜡烛块从各自的烧杯中捞出来并交换它们呢?

糟糕,现在大块的漂浮着,小块的下沉了。表面规则不再成立。这应该会引起一些好奇心!学生们开始不仅关注蜡烛块的重量,还关注它们所放置的液体的特性——这些液体当然有不同密度。与此同时,学生们在各种提示的帮助下开始构建更复杂的解释。Tina喜欢把这描述为帮助学生“升级”到更好的模型。学习一个复杂而微妙的科学概念可能需要几轮“升级”。

熟悉科学教育中一些流行思想的读者会注意到这个过程中一个强大且相对标准的步骤:创造异常现象。设计一些经验,让学生从一个简单的想法开始,并面临它失效的情况。但这个故事还有更多内容:选择特定的异常现象是经过计算的,目的是沿着复杂因果关系的维度进一步推动学习者。

例如,蜡烛实验解决了互动模式的维度,帮助学生看到决定沉浮的不仅仅是蜡烛块的属性,还有蜡烛和液体属性之间的相互作用。Tina积累的大量经验证据,这样的经验既能显著提高学习者对科学中特定概念的理解,又能扩展到对他们以后遇到的其他科学概念的更好理解。这意味着学习者不仅仅是在学习具体的知识,他们还在学习一些关于因果思维“隐藏游戏”的东西。

\section*{探究的隐性游戏}

1990年春,哈佛大学进行了一项旨在揭示探究游戏的大型实验。哲学系的Robert Nozick、古生物学家Stephen Jay Gould以及法学院的Alan Dershowitz三位杰出教授共同开设了一门名为“关于思考的思考”的课程。课程异常受欢迎,场地几乎立刻就被转移到了哈佛最大的礼堂之一。听众中只有一小部分是正式注册的学生,其余的人都是慕名而来,想看看这三位大师将如何应对这一挑战。

每周都有一个主题,涵盖了社会、科学和哲学等广泛领域,例如因果关系的本质、非法药物或遗传对智商的影响,三人轮流从各自的角度就主题进行二十分钟的阐述,然后进行自由讨论,并与听众互动。

在一次又一次的讲座中,三人经常相互辩论,他们定义和分类、通过类比进行阐释、提供证据和论证,并强调逻辑的连贯性。他们诉诸常识经验,考量概率和统计趋势,思考潜在的因果机制,并质疑法律等系统是否实现了其使命。例如,关于哪些药物应该被定为非法的讨论,将关于药物实际危害的证据问题与关于药物概念的连贯性以及特定药物的危害程度与针对它们的法律制裁力度之间不一致关系的概念性问题混合在一起。国家规定什么是危险药物(例如,大麻而非烟草)的权利与概念连贯性和经验证据的重要性相对立。

最终,尽管课程名为“关于思考的思考”,但实际上很少涉及思考本身——三人极少探讨思考这——但却展示了大量的思考。虽然这三位主角并没有刻意描绘他们所采取的行动,但我绘制了自己的图表。以下是一些要素:
\begin{enumerate}
    \item 范畴与定义。例如,什么是药物?我们对药物的概念是否一致?法律对非法药物的界定是否一致?
    \item 手段与目的的逻辑。例如,法律通过宣布某些药物非法试图达到什么目的?法律是否达到了预期的结果?权衡、副作用和相互冲突的目的又是什么?
    \item 逻辑连贯性。我们对药物的界定是否一致?不一致之处在哪里?我们如何解决这些不一致?它们重要吗?
    \item 意义与直觉判断。我们对某事物的直觉,例如对什么是药物的直觉,与法律或正式定义有何关系?当概念发生冲突时,哪个应该占上风?为什么?
    \item 概率与统计。特定政策的真正风险是什么?谁面临风险?风险有多大?在得出结论时,我们使用的是充分的样本,还是仅仅根据传闻进行推断?
    \item 因果推理。潜在的因果机制是什么?是否存在多重原因和多重结果?我们是否会因为两件事同时发生就被误导地认为一件事导致了另一件事?也许它们只是有一个共同的原因,这才是背后的真相。
\end{enumerate}

随着时间的推移,我开始把Nozick, Gould和Dershowitz看作是在进行认知游戏:之所以说是游戏,是因为其中有辩论、有行动、有活力,偶尔还带有一些趣味性。之所以说是认知,是因为这个词与知识以及我们如何判断什么是合理的、什么是不合理的有关。他们不仅仅是在解决问题或努力做出决定,他们是在拥抱全面探究中沸腾的混乱,他们试图围绕每周的主题构建知识和理解。
培养像Nozick, Gould和Dershowitz所展示的那种技能和投入,无疑是学科内部和跨学科教育的重要组成部分,也是培养普遍智力能力的重要组成部分。问题是,我们大多数人没有太多机会学习如何在这一层面进行探究。探究的游戏是另一种很大程度上隐藏的游戏。这项挑战至少有三个方面:最终得分问题、旁观者问题和游戏规则问题。

在大量的传统学习中,学习者听到的是最终得分——被认为是正确的理论、被最高法院驳回或维持的原判、 “正确的”历史叙述、“正确的”诗歌解读。通常,最终得分的由来会被象征性的提及,有点像晚间新闻中简单地说波士顿红袜队赢了,关键事件是七局的本垒打回了两分。但偶尔的亮点远不足以让任何人了解比赛的全貌。即使是专业出版物和技术期刊也无法真正揭示完整的游戏,它们通常只叙述主要的推理路线,而不会分享探究过程中混乱的困惑、失误以及退回去重新尝试的种种痕迹。

然后是旁观者问题。这是“关于思考的思考”课程面临的一个挑战。每周观看Nozick, Gould和Dershowitz的辩论不一定会让听众中的任何人成为更好的思考者。(那些真正注册这门课程的人参加了小组讨论,我相信这有所帮助。)更一般地说,当教育真正费力地超越最终得分,并通过讲座、书籍或其他来源完整地展现导致最终得分的游戏时,旁观者问题仍然突出。只有当学习者真正参与到游戏中而不是仅仅观看时,事情才会变得真实。

第三个挑战是游戏规则问题。不同的学科以不同的方式处理描述、解释和论证的问题。老练的Nozick, Gould和Dershowitz显然熟悉许多学科的基本游戏规则,并且可以在它们之间流畅地转换。然而,传统的教学方式很少区分或增强学习者对其所参与的特定学科游戏的理解。例如,数学中证据的黄金标准是形式化的演绎证明。在科学中,经验主义至上,理论要根据世界的运行方式进行检验。历史学家则从历史文献和文物中寻找证据,但要以批判的眼光看待。科学家不期望物质世界会欺骗他们,但历史学家知道“历史是由胜利者书写的”,并且由于这个和其他原因,过去的文字可能是经过精心设计的欺骗。

这些区分是显而易见的,但它们肯定并非如此。例如,我一直很喜欢数学教育研究员Daniel Chazen的一项关于学习欧几里得几何学生如何看待证明的调查。他问学生是否可能找到已被正式证明的定理的反例。他们认为,是的!如果他们足够努力地寻找,他们很可能会找到反例。他还问学生,对于几个不同的例子所支持的一般几何命题,他们会得出什么结论。他们说,该命题肯定是正确的!

他们的回答会让任何欧几里得的拥趸不寒而栗。形式证明的全部意义在于相对于其前提永远地确立一个结论。如果证明是可靠的,那么任何人都不应该能够找到反例。学生们也不应该相信几个例子。形式数学的整个大厦都建立在这样的理念之上:例子不能代替形式证明,因为人们很容易忽略会反驳该命题的关键例子。尽管学生对学校版本的欧几里得几何进行了大量的学习,但还是错过了认知游戏的核心。

那么,对于最终得分问题、旁观者问题和游戏规则问题,我们该怎么办?答案与如何应对隐藏的策略游戏或隐藏的因果思维游戏(实际上是更大的隐藏探究游戏的一部分)并没有明显的不同:将游戏呈现出来,并让学习者参与到游戏中,以适当的初级版本进行。

这种任务通常被认为只适合年龄较大的学习者,这是一种有害的思维定势,严重阻碍了学习者的潜力。为了提供一个反例,让我来告诉你一个关于幼儿园儿童参与绘画推理的有趣案例。当然,我们不会期望他们非常成熟,但他们说出的一些话语中包含着某种程度的洞察力,可能会让你感到惊讶。

多年来,我和几位同事一直在思维技能领域工作,并设计了各种方法,最近的一种称为“可见思维”。Shari Tishman领导开发了一种变体,使用艺术来培养学科学习中的思维技能和倾向。Ron Ritchhart则领导了一项旨在围绕学科学习建立“思维文化”的工作,其理念是让所有年龄段的学习者都参与到更深层次的学习中,同时不仅培养思维技能,而且培养对思维和学习的积极主动的态度。我们有机会制作了一些录像示例。我将向你介绍其中一个。

Debbie O’Hara在阿姆斯特丹国际学校教幼儿园。她将和孩子们一起进行一项名为“解释游戏”的活动。“猜猜看,并说出为什么”是鼓励孩子们观察事物并给出理由的口头禅。Debbie解释说,她尝试以多种不同的方式使用解释游戏,以培养普遍的警觉和给出理由的思维模式。例如,孩子们曾经用一个装满文物的盒子玩过解释游戏,以介绍有关邮政系统的概念。今天,重点是艺术,尤其是一幅由一名十二年级学生创作的大型绘画作品。这幅画是抽象的。在白色背景上,出现了数百个黑色和灰色的小波浪形。Debbie想知道她的学生会如何理解它。

Debbie问:“你们注意到这幅画有什么?你们认为它是什么?你们看着它时看到了什么?”

孩子们从相对具体的观察开始。第一个孩子只是说他看到了黑点。然后一个女孩评论说:“这幅画看起来像一条达尔马提亚犬的皮。”

“你为什么这么说?”Debbie问道。

“达尔马提亚犬有斑点,但那些斑点稍微大一点。我在中国的达尔马提亚犬也有一些。”

一个男孩说:“它看起来像一些动物在走动。”你为什么这么说?“我看到一只狮子在跑。”Debbie让他向大家展示他在看哪里。他照做了,那个波浪形确实像一只奔跑的狮子。

另一个孩子认出了一只鳄鱼。另一个孩子说这幅画像一只斑马。你为什么这么说?因为它是由黑色和白色组成的。另一个人说它看起来像日文。你为什么这么说?那个男孩指着一个例子,一个非常像日文字符的波浪形。

Debbie意识到孩子们正在挑出一些碎片,并专注于动物。也许她可以把他们从这种模式中拉出来。她问道:“如果我告诉你们它与动物无关,你们会为这幅画提出什么其他解释?”

一个男孩说:“我要说一件有趣的事情。我认为它是关于跑步的。”“你能详细说说吗?”黛比问道。“这些东西有脚,它们就这样做。”孩子用手指演示了跑步的动作。

那个女孩说:“它让我想起我妈妈的头发,因为它们是灰色、白色和黑色的。”

一个男孩说它有很多动感。它看起来像在快速移动。

一个女孩说:“它看起来像在跳跃。”你为什么这么说?“这些线条让我想起跳跃和快乐。”为什么呢?“我不知道,但这只是突然出现在我的脑海中。”

当然,孩子们的回答并不能代表一种博学的艺术史立场。但这些年轻人以认真和兴奋的态度对待作品,比许多成年人对抽象艺术表现出的认真和兴奋要多得多。他们建立了个人联系,随着谈话的进行,他们不仅对作品的各个部分做出回应,而且对整个作品及其表现力做出回应。他们轻松自如地给出理由。

虽然这种解释游戏可以用于许多物体,例如关于邮政系统的文物,但它实际上在艺术本身中有着渊源。它是根据Philip Yenawine和 Abigail Housen为纽约现代艺术博物馆设计的一种艺术作品检查程序改编而来的。该程序使用了两个关键问题:“这里发生了什么?”和“你看到了什么才导致你这么说?”这些是非常有力的提问,它们既适合幼儿园的孩子,也同样适合成年人甚至艺术史方面的专家,因为它们为每个人提供了在其自身成熟程度上做出回应的机会。

通过这种以及无数其他方式,要克服最终得分问题和旁观者问题,并将学习者引入探究的游戏中,并非难事。至于游戏规则问题,尽管这些孩子还很小,但Debbie引导她的学生们从零散的方法转向描述整幅画,这甚至可以算是朝着以一种成熟的艺术欣赏者的方式参与游戏规则迈出了一步。

\section*{权力的隐性游戏}

我在哈佛大学教育研究生院多年的同事Wendy Luttrell喜欢从人类学的角度看待教育,她的使命之一是揭示日常生活中那些深刻影响我们信念和行为方式的潜在假设。一项很有启发性的活动是让人们认真思考普通的学校桌椅。

你们知道它们是什么样的。椅子和桌子融合成一个方便的整体,桌子部分是一个相当小的平台,学生可以在上面放一本书或一个笔记本。书通常可以放在座位下面。Wendy引导人们意识到,这种非常普通的教育工具蕴含着许多值得重新思考的潜在假设和期望。在继续阅读之前,你们也许想猜猜其中一些潜在的假设和期望是什么。

例如,传统的桌椅偏爱右手学生,书写平台几乎总是在右侧。工作面不是很宽,所以显然学生不被期望协调多个书面信息来源或进行复杂的表征。此外,桌椅妨碍学生组成学习小组,并剥夺了他们共用的桌面空间,比如五六个学生围坐在一张桌子旁的情况。学习者是单独学习的!通常,教室里的桌椅只有一个尺寸。“一刀切”!

Wendy提出了另一个假设:只有那些身体能够适应桌椅的人才适合这种环境。在讲义中,她写道:

在进行关于少女怀孕的研究时,我痛苦地意识到了最后一个假设,当时我看到怀孕的女学生们试图以各种方式将她们隆起的肚子塞进桌子里,这让她们觉得自己在学校里格格不入。还有许多其他行为和活动证实了这一观点,即作为怀孕的女孩,她们不应该在学校里,但学校的桌椅以最微妙的方式传达了这一信息,这种方式难以质疑,因为它在我们称之为学校的世界中是如此熟悉和具有导向性的文化客体。

这里的寓意并不是说桌椅是Darth Vader设计用来压迫无辜学童的邪恶工具。设计桌椅的动机显然是效率和经济性。但关键是,效率和经济性最终压倒了其他因素,这些因素要么根本没有被考虑到,要么没有被认为足够重要而需要迁就。桌椅不仅仅是一种高效的设计,而且表达了一种关于学生和学习应该是什么样子的思维模式。

因此,即使是简单的桌椅也提醒我们,权力、特权和自以为是的表达在我们日常环境的建筑、我们阅读的文学作品、我们观看的电影、我们看的电视、我们穿着和看到的他人穿着的服装以及生活的无数其他方面是如何运作的。这里有一场游戏正在进行,广义上是隐藏的权力游戏,一场我们生活其中但常常知之甚少的游戏,这非常符合一句俗语:鱼不知道水是什么。

然而,有一些教育视角试图唤醒沉睡的公民,让他们意识到隐藏的权力游戏。这种议程的一个名称是批判教育学。在众多主要人物中,也许最著名的是巴西教育家Paulo Freire,著有《受压迫者教育学》和其他重要著作。批判教育学的理念是:教育应该建立一种深刻的意识,即文学、科学、大众媒体、治理形式、宗教以及社会其他方面是如何体现和表达权力、特权和自以为是的。在听说唱音乐、阅读Jane Austen的作品或观看电影《泰坦尼克号》时,我们当然应该追求审美参与,对形式和表达方式等有所理解。但我们也应该问这样的问题:谁在这里受益?以什么方式受益?谁的观点和利益得到了代表和推广?谁的观点和利益被否定或边缘化?

批判教育学是有争议的。有时它因走得太远而受到批评,因为它煽动敌意而不是周到的批判性思考;有时它因走得不够远而受到批评,因为它用知识化的论述代替了社会行动。但无论在取得良好平衡方面存在什么挑战,批判教育学肯定可以揭示许多关于隐藏的权力游戏的信息。

让我用一些例子来总结一下我们最喜欢的挑战权力的神话是如何在媒体中出现的。以电影《泰坦尼克号》为例。这部电影获得了包括1998年最佳影片在内的十一项奥斯卡金像奖,非常受欢迎。我认识的很多人都觉得它不是一部很好的电影,我倒觉得还可以。我很容易理解它获得奥斯卡奖的原因,部分原因是我是在一个特殊的背景下观看这部电影的。当时我正好在波哥大做一些教育工作,在那里观看这部电影让我敏锐地意识到它是一部多么具有美国特色的电影。

诚然,“泰坦尼克号”是一艘英国远洋班轮,但这部电影完全是关于一个出身普通的人如何对抗体制的。另一部获得奥斯卡奖的电影《角斗士》,这次的故事背景设定在罗马帝国,主题基本相同。美国西部片经常歌颂对抗体制的孤独者。换一个类型,《永不妥协》讲述的是一个带着三个孩子的单身母亲和律师助理帮助揭露一起工业中毒事件的故事。这些典型的故事对美国观众,乃至许多观众都具有很大的吸引力。在它们较好的版本中,我也喜欢它们。

但它们也充满了我们应该更加关注的假设:个人是英雄,体制是腐败的;个人或小团体是正义和公正的代表,而其他人则行动迟缓、毫无作为且胆怯;个人或小团体出人意料地拥有强大的力量,可以对抗一切不利因素。这些想法有多可靠?它们向谁承诺了什么样的正义?哪些通往正义的道路被边缘化了?谁是这些传奇故事的天然参与者?谁又发现自己很难想象自己是其中的一员?

\section*{游戏是如何被隐藏的}

在经典电影《绿野仙踪》中,巫师Oz令人印象深刻的形象直接展现在我们面前:巨大的头像笼罩着Dorothy和她的同伴,火焰喷发,神一般的声音。然而,爱管闲事的Toto在幕后四处窥探,向我们展示了截然不同的景象,真正的巫师是一个相当不起眼的小人物,挥舞着杠杆。

对于策略、因果关系、探究、权力以及其他隐藏的游戏来说,情况几乎恰恰相反。在幕布前面,巫术并没有那么戏剧性,不熟悉这些隐藏游戏的门外汉甚至看不到有什么特别的事情正在发生。至于那些入门者,熟悉的帷幕常常让他们看不到巫术。还记得我是如何没有意识到自己解决数学问题的方法的吗?同样,一个熟练的棒球投手、诗人或物理学家可能并没有那么清楚或善于表达隐藏游戏的组成部分。对他们来说,一切都像往常一样,是第二天性。换句话说,隐藏的游戏以一种容易被忽视的方式隐藏着。这就是为什么整体学习敦促我们寻找并参与这些隐藏的游戏。这也是为什么值得问:游戏是如何隐藏的?

我至少总结了五种隐藏方式:在简单化的地毯下、偏离常识的轨道、在“足够好”的边缘内、在缄默的外衣下以及在准备就绪的视野外。它们并非完全独立,但也并非完全相同。以下是关于其中每一种的一些说明。

\subsection*{在简单化的地毯下}

历史悠久的电视游戏节目《成交》也可以很好地概括传统教学环境中发生的大部分事情。从学习者的角度来看,这笔交易是:你们保持简单直接,我们会努力学习并取得合理的结果,不给你们的生活增添麻烦。这种交易并非教师和学生之间私下达成的默契协议,而是整个系统(直至州或国家课程标准和考试结构层面)中隐含的。现实情况是,揭示隐藏的游戏会让学习者和教师都面临更复杂的情况。整体学习的其他方面也是如此。

科学教育就是一个很好的例子。简单的层面获得了绝大部分关注,即使它们涉及的实际科学最少,大致如下:
\begin{enumerate}
    \item 科学教育的大部分时间都是在学习科学事实。
    \item 在剩下的时间,大部分用于学习和应用非常具体的模型。例如,学生可能会学习特定的公式,以计算物体在一定距离内下落所需的时间,就像第 4 章中的塔楼问题和洞穴问题一样。
    \item 在剩下的时间里,有时学生会研究整个建模系统,例如,牛顿关于速度、加速度、质量等的理论,该理论可用于模拟坠落物体和各种其他事物,并且更接近隐藏的因果游戏。
    \item 除此之外,也许只有极少的时间留给帮助学习者思考科学探究和他们的问题解决策略,以便他们可以通过掌握隐藏的策略游戏来处理整个建模系统。
\end{enumerate}

换句话说,故事越科学,你看到的就越少。多么遗憾!隐藏的游戏使事情变得更具启发性、创造性、洞察力、丰富性和相关性……但在许多情况下,“哎呀!太复杂了!”压倒了一切。

也许那是因为它做不到,也许学习者无法学习完整的游戏,尤其是隐藏的游戏。但如果我们应用初级版本的原则,他们是可以做到的。

\subsection*{偏离常识的轨道}

游戏隐藏的另一种重要方式是偏离常识。例如,我们理解世界的最有效策略之一是关注人们的有意识行为。那座高楼或那个巧妙的
开瓶器是从哪里来的?是某人那样制造的。

这是常识,但这种常识非常容易超出其界限。人们倾向于在没有意图的地方投射意图。还记得因果思维的隐藏游戏中的蜂王例子吗?人们认为蜂王统治着蜂巢,但她实际上只是一个像其他蜜蜂一样由 DNA 编程的小机器人。或者,从历史的角度来看,想想战争是如何发生的。人们很容易认为它们都是邪恶的帝国主义者的阴谋,但至少其中一些战争来自意想不到和不希望发生的升级循环。第一次世界大战就是一个常见的例子,一系列条约将一个又一个国家拖入了军事对抗的泥潭。

投射意图最著名和最臭名昭著的例子是关于达尔文进化论的。说到战争,今天,尊重支持达尔文地球生物多样性理论的大量证据的进化论者与鼓吹智能设计的原教旨主义者之间正进行着一场战争。在某种程度上,这是科学和宗教之间的争端。但在另一方面,这是分析立场克服了对意图解释的冲动与人类思维中根深蒂固的意图偏见之间的冲突。一只青蛙、一只鸭嘴兽、一棵棕榈树或一种真菌,它们在形态和生存策略上的所有微妙之处,看起来都是被设计出来的。即使没有特定神学的支持,智能设计的立场也强烈地吸引着人类大脑中根深蒂固的倾向。

还有无数其他例子。复杂因果推理的许多要素都违反直觉。回顾策略的隐藏游戏,像在尝试解决问题之前彻底理解问题这样的举动,与心理学家称之为“以解决方案为中心”的人类冲动背道而驰。回顾探究的游戏,在从经验中得出结论时,我们倾向于将一些生动的例子视为最终结论,因此对统计抽样的关注可能显得复杂。

偏离常识是一个自相矛盾的信号。有时,偏离常识是一个需要注意的危险信号。但有时,它们是通往更好理解的绿灯。是时候再讲一个体育轶事了:我和妻子偶尔会打乒乓球,打得不算好,但玩得很开心。现在,大多数新手都像和球拍握手一样,直接用拳头握住乒乓球拍。我在哪里读到了一条关于更好方法的提示,将你的手掌沿着球拍的一侧放置,用拇指从另一侧握住它。这样做的好处是可以轻松地在前手和后手之间切换。连续打了好几局都感觉非常笨拙,但我很快就能感觉到好处,笨拙感也逐渐变成了舒适和流畅。

很多学习中都会发生类似的事情。有时,要学习的东西感觉是错误的,会产生一种发自内心的抵触情绪。但这种抵触只是暂时的。教学过程需要揭示隐藏的游戏——这里不是红灯,只是通往值得去的地方的道路上的一个颠簸。

\subsection*{在“足够好”的边缘内}

游戏隐藏的另一个地方是在“足够好”的边缘内。一个长期以来我最喜欢的例子来自英国诗人、文学学者I. A. Richards。在他的《实用批评:文学判断研究》中,Richards写到了他为了让大学生认真对待诗歌而进行的斗争。他记录了他们遇到的各种困难,其中之一是倾向于片面解释。当被要求解释一首诗时,学生们会挑出几行,并讲述这些诗句的含义,有时是非常离奇的。Richards会问,这与诗歌的其余部分如何契合?通常根本不契合,与其他诗句相冲突,并将诗歌从一个谜团变成了一团糟。但这似乎并没有给学生带来太大的困扰。他们会极力捍卫自己片面的解释。对他们来说,理解几行诗句就“足够好”了,其余部分会以某种方式契合。

“足够好”的综合症随处可见。以表面价值来看待政治家的演讲难道不够好吗?尤其是当他看起来像一个你可以一起喝啤酒的普通人时!如果我用三点来论证我的观点,难道不够好吗?我真的需要回顾并反驳反对派明显肤浅的观点吗?说重物比轻物下落得快难道不够好吗?对铁砧和羽毛来说是这样!

新思想通常旨在更好地解释事物。然而,学习者可能看不到这种需要。以引人入胜的方式引入新思想可能不仅需要阐明新思想本身,还需要重新教育学习者对“足够好”的理解。这可能需要鼓励他们对看似微小的差异或次要考虑因素感到担忧。这可能需要更清楚地说明新思想应该完成什么,以及为什么这样做可能是有价值的。

例如,本章前面关于大小蜡烛块的实验颠覆了一个看似“足够好”的解释:大物体下沉,小物体漂浮。当大块在更换烧杯后漂浮,而小块下沉时,“足够好”就不再足够好,通往理解更深层次因果模式的道路就会显现。

\subsection*{在缄默的外衣下}

谢天谢地有缄默知识!我一边想着午饭吃什么,一边在十秒钟内走下了楼梯,没有滑倒或摔倒。一句常见的格言是,如果我们试图精确地思考我们是如何走路的,我们就会被自己的脚绊倒。这并非完全正确,但它说明了一个问题,就明确清晰的知识而言,我们并不真正知道自己是如何走路的。说一个词或一句话或进行一次谈话也是如此。总的来说,我们关于如何做这类事情的知识是缄默的。我们或多或少以符合语法的方式说话,而无需提醒自己规则。我们使用谈话轮换的缄默原则,而无需对它们有任何反思意识。
缄默知识并不总是完全无意识的。在家庭、工作场所、超市,我们经常对“这里的事情是如何做的”有缄默的理解和期望,如果需要,我们可以清楚地表达出来。然而,在大多数情况下,它们在习惯的层面上运作,而无需刻意关注。回想一下前几节我讲的关于数学问题解决策略的故事,在我读了波利亚的书之前,我并不知道自己拥有这些策略。

无论是完全无意识的,还是仅在我们意识的边缘(哲学家Michael Polanyi使用了引人入胜的术语“边缘意识”),缄默知识都是一种强大的资源。如果我们不得不明确地管理我们缄默地运用的所有知识,生活将是一个永无止境的繁琐计算的迷宫。研究表明,缄默的直觉性问题解决方法通常(并非总是!)比深思熟虑更有效。Guy Claxton在他生动的《野兔大脑,乌龟思维》中总结了这些证据,最近Malcolm Gladwell在《眨眼之间》中也考察了相关的领域。

话虽如此,缄默的优点有时会变成教师和其他导师想要分享他们的技艺和理解的障碍。就像我的数学问题解决一样,在某个领域相当熟练的人通常不确切地知道他们在做什么。对于因果推理的隐藏游戏也是如此,熟悉科学、经济学或历史推理的人并没有那么意识到他们思考事物方式背后的广泛的因果假设。

对于探究的隐藏游戏也是如此,每个学科都有其自身的风格。例如,证据在数学、物理、历史和文学中看起来不同,它们都很重要,但在每个学科中都有其独特的规则。然而,教学几乎从未直接涉及证据所采取的特殊形式。聪明而幸运的学习者从上下文中学习它,但对许多人来说这还不够。总的来说,专家知识的缄默性是隐藏游戏隐藏的另一种方式,揭示缄默性成为揭示隐藏游戏的另一部分。

这里存在一些争议。一些心理学家研究一种称为缄默学习的过程,人们通过渗透作用(无需明确的指导)来获得缄默知识。一些教育家认为,渗透作用比冗长地阐述规则和原则更好。这是一场复杂的辩论,不适合在此详尽回顾,所以我只就我看来提出三点。首先,缄默学习是一种真实且有充分记录的现象,这一点毫无疑问。例如,孩子们就是这样学习他们的第一语言的,我们所有人也以这种方式吸收了“这里的事情是如何做的”的许多方面。

其次,在我看来,缄默学习在充满相关思想和实践的环境中效果最佳,例如,孩子们学习他们周围每个人都在说的语言。在不太饱和的环境中,许多人错过了缄默游戏,仍然是边缘参与者。

第三,揭示缄默性当然并不意味着教授一系列规则并说“去吧,永远按此行事”。要使其有效,它需要包含大量的真实实践,将通常是缄默的游戏放回表面之下,在那里它可以最好地发挥作用,就像潜艇而不是游轮,隐身而不是炫耀。

当我们从专业知识的缄默知识转向社会许多方面中隐含的缄默假设时,情节变得更加复杂和黑暗,特权和权力阶层的人通常更愿意让这些假设保持在潜艇的水平,而批判教育学和类似的方法则将其强行推到表面。揭示隐藏游戏的挑战不再仅仅是技术性的。它变成了政治性的。除了习惯和惯例的障碍之外,还有不适和恐惧的障碍。我们敢于撼动我们每个人都在其中占据一席之地的建筑吗?尽管有些地方比其他地方好得多?不稳定是洞察力的必然代价吗?从怀疑的角度来看,我们能否期望在革命之后得到的除了别人的政权,以其自身的方式同样不公平?

无论如何,在可以摆脱困境的地方,解放性的教育路线似乎在道德上有义务尝试这条道路。当这样做时,它常常做得过火,但在许多情况下,它根本没有做。在生成性平衡中取得平衡是对每位教育者身上艺术家的要求。

\subsection*{在就绪的视野外}

幼儿园的孩子可以合理地尝试什么?就像在“你看到了什么让你如此认为”的活动中一样,幼儿园的孩子当然可以谈论绘画中发生了什么。人们可能不会要求他们讨论他们周围隐藏的权力结构,即使使用像学校桌椅这样熟悉的例子也不行。“在准备就绪的视野之外”是隐藏游戏隐藏的另一个地方。

我们对不同年龄和背景的学习者准备好做什么的判断是进入一门复杂学科(人类发展研究)的入口。从教育的角度来看,人类发展观中也许最重要的概念是准备就绪。无论不同的发展模型是否使用“准备就绪”一词,它们通常都与年轻人或成年人准备学习什么、哪些类型的任务和理解触手可及,以及哪些似乎超出了学习者的视野有关。

这里的“触手可及”与积累知识关系不大,而与心智的拓展关系更大。人们很难期望法语第一学期的学生阅读Proust的作品,但这基本上被视为积累知识、经验和技能的问题。相比之下,我们不会要求幼儿园的孩子思考学校桌椅的政治意义,因为它对心智来说太过牵强。它代表了一种我们怀疑孩子会发现困难和不安的层次、态度和维度的转变。

以任何简洁的方式描绘人类发展的复杂性都具有挑战性。存在相互竞争的儿童发展理论和成人发展理论。它们在一些重要问题上采取了相互冲突的立场。例如,一些理论确定了广泛的阶段,在这些阶段中,智力能力在大致相同的时间在广阔的范围内前进。在一年或两年的某些时期,年轻人据说在一系列活动中普遍变得更有能力,从推理数学概念到从多个角度理解社会情境。其他研究人员则认为,人类发展不那么广泛,而是在不同的理解和技能领域内独立前进,而不是一次性地在整个大脑中前进。

除了这类主张之外,研究人员还提出了不同类型的因果机制。发展上的进步可能反映了由于神经发育导致的大脑功能的飞跃,或者具有广泛影响的某些逻辑模式的获得,或者超过了丰富程度的临界点的知识库的积累,从而使一系列全新的理解成为可能。
此外,聪明的研究人员和教育工作者不断发现,年轻人通常在比他们之前想象的更早的年龄就表现出更强的能力。很大程度上取决于使用熟悉的例子和支持儿童思维的方式,这样他们就不必同时在记忆中保存太多东西。这就是我 在第 1 章中论证的原因,构建可访问的初级游戏版本在很大程度上是一个反复试验的过程,即从一个人最好的判断开始,设计第一次尝试,看看效果如何,然后进行调整。如果不构建一个版本来尝试,就很难确定任何特定群体对某件事的准备程度。

因此,与其深入研究关于人类发展的思想迷宫,不如分享两个在各个年龄段、各个阶段和各个主题中似乎特别有帮助的重要思想。第一个是迄今为止最熟悉的,来自俄罗斯心理学家Lev Vygotsky(1896–1934) 的“最近发展区”的概念。以下是非常简短的版本。以日常社会行为为例,幼儿在某种技术意义上倾向于以自我为中心。,们很容易学会遵守许多社会规则,但他们并没有真正从相关其他人的角度看待情况。例如,他们没有意识到某人可能会因为他们所做或所说的事情而感到受伤。然而,并非他们与这种敏感性相去甚远。一个故事、一个解释、一点头脑风暴、一个邀请他们从Sally或Alfred而不是他们自己的角度思考情况的邀请可能会产生一些真正的见解。他们无法自发地做的事情,可以在支持下被引导去做。

学习者在帮助下可以表现良好,但单独无法表现良好的地方就是他们的最近发展区。该区域不仅是学习者要去的地方,也是可以帮助学习者去的地方。学习者不会过一段时间就直接滑到下一个复杂程度。学习者通过偶尔以更复杂的方式在任何可用的支持下进行功能来向前攀登。有时这种帮助是故意的,是导师、父母或老师的干预。有时这种帮助是偶然的,是电视节目或故事书或年龄较大的同伴倾向于互动的方式的影响。无论哪种方式,在最近发展区的功能都会将学习者拉向新水平的自主理解和行为。

对教学过程的实际意义是直截了当的:保持比学习者稍微领先一点,但不要太多。不要让学习者只做他们无需帮助就能轻松完成的事情。提高标准,提供帮助,然后随着他们发展出自己管理该复杂程度的能力,逐渐减少帮助。这就像自行车上的辅助轮。它在文献中甚至有一个名称:支架。在建筑工人的世界里,支架是一种临时结构,旨在方便建造永久性结构。同样的想法也适用于人类发展和学习的世界。

我想概述的关于人类发展的第二个非常普遍的想法来自发展学家Robert Kegan,他是哈佛大学教育研究生院的同事。它被称为主体客体理论。在这里,社会视角转换提供了一个很好的例子。尽管孩子们以自我为中心,但他们一直在与其他的孩子和成年人打交道。但是,他人的观点并不是他们世界观的一部分。事实上,他们并没有真正区分自己的观点和他人的观点。从某种意义上说,他们对观点一无所知,因为他们只有一个观点,即他们自己的观点。他们受制于各种观点,而不是将自己和他人的观点视为他们可以比较和对比的客体。发展到完全意识到自己和他人的观点需要一个主体客体转变:他们以前受制于的东西现在变成了他们可以思考的客体世界。在转变发生之前,下一个游戏隐藏在受制于的迷雾中。

几乎任何类型的学习都可以这样看待,这本质上是揭示缄默性的一种方式,让我们以学习语法的精妙之处这样平凡的事情为例。通常,我们只是像小时候学的那样说话。我们受制于这些说话习惯,并且完全没有意识到语法本身。当我们被教导一些官方正确说话的规则时,随着时间的推移,我们逐渐意识到我们说话的语法是一个客体,是我们可以控制并根据新标准进行调整的东西。诚然,这个过程的进行常常没有充分尊重不同族群的自然说话方式。承认这一点,所涉及的是一个有时笨拙而痛苦的主体客体转变过程。
举一个更复杂的例子,爱因斯坦的相对论引导我们进入另一个主体客体转变。经典物理学中的时间是一个非常有限的概念,只是一个告诉我们过程进行到什么程度的参数。我们受制于这种普通的时间概念,因此很难理解相对论的时间概念,后者以一种奇怪的方式将时间和空间视为在某些方面可以互换的。这是我们大多数人没有做出的主体客体转变,事实上,我们大多数人出于普通目的没有特别的理由去做,但它是对普遍时钟装置进行复杂理解必不可少的。

就像Vygotsky的最近发展区一样,基根的主体客体概念对学习的支持具有广泛的意义。它说:根据进入视野的客体来思考要学习的内容,学习者以前只是受制于这些客体。事实上,最近发展区和主体客体转变交织成一个单一的场景。正是下一个可用的主体客体转变构成了最近发展区。当我们揭示隐藏的游戏时,新的客体定义了好奇心的视野和揭示性的阈值体验。

\section*{学习的奇迹:揭示隐藏的游戏}

我在思考如何为学习者揭示隐藏的游戏——确切地说是许多游戏。当我思考这个问题时,我开始看到策略、因果思维、探究、权力
等等的隐藏游戏。我可以通过例子和讨论向学习者揭示这些游戏,或者将学习者引向正确的方向,并询问他们看到了什么。

我在思考如何以易懂的方式对待隐藏的游戏,让学习者感到兴奋和赋权,而不是感到负担。我可能需要先用非常初级的版本来保持简单。我可以唤醒好奇心并诉诸不断增长的能力。我可以鼓励自我管理,而不仅仅是好的举动。

我在思考如何超越学习者通常看到的东西来揭示隐藏的游戏。学习者通常只看到结果——结论、发现、最终版本——而忽略了达到结果的游戏是如何进行的。我可以让他们专注于过程。学习者通常扮演旁观者的角色,而不是参与者。我可以让他们成为参与者。“游戏规则”通常没有被探索和讨论。我可以帮助揭示规则。

我在思考如何找到安全和勇气来揭示敏感的隐藏游戏,例如渗透到社会中的权力游戏。我可以选择我的战斗,专注于那些具有启发性但又不太敏感以至于会引起麻烦的问题。我可以让学生研究其他与我们自身相似但提供一定距离和超脱的环境。

我在思考如何确定隐藏游戏隐藏在哪里。我可以记住一些典型的藏身之处:在简单化的地毯下、偏离常识的轨道、在“足够好”的边缘内、在缄默的外衣下以及在准备就绪的视野之外。

