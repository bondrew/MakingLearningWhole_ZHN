\chapter{学习学习的游戏}
最近,我和妻子在马萨诸塞州波士顿地区的常住地以外的地方度过了几个月。当你搬到一个新的环境居住一段时间时,关于新社区的“是什么”(what)和“怎么办”(how)的问题会立刻出现:你想去哪里,以及你打算如何到达那里?“是什么”的问题与生活的实用性和娱乐性息息相关,实用性方面包括当地的超市、药店、加油站和百货商店,娱乐性方面包括电影院、博物馆和乡村漫步。
然后是“怎么办”的问题。碰巧我的妻子负责我们大部分的驾驶。在我们到达几周后,她需要回波士顿地区处理一些事情,我发现自己独自待了一段时间,终于可以开车了,我发现许多人在类似情况下都会遇到的问题:我真的不太确定如何找到那些实用和娱乐的地点。我几乎和她一起去过所有地方,但存在“乘客效应”。当你是一名乘客时,你只是搭便车而已。看着街道掠过会让你学到一些关于导航的知识,但会遗漏很多东西。

“学习学习的游戏”的故事也类似。“是什么”和“怎么办”的问题再次出现:关于学习的游戏可以学到什么,以及它将如何被学到?在“是什么”这一方面,有很多关于学习的游戏值得学习——记忆策略、问题解决策略、深度阅读和快速阅读技巧、时间管理等等。一些学生以某种方式培养了一系列好的方法,但另一些学生则没有。用第五条原则的话来说,学习的游戏是一个隐藏的游戏。它很大程度上发生在人们的头脑中,不像棒球或长除法那样清晰可见。

一个很好的关于这个隐藏游戏的描述,不仅对教师而言,也对学习者而言,来自整体学习的原则本身。例如,假设作为一名学习者,你已经学会了寻找完整的游戏,无论是学术学科、体育运动、爱好还是经营企业。那么你就学到了一些关于学习的游戏的重要内容。假设你已经学会了努力使手头的游戏值得玩,即使它起初看起来不是很吸引人,你也学会了寻找个人联系,去寻求既不会令人沮丧也不会令人厌倦的挑战水平,坚持不懈并享受逐步的进步,而不是期望所有事情都一次性到位。那么你又学到了一些关于学习的游戏的重要内容。其余的原则也是如此。

所有这些将在后面的章节中得到更多关注。然而,“怎么办”的问题仍然悬而未决:它将如何被学到?在这里,几乎没有什么比“乘客效应”及其积极方面“驾驶员效应”更重要的了。对于学习的游戏,就像在社区里找到路一样,人们只有通过自己驾驶才能学到他们需要的东西。

教练说该进行击球练习了。老师布置了这十五道题,要求下周二交。在许多正规学习环境中,学习者自己几乎不做任何“驾驶”。他们没有很多选择。作者、课程设计者和教师为他们安排好了一切——清晰而完整地定义要学习的游戏(完整游戏与否),通过激励和相关性的论证来激励他们,预先定义难点,并确保难点得到充分的练习。一般规则:为他们详细说明!

当然计划很重要,这正是整体学习框架的重要信息。然而,就像我妻子大部分时间都在开车时的我一样,当我们为学习者微观管理整个过程时,他们可能会学到目标内容,但他们不太可能学会如何学习。他们不太可能学会这些技能,也不太可能学会掌控一切的心态,即管理自己学习的意愿。

因此,整体学习的最后一条原则“学习学习的游戏”提出了一个几乎是悖论式的挑战。我们应该为学习者组织学习经验,以进行整体学习,但又不能组织得太多,以至于他们永远无法坐在驾驶座上。相反,我们希望以小的方式让他们坐在驾驶座上。我们希望为他们创造关于“驾驶”是什么样子的门槛体验。然后我们希望使自主性更大,门槛更高。我们想教他们开车,如果我们不让他们开车,我们就无法做到这一点!

\section*{驾驶座}

最近,我与一所创新公立学校的校长、一位教师和一位家长进行了一次长达两小时的谈话。我主要听取了他们讲述的故事(为了保密,我会稍微修改一下)——关于这项或那项活动是什么样的,某个倡议是如何展开的,以及在一个棘手的情况下发生了什么。这些故事揭示了很多关于如何让学习者坐在驾驶座上的情况。

稍后回顾这些故事时,我发现了一个共同的主线:学生的自主性。这些相对年轻的学生——这所学校只到五年级——一次又一次地发现自己坐在驾驶座上。这是学校文化的一部分,以许多微小的方式贯穿于其结构之中。这所学校的“隐性课程”(而且并不是那么隐蔽!)与其说是关于顺从的乘客,不如说是关于负责任的年轻驾驶员。

例如,校长讲述了一天,她同意照看老师的四、五年级混合班,因为他不得不离开。最大的意外是她几乎无事可做。学生们知道议程,并自行着手进行。当然,作为一项责任,少量的监督是重要的……但只是责任,而不是微观管理,甚至不是管理。

然后是考试的角色。与其他许多学校一样,正式的高风险考试是这个教育场景的一部分,学生们通常表现良好。然而,学校强调的是诊断性测试,即探索每个孩子在特定领域取得了多大进步的方法,以决定接下来需要关注什么。给我留下深刻印象的是,学生而不是老师首先评估了自己的进步。这些测试被明确地定义为提供信息的工具,而不是对价值的评估。学生们判断哪些项目他们能轻松处理,哪些项目不太容易。然后问题是:好的,这对现在的重点意味着什么?当然,教职员工会提供帮助。例如,有时学生们的目标不够远大,对他们可能取得的成就过于犹豫,或者对轻松的道路过于安逸。老师会与这些学生一起探讨一个合理的目标,并一起制定方案,与学习者共同决定,而不是替学习者做决定。

各个班级的学生可以集体做出某些选择。这位老师描述了一个场景,他的学生们正在为如何管理特定的课堂资源而苦恼。他承认他很想干预。壁橱里有很多他们想做的材料,但他克制住了自己。随着谈话的进行,结果表明,手头有材料并不是主要问题。问题在于丢失的材料以及它们是否可以找回。学生们负责任地自行解决了难题,这让老师非常高兴他让他们继续进行下去。

由于学生们更频繁地坐在驾驶座上,任何老师自然会担心失去控制,因此一个关于一个聪明、调皮且善于操纵的女孩的故事尤其令人印象深刻。一次特别冒犯的行为意味着她将不被允许参加一次班级旅行。但她会因此受到什么惩罚呢?停课几天不会有什么效果。学校是用来学习的。相反,她会留在学校继续学习。她会学到什么呢?

一项练习是写作。老师没有给母亲写便条,而是要求这个女孩写一封信来描述情况——而且要写好:整洁、语法正确、解释清楚。这封信最终需要修改三稿。如果孩子没有把信交给她的母亲怎么办?校长解释说,在这种情况下,孩子会通过电话与她的母亲交谈。不是我,校长强调说。我们会从我的办公室用免提电话拨打,我会尽量少说话,我们会看看谈话进展如何。

所有这一切都充满了民主精神。很容易承认,这种“驾驶座”学校文化非常有利于培养学习如何学习的技能和意愿。同样重要的是,它们对具体的学习也非常有益。

\section*{乘客座}

相比之下,即使学生今天完成了作业,即使他们以肤浅的方式进行着“完整游戏”,让学生一直扮演乘客的角色也会在不知不觉中起到破坏作用。例如,许多高中化学或物理实验经历会引导学生完成科学探究的“完整游戏”,但在每一步,他们都会被告知下一步该怎么做。这种乘客角色完全把学习者当作是搭便车的人,这趟旅程是别人计划的,也是别人在进行的:请做个好旅客,遵守旅程的安排,不要惹麻烦。

在乘客座的学习文化中,学生们常常陷入肤浅的学习模式。关于学生学习心态的大量研究存在,其中令人惊讶的一部分是在大学层面进行的,你可能会期望在那里看到成熟,但通常并没有看到。在一次著名的分析中,瑞典哥德堡大学的Roger Säljö揭示了截然不同的学习方法。一些学生表现出一种倾向,即获取、复制和应用事实性的信息;另一些人则将理解置于中心位置,并渴望探索不同的视角。

Säljö, Ference Marton, Noel Entwistle, Dai Hounsell以及许多其他人继续撰写了关于肤浅、深度和策略性学习方法的文章。“肤浅”的方法侧重于弄清事实和技能,努力表现良好,尤其是不想表现糟糕。“深度”的方法追求全面的理解,并珍视内在动机,而不是追求表面上的良好表现。“策略性”的方法兼具两者的特点:它通过成绩和其他赞誉来追求对良好表现的认可,并强调良好地管理工作过程。它比肤浅的方法不那么表面,但也不像深度方法那样全身心投入。在一项相关的调查中,密歇根大学的保罗·平特里奇等人将许多学生描述为具有“表现导向”而不是“掌握导向”,他们努力表现良好,而不是真正地做好。

另一项相关的调查涉及学生根据他们对能力的看法而产生的坚持性。第 2 章提到了这一点,探讨了学生的期望对表现的影响。多年来,Carol Dweck和她的同事一直在研究那些认为能力具有固定的“要么你行,要么你不行”特征的学生是如何变成过早放弃者的,他们认为,当他们无法相对快速地掌握某些东西时,他们可能根本无法掌握它。他们通常也会试图掩盖自己的缺点。

但这完全不是个人一成不变的倾向的问题。环境的文化会产生巨大的影响。教师以无数微妙和明显的的方式发出信号,表明他们期望学习者扮演的角色——被动或主动、顺从或敢于冒险、被管理或自我管理。肤浅与深度的方法、表现与掌握的导向、过早放弃与坚持,这些都不是像学生眼睛的颜色那样不可磨灭的特征,而是由他们的先决条件和特定环境的动态相互作用而产生的特质。

Dweck通过描述教师在学生遇到困难时做出的一些截然不同的反应,很好地说明了这一点。例如,假设当老师叫Johnny回答第七道数学题的解法时,结果发现Johnny遇到了很多麻烦。老师可能会说:“好吧,尝试得不错。数学很难!让我们看看还有谁有想法。”或者老师可能会说:“好的,你已经在那里迈出了第一步。你认为下一步好的步骤可能是什么?”这两种回应传递了非常不同的信息。第一种告诉Johnny,数学很可能超出了他的能力范围,而第二种则说要一步一步来,看看你能弄清楚什么。第二种鼓励Johnny坐上驾驶座并驾驶!

传递正确的信息对于良好的辅导的艺术和技巧也很重要,这是前一章的主题。根据斯坦福大学研究员Mark Lepper的研究结果,专家导师——而不是那些技巧较差的导师——会将互动集中在问题和提示上,而不是命令上,以鼓励被辅导者认为自己处于掌控之中并对自己的进步负责。被辅导者通常会形成一种印象,即他们对自己的成功负有比实际更大的责任。这是一个富有成效的错误!他们的问题部分在于他们缺乏对自己作为学习者进行指导的信心。

许多学生敏锐地意识到,他们可以对学习采取不同的心态。他们认识到他们希望学习的方式与学习实际展开的方式之间存在一种张力。我很高兴了解了以色列学者Linor Hadar的研究工作,她在哈佛大学教育研究生院度过了一年的博士后研究。她的研究兴趣之一是学生对良好学习的看法。她曾在以色列的三个不同中学环境中对此进行过探索。

当被问及什么是良好的学习时,很大一部分学生非常自然地将典型的学校乘客座的学习观念与理想的驾驶座的观念进行了对比。以下是学生们反复描述学校学习的一些方式:成绩,课堂行为,接受知识以及在课堂上适当地积极。

学生们对理想学习的看法具有更广泛、更个人化和更有意图的特征,例如包括对学习的渴望(例如,“学习包括自我学习的愿望……而不是为了满足老师或父母”),个人视角(例如,“学习是帮助学生发展自己新的视角,而不是引导他们走向特定的东西”),独立学习(例如,“当你自己学习时;你不需要老师‘扔’给你材料”),或实践(例如,“能够将你学到的东西应用到你的日常生活或未来生活中”)。

这些对比并不意味着它们没有任何共同之处。例如,无论是对于学校学习还是理想学习,学生都强调了获得新知识和实现深刻理解的重要性。认为乘客座文化的所有特征都完全不合适也不是重点。更多的是关于基调和精神的担忧。回到学习的游戏,学生们似乎在课堂上看到了你应该玩的学校游戏,通过随波逐流来相处,而不是更大、更真实的学习游戏。他们一直生活在持续的乘客效应中,而且他们也知道这一点!

\section*{驾驶员视角}

在开始打棒球时,你自然而然地希望尽早学习基本规则。但是,学习游戏的规则是什么呢?“规则”这个词可能并不完全适用,但至少应该有一些指导方针。经验丰富的学习者如何在学习游戏中开辟道路?

我一直对一位四年级学生的以下简洁描述印象深刻。多年前,一些同事在我们的研究中收集了这段描述:

首先我会问自己:这是什么?然后,我们为什么需要它?它是如何运作或如何发生的?例如,如果我看不懂一个词,我会读标题并思考标题的含义。然后我会把这个词的前后句子各读两遍。接下来,我会读那个句子,并用一个可能适合放在那里的词来替换它。
这是一位非常聪明的学习者!她自信的语气提醒我们,玩好学习游戏既是态度问题,也是技巧问题。想象一下,如果大多数学习者都拥有这种精神的方法和思维模式,那将是怎样的教育环境!

当然,学习者可以学习很多关于学习游戏的内容。其中一些我们已经在其他形式中见过了。关于隐藏游戏的第 5 章包括策略的隐藏游戏,并涉及了对学习者有益的阅读和问题解决策略。关于向团队学习的第 6 章包括对同伴问题解决(一种旨在培养元认知自我管理的技术)以及工作室学习(培养一系列学习和思维技能及态度)的回顾。

在一种被称为认知学徒制的广为人知的促进学习的观点中,Allan Collins, John Seely Brown和Susan E. Newman描述了如何将学生带入深入和自我调节的学习参与中。建模、辅导和脚手架是三个关键要素,教师通过建模战略实践、指导学生掌握这些实践,并提供“脚手架”,即提供支持并逐步撤回支持,以培养学生的自我管理能力,正如第 5 章所讨论的那样。类似精神的另一种实践包括学生保留学习日志,他们在日志中反思自己的学习。

积极提问是自主学习者的另一个特征。Marlene Scardamalia和Carl Bereiter报告说,如果仅仅是鼓励他们提问,年轻人就拥有提出深刻、广泛问题的非凡能力。事实上,他们对尚未学习的主题提出的问题比对已学习的主题提出的问题更好,这大概是因为正式的教学缩小了他们对主题的理解范围。鼓励广泛提问的课堂文化肯定会培养出广泛提问的学生。

除了这些想法之外,我们还可以添加记忆策略、时间管理方法、创造力练习、有效的练习计划、论证技巧、表达技巧、写作技巧等等。这些的不同版本捆绑在课程和书籍中,提供了大量的机会。事实上,针对这样或那样的事情,有很多好的学习实践,如果没有对整体的良好把握,很容易迷失在细节中。

因此,在本章的剩余部分,我将更多地关注自我管理学习和学习如何学习的总体精神和形式。那么,管理自己的学习是什么样的呢?为了有一个可以讨论的学习议程,让我们想象学习者正在学习exology,先不管这是什么——也许是对各种未知的研究,也许是与前任相处的艺术,也许是对字母 X 起源的历史研究。所以,我正在学习“exology”。考虑到整体学习的七项原则,我对自身学习的管理是什么样的呢?

\subsection*{玩完整的游戏}

即使没有提供完整的游戏,自我管理的学习者也会寻求对它的理解。如果我正在学习“exology”,我会尽早了解这门学科的全貌:exology家们试图做什么,以及他们是如何做的?我如何在边缘参与其中,也就是前一章所说的“合法边缘参与”?我如何才能完全承担一个初级版本的工作,仅仅是开始,但其简单的丰富性足以提供对完整游戏的一些感受?

以这种方式积极主动的学习者正在努力掌握完整的游戏,而不仅仅是等待它被摆在他们面前,或者满足于元素主义和泛泛而谈。有时让学习者坐在驾驶座上并鼓励他们探索和定义“游戏”的教师,正在帮助学习者变得积极主动。

\subsection*{使游戏值得玩}

也许我一开始对“exology”并不感兴趣,但由于某种原因我需要学习它。我能做些什么来培养我自己的投入和兴趣?以完整的游戏来对待它已经有所帮助,但除此之外,“exology”的哪些部分与我感兴趣的东西最相关?让我在学习“exology”的过程中尽量突出这些部分。对我来说,一个好的挑战水平是什么样的?既不会因为过于简单而无聊,也不会因为过于复杂而令人沮丧?让我找到并参与接近我最佳状态的初级版本。

总的来说,积极主动的学习者会努力使游戏对他们自己来说值得进行,而不是过多地依赖来自他人的碰运气式的灵感,也不是依赖奖惩的强迫。鼓励学习者在一定程度上掌控自己的动机的教师,正在帮助他们发展作为学习者的自主性。

\subsection*{攻克难点}

在这里,聪明的学习者会对他们自己说:我知道进行完整的“exology”游戏比孤立地找出难点并试图掌握它们更吸引人——但更吸引人并不一定更有效!是的,我知道教练或老师可能会为我分解它,但我可以自己做一些。那么,作为一名学习者,我的症结所在是什么?我在哪里感到困惑,我在哪里技能不足?我如何安排时间来攻克难点,并在此基础上将我提高的技能和理解重新整合到完整的“exology”游戏中?

总的来说,以这些方式积极主动的学习者不需要完全依赖老师、教练、课本或提示单来建立有针对性的刻意练习的严格制度。培养和期望这种责任感的教师正在帮助学习者发展他们的自主性。

\subsection*{异地作战}

当我在进行“exology”的一个初级版本时,我开始留意另一个版本。我不必总是等待别人给我布置作业。在某种程度上,我可以弄清楚我自己最好的下一步是什么。什么是新的变体、不同的风格、下一个挑战级别?如果“exology”有不同的角色,让我尝试一下。如果“exology”有不同的方法,让我尝试一下。如果“exology”内部存在关于什么是有趣问题的辩论,让我参与到这些辩论中,并形成我自己的问题意识。如果有人批评整个“外生物学”事业,他们会说什么,又该如何回应?

在追求这些问题的过程中,积极主动的学习者避开了教育中最常见的陷阱之一,即人们只接触到某个主题的单一标准版本,当他们走出官方课程进入混乱的现实世界时,会感到困惑。进行“异地作战”的学习者更有可能在以后广泛地迁移他们所学到的东西。鼓励个人建立联系的教师正在帮助学习者克服这样一种心态,即他们学到的一切都必须放在课本或讲座的盘子里端上来。

\subsection*{揭示隐藏的游戏}

学习的艺术一部分在于知道总有隐藏的游戏。会有“exology”的表面版本、简单的步骤、直接的玩法、正确和预期的答案。但在这一切之下,不仅会有隐藏的游戏,而且会有多个隐藏的游戏。积极主动的学习者不会总是等待老师或课本的启示。他们会说:我想留意隐藏的游戏。可能有一个策略层面,即作为问题发现者和问题解决者的自我管理,即了解各种情况的技巧。那是我想要了解的游戏。如果“exology”是一个学术领域,那么可能存在一个证据游戏,即什么算作好的证据和反证,以及证据的难题和陷阱是什么。这也是我想更好地理解的游戏。

总有隐藏的游戏——权力与竞争的游戏、欺骗的游戏、与基础探究相对的应用游戏。积极主动的学习者会寻找这些隐藏的游戏,而关心培养积极主动学习者的教师会给他们机会和鼓励这样做。

\subsection*{向团队和其他团队学习}

“exology”很可能不是一项天生的孤独追求。它依赖于合作,也许依赖于竞争。即使“exology”鼓励单独参与,仍然有很多东西可以从已经掌握这门艺术的其他人那里学习。即使我的老师或教练不这样认为,我可以问自己:我可以向谁学习?我的邻居在做什么?这位更有经验的学习者在做什么?我可以在哪里寻找导师?我可以教给别人我所知道的东西,从而更好地了解它?我可以与谁合作以推进更大的议程,并在这一过程中学到更多?

积极主动的学习者会将这些问题放在心上,并在环境允许和鼓励的情况下主动向团队和其他团队学习。关心培养积极主动学习者的教师不会总是微观管理团队学习模式,而是给学习者机会进行混合搭配并创造他们自己的模式。

\subsection*{学习学习的游戏}

当以自我管理的策略性方式学习“exology”时,我也会更广泛地关注我自己的学习实践。我正在组织我对外生物学的学习,但这说明了我可能如何学习原子物理学、双陆棋或视频编辑?我如何才能将在这里行之有效的方法应用到其他场合,并摆脱那些行不通的方法?

积极主动的学习者会问这样的问题……因为他们的老师会告知、鼓励、尊重并抽出一点时间来回答这些问题。

当然,这种对积极主动学习者的描述留下了一些问题,即何时以及如何将这些思维方式放在学习者的脑海中,并帮助他们变得专注和热情地运用策略。这是一个“驾驶员教育”的问题。

\section*{驾驶员教育}

几乎任何正规学习环境都是一个繁忙的地方,无论是夏季少儿棒球联盟、物理实验室、主日学校、童子军或女童子军营地、专业研讨会、学徒计划还是考古挖掘。我们已经研究了整体学习的其他六项原则,现在,学习学习的游戏又为我们提供了一个需要推进的议程。我们如何将其融入其中?

如果是课堂,就让我们教授一些策略性阅读技巧,以便学生们学会更好地管理阅读学习这一非常基础的“游戏”。让我们让学生们查看自己的测验结果,并决定他们需要把注意力集中在哪里,这样他们就能在攻克难点方面更加自主管理。让我们介绍结对问题解决(第 6 章),作为学生可以自行组织的策略,鼓励他们自主管理向团队学习。而且,我们不要忘记记忆策略、时间和压力管理原则、应试技巧、积极倾听、记笔记、复习方法、检查自己作业的策略等等。

我想起了单口喜剧演员Steven Wright的一句简洁的俏皮话,“你不可能拥有一切!你把它放在哪儿?”他完全可以把教学也考虑在内。在追求诸如编织、读写能力、哲学、生物学等游戏的过程中,我们哪里有时间也去追逐学习的游戏?

这里至少有两种不同的选择。一种选择涉及结构,“你把它放在哪儿”,即将学习如何学习作为一个单独的时间段——比如每周五上午半小时的迷你课程——与将学习如何学习融入现有课程进行对比。另一种选择涉及风格,“你把它放在哪儿”,即将学习如何学习的原则以口头、海报和黑板上的形式明确地展示出来,与仅仅围绕学习者创造一种默认的驾驶座文化,从而隐含地鼓励学习如何学习进行对比。

让我们先来看第一种:单独的时间段与融入现有课程。当然,没有唯一的答案,但一些见解来自思考技能运动的历史,该运动也面临着类似的困境。需要明确的是,学习如何学习与思考技能并不完全相同。即便如此,它们彼此之间有很多关联,而且它们都必须面对史蒂文·赖特的“你把它放在哪儿”的挑战。

为了回答这个问题,研究人员和教师探索了两种非常广泛的策略,通常称为独立方法和融入方法。独立方法主张创建一个单独的课程,并根据内容范围和可用时间,每周教授一次到几次。这样的课程需要仔细关注学习的迁移——异地作战——以便这些想法和实践能够迁移到课程之外的环境。相比之下,融入方法建议将这些想法和实践融入到常规的教学模式中。线性方程、乔叟作品、罗马历史的学习都融入了对更普遍的态度和策略的关注。

许多教师倾向于融入方法,他们认识到学科教学可以利用更用心的关注来促进,珍视由此产生的丰富的学习,并意识到融入另一条线索可能比为一门全新的课程寻找时间段更容易。所有这些都是明智的考虑。然而,这个决定比看起来要复杂。虽然融入在原则上听起来很有吸引力,但人们很容易因为普遍的忙碌而忽略对思考和学习的一般技巧的预期关注,并将其简化为一种象征主义,这里一个提示,那里一个提示,偶尔一次简短的练习。一个做得好的独立方法的优点是,它可以保证用于学习思考和学习的有针对性的时间。

那么,这两种方法的记录说了什么?据我所知,没有系统的研究将干预的独立版本与同一干预的融入版本并列比较。事实上,进行此类调查将很困难,因为独立与融入的特征非常不同。尽管如此,围绕思考教学的普遍经验总结给出了一个清晰但不那么果断的答案:两者都可以很好地发挥作用。这取决于在上下文中什么是最可行的。

这对整体学习,尤其是学习学习的游戏意味着什么?首先也是最重要的,这是对评估上下文的建议。如果在特定的学校,在特定的时间,特定的教育工作者有远见和精力来组织和维持一门关于学习如何学习的课程,那就让它发生吧。如果在其他地方,融入方法似乎更容易部署,那就让它发生吧。只要有认真的事情发生!

所有这些都得到承认,当有选择时,我倾向于采用融入方法来学习学习的游戏,或者将独立要素与强烈的融入相结合。为什么?整体学习直接而明确地关注学术学科、专业实践、技能和技艺以及体育和游戏的学习。整体学习不仅仅是关于学习学习的游戏,还关注前六项原则。为了使整体学习或具有类似精神的东西拥有动力,这些原则需要每时每刻、日复一日地蓬勃发展,使环境成为鼓励学生掌控自己作为学习者的自然环境。

现在让我们来看第二个选择,显性与隐性。一些教育工作者认为,一般原则、风格和实践模式最有效地是通过耳濡目染来学习的。培养更好思考和学习的最佳方式是简单地让学习者以深思熟虑的方式进行学习——提问、辩论、努力解决复杂的问题、尝试进行深刻的分析,并经常在适当的帮助下找到他们自己得出结论的途径。显性反而会干扰而不是帮助。把策略放在抽屉里,把概念放在架子上,直接去做就行了!

这种吸引力是自然的,但关于什么有效的研究却走向了另一个方向。虽然没有对独立与融入进行直接比较,但有一些关于思考教学的研究将隐性方法和显性方法并列比较。显性方法获胜。隐性方法的问题是,许多学生没有领会到其中的信息。第 5 章探讨了任何努力表面之下的隐藏游戏,简要地提到了伯克利的Alan Schoenfeld的工作,强调了自我管理策略对问题解决的重要性。Alan Schoenfeld还探讨了对策略的默认关注与显性关注的影响。

在一项小型但控制严格的研究中,一些大学数学专业的学生参加了一系列五次课程,学生们在课程中尝试解决问题,然后看到了使用几种强大的问题解决策略演示的解决方案,但没有命名或解释这些策略——希望通过耳濡目染来起作用。其他学生参加了类似的课程,在这些课程中,策略被标记出来,并在执行关键步骤时进行了解释。这些课程的主题和问题相同。

所有学生在教学前都进行了一次包含几个问题的测试,然后在教学后进行了一次包含不同问题的测试。隐性条件下的学生在后测中的表现并不比前测好。但显性条件下的学生表现好了一倍以上。除了这项研究之外,还有其他具有相同效果的发现。

但仅仅因为显性重要并不意味着隐性不重要。那些被隐性方法吸引的人肯定是对的,即普遍用心的驾驶座学习文化的力量。事实上,在明确地教授思考和学习的实践,而教育过程的其他部分却很少为它们留下空间时,存在一种不诚实。我们不希望显性取代默认,而是希望不时地强调它,就像海豚偶尔跃出水面一样。如果整体学习通常在进行中,那么一点点就可以起到很大的作用:这里一个标签,那里一个短语,也许是墙上的一系列原则,一个快速的解释,一个鼓励的瞬间,偶尔的练习,三分钟的演示,五分钟的汇报,所有这些都是为了学习学习的游戏。

相比之下,无论关于思考和学习的想法如何明确,许多学习环境的文化都会隐含地反对它们,这不是通过一些故意的阴谋,而仅仅是因为照常营业通常最终会为它们留下很少的空间。学习过程是按照元素主义和泛泛而谈的方式组织的,即掌握关于学习者可能正在进行的游戏的零星的知识和技能。即使有一个完整的游戏在进行,大多数时候学习者也可能只是被微观管理的乘客,做他们被告知的事情。因为他们不必为自己的学习承担任何驾驶责任,所以他们没有学到很多关于如何自己驾驶路线的知识。

相比之下,选择要点、自我评估、尝试不同的角色、开放式项目、从辩论到对话到合作倡议及其他不同的互动风格、学习者自己采取的不同挑战级别——这些要素是驾驶座文化的主要支柱。为了帮助学习者更周到地参与完整的游戏,人们甚至有时可以提供多种进入方式,让他们选择。有些人可能喜欢先听一位经验丰富的exology实践者谈论它。其他人可能喜欢作为边缘参与者谨慎地涉足行动。其他人可能喜欢尝试一个非常初级的 exology版本。或者暂时转向攻克难点的原则,让每个学习者负责弄清楚他或她自己的难点和易点是什么。或者提到异地作战的原则,要求学习者探索和阐明个人联系,他们可能会如何运用或玩转exology。
总而言之,对于学习学习的游戏,倾向于采用融入而不是独立的方法可能是有意义的,但如果独立在上下文中更可行,那就很好。无论哪种方式,明确的原则都可能很好地为学习者服务,而不是作为需要记忆的教义问答,而是作为反思和行动计划的对象——关于诸如策略性阅读的具体原则,以及像整体学习原则本身这样的普遍的首要原则。但是,这一切都不应该取代通过整体进行周到学习的强大的默认文化。没有驾驶座文化,学习的游戏就没有地方可玩。

忠告、鼓励和适当的自由空间的良好结合意味着学习者可以找到自己进入游戏的方式,并在游戏中走得更远。这并不意味着绝对的自由,而是适度的自主权,它既支持又释放,既引导又允许,既塑造又容许。

\section*{驶向未来}

如果我们除了非常迫切的需求之外,在正规教育结束后就停止学习新思想,我们将过上多么奇怪的生活。是的,我们学习了关于总统、国王、方程式、轨道、细菌、十四行诗,甚至棒球的知识。到目前为止还不错,但作为一个多年前就完成了所有毕业典礼的人,如果那是终点,我会有很多东西知之甚少。

以下是我一路走来不得不填补的一些空白:生态学和地球的保护、复杂性和系统思维、其他文化特别是西方文化以外的文化历史、种族和民族之间复杂的紧张关系、当代对宇宙及其构成的概念、特殊利益集团在政治中令人困扰的角色、共产主义的兴衰以及罗马帝国的兴衰。其中一些主题是我在正规教育期间没有遇到的,一些主题只是略有提及或被忽略,还有一些主题当时尚未突出。
现实情况是,当我们手持学位走下讲台时,我们大部分需要学习的东西仍然在我们面前。这不仅包括学术知识领域,还包括专业领域的理解、生活的人际关系维度、与其他文化的思想和艺术的邂逅等等。

不仅如此,我们作为个体最终可能需要知道的确切内容是未知的。一方面,每个人需要知道的东西很可能大相径庭。另一方面,在我们快速变化的世界中,没有人或几乎没有人知道在未来一二十年里我们中的许多人需要知道什么。一些值得学习的游戏甚至还没有被发明出来。这就是为什么整体学习的最后一条原则也许是最重要的:学习学习的游戏。

\section*{学习的奇迹:学习学习的游戏}

我在思考如何培养学习学习的游戏。最广泛地说,我最好避免“乘客效应”,并利用“驾驶员效应”;除非学习者经常坐在方向盘后面并进行一些自我指导,否则他们不太可能学会学习的游戏。

我在思考应该鼓励什么来形成驾驶座文化。我可以培养允许学习者拥有相当大的自主权和选择权的互动模式,从而促进反思和自我管理。

我在思考如何防范乘客座文化。我可以尝试避开一些典型的危险:肤浅而不是深入的学习方法、“要么你行,要么你不行”的思维模式,以及认为学习只是接受、顺从、有条不紊以及只追求传统的掌握和好成绩的期望。

我在思考究竟需要学习什么才能学会学习的游戏。在这里,我可以求助于整体学习的七项原则,这些原则为学生自我管理学习提供了一个广泛的框架,而不仅仅是为了我管理教学。除此之外,我还可以帮助学生掌握许多特定的技能:良好的阅读习惯、时间管理、问题解决策略等等。

我在思考如何为所有这些找到空间。有这么多事情需要我管理,“驾驶员教育”会是什么样的?独立或融入的方法都可能有效,但如果可能的话,融入或混合的方法似乎更好。我还想记住,良好的学习策略受益于明确的关注,而不仅仅是从普遍积极的驾驶座文化中耳濡目染。但是,如果没有围绕它们的驾驶座文化,它们就不会蓬勃发展。

我在思考学习学习的游戏是否值得付出努力。然后我想:对于在变化的世界中漫长的一生来说,学习的游戏可能是最值得学习的游戏。
