\chapter{玩整个游戏}
听说过“望山跑死马”么?这就是我作为一名博士生写毕业论文的经历。从远处看,这座山似乎并不那么可怕。但真跑起来,却仿佛怎么也到不了。

我在MIT 取得学位,主修数学。本科毕业后,我继续攻读博士学位,并对用数学方法研究人工智能产生了浓厚的兴趣。人工智能研究如何让计算机进行智能活动,例如下棋、证明数学定理或控制机器人做有趣和具有挑战性的事情。对人工智能的研究激发了我对人类思维和学习的兴趣。完成学业后,我进入了认知心理学和教育学领域,此种原因另当别论。现在,你可以想象我在论文的山脚下,思考着应该尝试什么样的人工智能研究。

这里的问题是发现问题。解决问题和发现问题有一个非常有用的粗略区分。解决问题是一门艺术,是一种处理已经相当清晰的问题的技巧。有时候它们会在一本书中被发现,有时它们会在日常生活中会作为需求冒出来。无论它们来自哪里,它们就在那里。我们钻进去,我们挖掘它们。虽然它们的轮廓很清晰,但并不意味着它们很容易解决。例如,在Thomas Edison最终破解电照明的问题之前,人们早就认识到这一问题并有许多发明家一直在研究它。像Fermat最后定理这样的经典数学猜想,在被解决之前甚至会存在几个世纪。

发现问题则是另一回事。发现问题首先要弄清问题是什么,其次还涉及对问题进行良好的表述以使问题易于理解。通常,在解决问题的过程中,由于怀疑自己所研究的问题可能并不完全正确,还可能需要对问题进行重新定义。

我的论文问题就是发现问题。我真的不知道该如何去寻找一个好的课题。我有一套很好的技术知识工具包,在解决问题方面很有能力,甚至很有创造力,但发现问题却是另一回事。

我在想,为什么是这座山?我回想了一下自己在MIT的本科和研究生学习经历,意识到了一件当时让我感到惊讶并一直伴随着我的事情:在我的技术课程中,我除了解决问题,很少做其他事情。我几乎每次都能成功,但问题都来自课本或导师。我从未承担过类似的项目或开放式的研究。结果不可避免:我拥有大量解决问题的技能,却几乎没有发现问题的技能。

我在人文学科方面的经历则截然不同。MIT在文学、哲学、音乐和其他领域都有很强的实力,并有著名的教授。我对人文学科有广泛,选修了各色课程,在这里发现问题犹如家常便饭。一门课程的主要作业通常是写一两篇题目有很大自由度的论文。我经常自问:什么样的问题值得探究,我是否能提出一个很好的论点,到哪里去寻找相关资源,以及如何将这些问题整合成一个令人信服的论述。

我在此申明:MIT为我提供了极好的本科和研究生教育,给予我慷慨的支持和灵活性。能在那里学习是我的荣幸,我学到了很多东西,而这些东西自那时起就被证明是既有趣又有帮助的。说到我在MIT的经历,只是想指出一个难题,即解决问题与发现问题。

这是一个关于玩整个游戏的难题。毕竟,发现问题是整个游戏的一部分。看看你想要的任何正式教学,任何学科,任何年龄,简单估测一下:是否涉及发现问题?如果没有,你就可以确定,学习者没有在玩整个游戏。

\section*{对完整游戏的追求}

当我思考学习者玩整个游戏的样子时,我想到了我所认识的那些将整体游戏作为其教学策略的教师们,我想到他们是如何创造性地发明和改编整体游戏以服务于学生学习的。Lois Hetland就是其中的一位,她现在是教授和我的研究同事,但几年前还是正在参与“面向理解的教学”R\&D项目的一名七年级教师 。

Lois当时正在教一门重点为美国殖民时期的综合人文学科,她围绕几个基本问题组织学生的任务,班里同学全年都在思考这些问题。其中一些问题集中在土地的作用上:土地如何塑造人类文化?人们如何看待土地?人们如何改变土地?其他问题则探讨了棘手的历史真相问题:我们如何查明很久以前或很远的事情的真相?我们如何看穿资料来源的偏见?

Lois将所有这些问题称为“纲”\footnote{Throughlines。中文有“纲举而目张”的说法,正与作者的意思相通}。这是Constantin Stanislavsky表演学派的方法的一个典故,指贯穿整部剧的中心主题。Lois Hetland强调,无论讨论什么主题,都要让课堂回到这些“纲”上来。这样做的目的是加深对美国殖民地时期的理解,但更多的是让学生了解探究的特点和节奏,以及学生对自身学习的管理。

关于面向理解的教学思想,我还想到了天才英语教师Joan Soble。Joan不知道该为一群被认为是高危学生的九年级学生做些什么。用她的话说,他们被学校的要求“永远压制”。她为他们设计了一门写作入门课程,课程体验涉及各种活动,其中包括通过拼贴画为写作做准备,以批判的眼光辩护和审查作品,以及阐明和追求个人目标。在关注个人目标的过程中,学生们由一份表格作为辅助,该表格列出了他们可能想要提高的各种写作技能。换句话说,就是要攻克难点。这些技能包括句子结构、修改方法以及更好地管理自己的工作模式的策略。

读者可能会想起我刚提及的在MIT的经历,并推测:整体游戏在人文学科中比在数学和科学中更容易起作用。然而,在这些学科中找到例子也是俯拾即是。Chris Dede是哈佛大学教育研究生院的一名研究员,他一直致力于研究和开发科学的学习方法,以及如何让学生在学习科学方法的同时动手实践。他和同事构建了一个名为River City的 MUVE 。青少年在网上玩的许多流行游戏都具有这一特点:参与者在虚拟世界中穿梭,这些虚拟世界由头像图标代表,他们会遇到其他玩家并与之互动,而这些玩家可能身处北京、开普敦或里约。

在River City MUVE中,学生们面对一个问题:各种疾病正在席卷虚拟人群。病因是什么?在探索River City的过程中,学生们可以在不同地点进行观察、测试水质,并通过其他方式调查流行病的可能来源。在这样做的过程中,他们不仅学到了一些科学知识,还参与了科学探究过程本身。

West Virginia州的 Kenna Barger 是2001 American Teacher Awards的获奖者之一。在我的同事Ron Ritchhart制作的有关创造性教学的视频光盘中,可以找到她教九年级代数的精彩片段。她带领学生进行水球蹦极,其大纲由Arizona大学的一个名为 M-PACT(learning Mathematics with Purpose, Application, Context, and Technology,有目的、有应用、有背景、有技术地学习数学)的项目制定。

水球蹦极是一个完整的数学建模练习。九年级学生已学过线性方程。活动开始时,他们组成小组,测量带有砝码的橡皮筋的伸缩性。各小组利用代数建立一个模型,说明多少重量产生多少拉伸。这项活动没有套路也不是公式化的。学生们在因变量和自变量以及如何表达条件等问题上纠结不已,而Kenna Barger则在一旁巡回指导。之后,整个班级来到室外,各小组依次从学校屋顶投下系着橡皮筋的水球——这就是水球蹦极部分。学生们已经用方程式预测了多大的弹力能将气球带到恰好离开地面。团队中的一名学生通常会躺在下降的球下面。学生面临的挑战是让球尽可能接近地面而不能弄破球(或学生)。整个游戏包括将实验与使用线性方程的数学建模结合起来,尝试理解整个系统是如何工作的,并做出有效的预测。

Barger强调,这只是为期一年的代数教学工作的一部分,他认为代数不仅是一个操作符号的抽象系统,还是一个数学建模的过程。Barger评论说:我是学生的时候总是教室后面那个讨厌的人,总问“为什么?”,直到我开始在一所强调现实世界成就和教师与学科合作的学校任教,我才真正得到了这个问题的答案。

这样的例子并不难找。许多例子都可以同一张光盘中找到,或在《Teaching for Understanding》一书中找到,或在教育界可获得的其他无穷无尽的资源中找到。那么,玩整个游戏的标志是什么?我们怎样才能知道我们是否玩了一个完整的游戏呢?

在学习环境中,一个完整的游戏通常是某种广义上的探究或表现,它涉及解决问题、解释、论证、证据、策略、技能、技巧。通常会创造出一些东西——解决方案、图像、故事、文章、模型,如此之类。

它绝不仅仅是内容,它让学习者尝试更好地做某件事。Joan Soble的学生在努力提高写作能力,Lois Hetland的学生试图更好地理解殖民地时期的美国并进行历史探究,Kenna Barger 的学生努力提高数学建模能力。

它绝不是套路,它需要用你所知道的知识去思考并进一步推进。它涉及的不仅仅是标准的常规问题,而是开放式或结构不严谨的问题。写作、反复思考“纲”、模拟水球下落,所有这些工作都要求学习者超越已有知识,外推到新的和苦难的情况。
它不仅是解决问题,它也包括发现问题。在Joan Soble的写作课程中,学生们设定了自己的目标。在美国殖民地课程中,Lois Hetland希望她的学生们能够帮助她在新的主题背景下,对“纲”进行细化和诠释。Kenna Barger的水球项目可能是最明确的,但即使在这种情况下,也有许多不同的方法。

它不仅是正确答案的问题,它还涉及解释和判断。所有的学习者都必须解释和判断他们在做什么,以及他们是如何走到今天这一步的。

它不是没有感情,它涉及好奇、发现、创造和友情。Kenna Barger的学生们在水球任务中以一种善意的方式进行竞争,努力让这些线性方程做一些事情。Joan Soble的学生投入到写作中,渴望做得更好。Lois Hetland 的学生发现,他们对美国殖民地的好奇心一再被激发。他们不仅在学习,还在培养学习的态度,比如好奇心和毅力。当然,并不是每个学习者都会对所有的事情感兴趣,但大多数学生都能在一定程度上产生兴趣。

它不在真空中,它涉及一个或多个学科或其他领域的方法、目的和形式,处于一个社会背景中。Joan Soble的学生以合作的方式处理写作的方法、目的和形式。Lois Hetland的学生在讨论历史探究的方法和目的时,在对话和写作中使用了适当的论证和解释形式。Kenna Barger 的学生以小组为单位处理数学形式和实验问题。

这些都是整体游戏的标志,也可以作为构建整体游戏的指南。你可以从任何地方开始,比如说,从分数算术的例程或几条语法规则开始。虽然还没有看到完整的游戏,但一些问题已经指引了正确的方向。问自己:如果这个主题不仅是关于内容,而是要让学习者能更好地做某件事,那么它应是什么样的?他们会在什么方面做得更好?问自己:如果这个话题不仅是套路,而是需要用你所知道的知识进行思考并进一步推进,那么它应是什么样的?问自己:如果涉及到发现问题,问题会出现在哪里?对类似问题的每一个回答都会在最初有限的主题周围画出一个更大的圆圈。随着圈子的扩大,不难得出整个游戏的合理图景。

\section*{整体游戏的分类}

正如好的答案不止一个,整体游戏的好版本也不止一个。例如,围绕历史就有许多满载思考的游戏。学习者可以通过仔细研究原始资料形成猜想并为之寻找证据;学习者可以对比不同的历史记载,甚至是不同国家的教材,以发现共性和差异并思考差异是否反映了偏见;学习者可以考察凯撒在罗马掌权等关键事件或凯撒时代罗马日常生活的特点,然后比较当时和现在的权力争夺,或者比较当时和现在的日常生活。

“游戏”在这里并不像棒球或国际象棋那样定义明确,事实上也不必那样。任何一门学科都有大量的实践用来“游戏”。尽管专业人士有时候会争论哪种做法是正确和恰当的——做历史、经济学或文学分析的正确方法——但我们不必担心这些。玩整个游戏的挑战不在于找到唯一正确的经典版本,而在于将一些合理的版本付诸行动。第5章详细介绍学科思维模式。

有时,游戏是整合性的。它跨越一系列学科,将多个学科的思想交织在一起。一个关于社区进行生态调查的班级项目,在过程中可能要应用生物学的概念,用数学来描绘问题和趋势,用阅读和写作技巧来综合结果并提出社区行动计划。一个关于将艺术用于政治目的小组调查,可能涉及研究几个正面和反面的案例(例如南非的抗议艺术、纳粹宣传),考虑文学和美学价值,识别政治操纵,并用统计数据估算其曝光率和影响力。

社区生态调查和政治艺术小组调查是项目式学习的范例,也是组织全面学习的几种方法之一。项目式学习有许多实例,George Lucas教育基金会维护的Edutopia网站提供了大量的视频示例。

根据项目式学习的定义,它要花一些时间来完成的大整体。但整体游戏并不一定是一个大游戏!我们应认识到,在某些教育环境中,大游戏因其时间安排和任务规定而并不能很好地发挥作用。小游戏总是有其存在的空间,在小游戏中,整体学习可以进行得相当快。看一首诗、一件艺术品或一篇报纸社论,对其进行反思和讨论,就是一项有完整意义的活动,半小时就能完成。

另外,整体游戏往往并非一次性完成的,而是分阶段进行的。Lois Hetland的学生们反复回到他们的“纲”,越来越深入地探究同一个问题。学生们试图在River City环境中找出疾病的源头,并多次进入游戏。

我们熟悉的其他一些具有整体游戏特点的做法包括:基于问题的学习、基于案例的学习或案例研究法、社区行动倡议、角色扮演情景剧、正式辩论和工作室学习。这些方法各有特色,但很难做出清晰的区分。同一个例子通常可以用来佐证其中的多种做法。这里我只想再谈三种。

角色扮演情景剧,也是一种可以相对简短的做法,是在某一领域培养观点和开放性思维的好方法。你可能认为自己知道自己的核心价值观是什么,如果是你管理公司或管理国家会怎么做。然而,当学习者置身于角色扮演情景中时,他们的新态度往往会让他们大吃一惊。观念模式不仅是我们所持价值观的产物,还是我们所扮演角色的产物。

在基于问题的学习中,学生以小组为单位共同解决问题。根据问题的范围,一个问题需要一节课或更长的时间。这些问题被故意弄乱,一般来说也没有完美答案。学习者需要在已知之外寻找信息进行学习,教师为这一过程提供便利。前面的水球蹦极和River City的例子,都可以看作是基于问题的学习。

由Vanderbilt大学的学习技术中心开发的 Jasper Woodbury数学问题解决系列就是一种基于问题的学习。这种方法使用锚定指导,提供一个可带入课堂作为问题背景的生动场景(即锚)。Jasper Woodbury系列以十几个视频为中心,主人公Jasper Woodbury出现在处理各种需要数学推理的情境视频中。

例如,在该系列的第一个视频中,学生们看到Jasper开他的船去上游检查并最终购买了一艘更大的新船。由于新船的夜行灯无法工作,Jasper必须计算出他能否在日落前将新船开回他家的码头。学生们要处理这个问题,必须考虑日落时间、距离、汽油消耗量、一箱汽油是否够用、如果不够用Jasper可以在哪里获得更多汽油以及其他因素,包括一些只能猜测的缺失信息。在整个视频中,相关信息会在不经意间出现,如随口的评论、河边的标志、报纸等,而这些是与无关信息自然地混杂在一起的。学生通常以小组为单位,通过前后翻视频来寻找所需的事实。研究表明,Jasper Woodbury历险记提高了学习者解决数学问题的灵活性。

基于问题的学习的另一个常见应用是医学教育。在医学教育中,准医生们不是坐在教室里听解剖学和生理学的长篇大论,而是以小组为单位处理他们还不太了解的疾病的模拟病例。这是初诊的症状,你认为可能是怎么回事?你需要从哪里寻找答案?您需要了解和掌握哪些解剖学和生理学知识?学生们把问题分门别类,找出一些答案,互相传授,然后得出一个试探性诊断并找出进一步检查的方法。与技术讲座相比,基于问题的学习更有可能在积极运用知识的基础上培养诊断推理能力。

在大学层面,基于问题的学习也可以被视为一种基于案例的学习。哈佛大学商学院的David Garvin教授就是在这种语境下对哈佛大学的三个专业学院——医学院、法学院和商学院——的案例教学法的使用做了对比。Garvin强调,不同的背景设定孕育出了案例教学法的独特案例。医科学生以小组形式专注于诊断过程,成员在一定的帮助下进行自我管理。法学院的学生大部分时间都是单独工作,以大班形式上课。教授随机召集学生了解案件的事实和问题并开展全班讨论。学生之间互不交谈,大部分直接互动发生在学生和教授之间。讨论的重点是案件的关键特征以及微小的差异如何产生巨大的法律影响。在商学院,学生以个人或学习小组的形式准备参加全班课程,商业案例通常会提出问题情境并要求采取下一步措施:如果你是老板,你会怎么做?学生需要通过详细的分析和论证来支持自己的想法。第一个发言的学生——突然被叫到,或者最多在几分钟前被提醒——要对全班做一个 5~10 分钟的演讲。

Garvin指出了这三种不同版本的案例教学法的特质和局限性,并记录了三个专业学院是如何努力改进案例教学法的。三个学院的共同点是让学生参与适合其专业的推理:医学诊断、辨别案例特征的法律含义、以有理有据的决策和计划处理商业问题。

我希望这个简短的综述能清楚地表明,具有整体游戏意味的学习有许多变体,有个有明确的名称,如基于问题的学习、基于案例的学习等,有些是独具慧眼的教师设计的组合活动模式。它们中的大多数都有或长或短的版本。整体学习简单明了,只要选择一种方法并加以运用就可以了。

别急!俗话说,魔鬼在细节中。上述任何做法都把学习者代入到一场完整的游戏中。然而……

不仅仅是形式,还有内容和思维。当你决定采用基于问题的学习方法时,这只是开始。问题是什么?要培养什么内容和技能?要培养什么样的思维——筛选历史证据、发现因果关系、采用不同的视角?基于问题的学习或任何其他类型的学习的一般理念对这些问题只字未提。构建一个完整游戏的主要挑战不在于选择一个像基于问题的学习这样的框架,而在于用对游戏的深刻理解来充实这个框架。

不仅仅是玩整个游戏,还有其他六项原则。基于问题的学习、基于项目的学习,都有好的或差的版本。怎样才能使游戏值得一玩?将难点突出来、集中关注,然后重新整合了吗?出城去玩如何实现以加强学习迁移?有哪些举措可以揭开隐藏的游戏?

最后,这不仅是发现式学习。不经意的阅读可能会让人觉得,学习者在这些实践中进行的是相对自由的开放式探究。其实不然!这些参与学习的模式一般是有结构的,通常涉及大量的前期信息,例如,商科学生在预习时阅读的书面或多媒体商业案例中包含预期的互动节奏、谁在何时与谁交谈以及发展阶段:首先会发生什么,接下来会发生什么,最后会发生什么?

Paul Kirschner、John Sweller和Richard Clark总结的一系列研究警告说,自由形式的实践对领域的初学者来说效果并不好。某些版本的基于问题的学习、基于项目的学习可能过于松散,尤其是在学习者刚开始学习的时候。学习者需要清晰的、经过实践检验的例子和强力的指导,然后才可以逐渐淡化。

通过整体游戏进行学习的意义并不在于把学习者从课本中解放出来,去进行个人探索。整体游戏的意义在于让学生参与到我们真正希望他们做得更好的事情中去。即使在大学阶段,初学者也不能直接从医院、法庭或会议室开始。在低年龄时,初学者要想了解日报、《麦田里的守望者》或当地河流的污染情况,更不是从深奥的论文和统计分析开始。那么,他们可以从哪里开始呢?这就是初级版本的问题。

\section*{对初级版本的追求}

Chris Dede的学生不是在寻找真正细菌和毒素,Kenna Barger的学生不是在Cape Canaveral发射火箭,Lois Hetland的学生不是在历史档案中钩沉美国殖民时期的原始档案,Joan Soble的学生也不是在为《The Atlantic》杂志撰写文章。他们不是在打真正的棒球,不是打满九局,不是九人组队,不是按照条例规则。

他们的努力之于真正比赛之间的关系,就如后院棒球之于正式棒球一样。初级版的技术要求低、时间短、常以模拟的代替真实的。例如,模拟的案例文件、模拟的整个环境(如 MUVE)、历史文献的重印本。不过,这些初级版本的游戏都能捕捉到全本游戏的一系列基本结构特征。它们要求探究、问题发现、判断、解释,实际上就是前面列出的所有特征。初级版本是整体学习之所以实用和强大的关键。在上一章中我们指出,教育始终面临如何处理复杂性这一根本问题。每位教师、每本教科书、每位家长、每位教练都必须找到应对这一问题的方法。比较直接的解决方法就是先打好基础与多多了解,但它们往往会恶化成"基础炎"和"了解炎"。

更好的解决方案是初级版本,之所以更好是因为初级版本从一开始就让学习者有意义地参与到整个游戏中,并将零碎片段有意义地置于更大的视野中。理想情况下,初级版本可以为学生提供引言中所说的“阈值体验”,导引他们进入棒球、历史探究、写作、数学建模或其他方面的新世界。我从一个专注于大学阶段学习的研究机构借鉴了“阈值体验”概念。这项工作最初是由Ray Land和Jan Meyer发展的,主要思想是认为存在一些阈值性的关键概念,一旦学习者理解了它们,对一门学科就会有更深、更广的认识。在整体学习中,我想强调的不仅仅是阈限概念,还有阈限体验。

对JoanSoble, Lois Hetland, Chris Dede, Kenna Barger及其他老师、导师、家长或其他正式和非正式教育从业者来说,精选适合初学者的初级版本是一门艺术。该艺术的一个要点是在丢掉不那么重要的东西的同时无伤游戏的总体精神和形态。该艺术的另一个要点是替换,例如,MUVE模拟装置、历史文献复制品。该艺术的另一个要点是保持适当的挑战性,别把初学者当专家。游戏的规则应大致相同——人们通常不会在缩小了的4×4棋盘上学习下棋——但难度不应劝退。游戏制造商也接受了这一原则,Monopoly Junior, Junior Scrabble和Clue Junior等流行棋盘游戏的初级版就是明证。

在寻找一个好的初级版本的时候,综合考虑舍弃次要、对象替换和保持合理挑战性,不仅要考虑到方便性,还要考虑到教师对学习者已经掌握的知识的了解程度,这样才能决定怎么走下一步。这要求我们关注学生的年龄和历史,关注他们实际学到了什么,以及关注他们学习的敏捷程度——这也是差异化教学的出发点。学习者可以通过多种差异化方法和层次参与整体游戏。

已有知识是学习者赖以学习的平台。如果青少年在理解 方面还有困难,那么要求他们成为缜密的策略性阅读者就没有多大意义。当青少年不知道什么是线性方程,让他们用线性方程建立数学模型也没有多大意义。那该怎么办呢?

通常的方法是从基础着手。相反,整体学习建议重新考虑学习者已经具备玩什么级别初级游戏的能力。在理解方面有困难的儿童尽管还不能有策略地阅读文本,但他们能有策略地倾听。从这里开始,读给他们。对刚开始学习代数的学生,可以用表格、图形和基本公式建立简单的模型,激发他们的兴趣。日峰值用电量如何随着日峰值温度变化?根据消费者数据,小幅提价会减少多少销售额?对廉价商品和奢侈品的影响是否相同?候鸟的大小与平均迁徙距离之间有什么关系?在这里起作用的就是所谓的问题发现:学生从这些问题出发,搞定如何将这些问题具体化,甚至提出自己的问题。

如果因为学生缺乏构成性技能妨碍了公认初级游戏的使用,也别放弃,又退回“基础炎”。更初级一点儿!这并不意味着要停止理解、代数或任何其他构成技能的基础学习;相反,当这些活动被视为对持续改进的整体游戏的下一阶段有贡献时,它们会更有意义。

与学习者已有知识相关的是发展就绪问题。在这里我不打算深入探讨特定的发展理论和实践。一方面,它本身就是一个整体世界,有许多资源可供教育者使用。另一方面,教师和研究人员一再发现,对不同年龄段的儿童能做什么和不能做什么作出绝对的判断是有风险的。只要任务布置得当,使用熟悉的材料,避免使用儿童可能误解的语言,提供提示和暗示,儿童往往会表现出比预期更高的技能和洞察力。这在很大程度上取决于选择一个好的初级版本!在第5章中,发展主题将再次出现。现在,我们只需要对孩子们在从幼儿园到高中及以后的学习过程中的知识、理解和自我意识的发展有一个基于经验的粗略认识就足够了。

现实情况是,在你设计一个初级版本的游戏时,你要对学习者已经掌握的知识和他们的发展水平做出最明智的猜测。你要先制作一个初级版本,通过试玩来评估它是太难、太简单还是恰到好处。第一次试玩对你和学习者来说都是一次学习,因为在某些方面你肯定是错的。这当然是我作为教育工作者的经验。只有在真实的情境中与真实的学习者一起工作二三周,我们才能期望找到真正校准的良好初级版本。

但如果根本就没有初级版本呢?如果我们能做的最好的事情就是先学习基础,直到学习者掌握了足够多的基础呢?事实上,很多事情不都是这样吗?

例如,你可能认为游泳就是一个很好的例子。几乎没有人跳进湖里就会游,甚至笨拙地、断断续续地游也不可能。我学习游泳的方式,也是大多数人学习游泳的方式,似乎就是基础优先:双脚站在水里,一直走到没腰处,弯腰,脸浸入水中,侧着头练呼吸,练划水;或者扶着杠,练习各种踢腿动作;或用水翼支撑。

其实,传统的游泳教学并不像表面看起来那么基础优先。首先,也是最重要的一点是,学习游泳的儿童和成人,无论他们自己的能力如何,都会对游起来的整个表现有所了解。他们经常看到别人游来游去。相比之下,三年级的孩子在学习算术时,通常根本不知道数学到底是用来干什么的,即使是初级版本的数学。

其次,握住杠练习踢腿和呼吸就是初级版本。它是如此的初级,以至于你甚至不能让自己保持漂浮,但你在那一刻以一种协调的方式做了除了防止你下沉的握力之外你能做的一切。其他早期游泳练习也是如此。从一开始,人们就在努力将各个部分组合在一起以免溺水。

如果游泳与通常的教育工作相去甚远,那么不妨再考虑一下早期阅读。对游泳的抱怨同样适用于阅读:当青少年连理解都做不到时,我们怎么能让他们进行整体意义上的阅读呢?意识形态色彩不那么浓厚的全语言阅读法很早就已经对此给出了很好的答案。研究清楚地表明,阅读的理解方面得益于语音学方法。理解叙事、论证、解释和其他语言现象,始于口头交流,而不是仅依赖于理解。事实上,有关阅读发展的研究表明,青少年读者遇到的问题是理解困难、口头语言和词汇有限,以及背景知识缺乏等综合因素造成的。丰富的口头语言交流可以帮助解决这些问题。从这个角度来看,即使学生们当时的实际阅读任务主要集中在理解上,但像仔细聆听和讨论一个故事这样的全局性任务也应被视为阅读这一更大事业上的工作。

初级版很难找到?想象力再丰富一点。从更高的视角来看。进行必要的调整以防"有人溺水",但要尽可能从一开始就把整体游戏放在心里。除此之外,还要确保学习者就像学游泳的孩子一样,能够看到整个游戏,并能参与到游戏的方方面面,只要玩了都能对游戏的样态产生感觉。认知和发展心理学家Jerome Bruner在 1973年发表过一个著名论断:我们从这个假设出发,即任何学科都能以某种在知性上坦诚的方式有效地教授给处于任何发展状态的任何儿童。

最后,假设我们已经找到了优秀的初级版本,并让学习者参与其中。然后呢?我们如何进入游戏的完整版本?

通向完整版的整个游戏的过程就像一个由初级版本组成的阶梯,每个台阶依次变得更复杂,要求也更高。数学建模的早期经验可以从简单整数运算开始,进而到分数和小数,再到代数及其他。扩展的是数学概念、工具的范围以及建模挑战的复杂性,一以贯之的是用数学来表示世界的某些部分以揭示模式和计算序列的想法。文学解读的早期经验可以从简单的故事和问题开始,如“这对你意味着什么?”和“你这么说是因为从故事中看到了什么?”。接下来,推进到对故事中神话元素的思考、由内部冲突驱动的人物发展甚至更多。扩展的是文学概念、工具的范围以及文本的复杂性,一以贯之的理念是对作品意义和手法的有据可依的阐明。沿着这样一个初级版本的阶梯——每个台阶都可能是另一个台阶的阈值体验——逐步获得更复杂、更精深的理解。

这一切的终点在哪里?对任何财富追求都没有真正的终点。推动学术或实用技术进步的可能性是无穷的,今天最先进的版本很可能只是明天的初级版本,我们不必忧虑阶梯顶端以及是否存在顶端的问题。大多数教育所面临的挑战都在阶梯的开端:让学习者学起来,让他们在整体游戏的有意义的版本中持续前进。

\section*{对恰当游戏的追求}

最近,两位富有奉献精神和创造力的教师在一次会议上简略地分享了两个有趣的生物教学案例:有丝分裂舞 和设计一种鱼 。如果你还记得基础生物课,你可能还有印象:有丝分裂是无性细胞繁殖的过程,通过这个过程,细胞一分为二,每个子细胞共享母细胞的全部基因。有丝分裂过程相当复杂,分为多个步骤。当然,有性生殖中的减数分裂过程更加复杂。

学生很难理解的有丝分裂的步骤,这位老师在教学中用有丝分裂舞的方法帮他们做到了。学生以小组为单位扮演细胞的各部分,用一种特别设计的舞蹈来表演有丝分裂的步骤,以一种积极的、充满活力的方式为自己重新编码,再现了这一基本的生物过程。虽然也有一些有丝分裂舞蹈的俗套版本,其中学生只需按预设的舞步动作;但这位教师的做法初衷是要学生们编排自己的版本,这是一种更具建设性的努力。

设计一种鱼也要求学生扮演积极主动的角色。这里的主题是适应生态,要求每个学生设计一条适应某种水体生态的鱼。学生必须设计出独特而合理的适应性,使他的鱼拥有自己的生态位,并说明它的生活方式、适应优势和类属。我有幸仔细阅读了一些学生撰写的关于他们的鱼的报告,这些报告展现了令人印象深刻的细节和对生物想象力的追求。

跳有丝分裂舞和设计一种鱼都是整体游戏。它们都涉及探究,要求创造出一些东西,赋予原本可能看似枯燥的信息以意义。两者都提供了进入复杂性的方法。渐渐地,我开始意识到,它们在某一方面是截然不同的:设计一种鱼比跳有丝分裂舞更能体现生物学科的特点。
设计一种鱼要求学生在创造生物时进行生物学的思考,考虑可获得的食物、竞争、捕食者等问题。同样的思维模式在也反复出现于其他生物探索中。对学生来说,设计一种鱼可以成为生物学思维的一个阈值体验;相比之下,"有丝分裂舞"要求学生进行的是舞蹈思维,而不是生物思维。有丝分裂的步骤是课本上的内容,这种编排方式有助于学生了解这些步骤是如何运作的,就其本身而言是很好的,但它实际上并没有让这些知识更进一步。那你能用它做了什么呢?如果把生成知识视为目标,那么学生获得的是舞蹈编排的而不是生物学的阈值体验。

当我翻阅许多教师开发的带有整体游戏特征的学习实例时,我发现"有丝分裂舞"与"设计一种鱼 "的矛盾一再出现。换句话说,我们有一个整体游戏,并不意味着它就能突出我们想要的东西。很多有趣的部分可能聚焦于“椟”了 。

道理很简单:如果我们想促进学生对某一学科或学习领域的理解和投入,仅仅在该主题附近弄一个老掉牙的整体游戏是不够的,我们要的是一个目标明确的整体游戏,一个能让学习者集中参与该学科或领域的生成性知识和思维的整体游戏。由于令人兴奋的活动对教师和学生都极具诱惑力,上述目标很容易迷失。

综上,我很乐见学生们跳有丝分裂舞,而不是简单地背诵阶段划分;我也很高兴他们在这个过程中学到了一些舞蹈知识。不过,我倾向于认为,这可能让他们对舞蹈的理解比对有丝分裂这个非常特殊的主题的理解的价值更大。

\section*{让游戏持续运转}

在1970 和 1980 年代,众多研究机构发展了一个听起来枯燥但却具有实际意义的概念:学术学习时间。教育家David Berliner提出了一个问题:教学时间的争议是怎么回事?他本人对这一概念和结果进行了很好的总结。这个故事源于一个观察:在许多学习环境中,似乎存在着相当量的松懈状态。有些是由于设置时间和转换时间造成的,有些是由于被动听讲造成的,有些是由于选择的活动没有真正关注教学目标造成的,有些则单纯由于无聊和走神造成的。

为了获知学生参与学习的程度,研究人员构建了一些度量指标,如分配时间、参与时间和转换时间。特别有说服力的是学术学习时间,大致是指学生参与到旨在实现预定目标并逐步取得中高等程度的成功的这类活动的时间。对低年级学生来说,相对较高的成功率尤为重要;低成功率总是危险的,它会挫伤学生的积极性,提示布置的任务过于艰巨,不利于有效学习。

学术学习时间能很好地预测学生的学习效果,比坐教室的时间预测效果要好得多。这些研究揭示了学习环境的微妙情况:学习者在那里并不意味着他们学到了很多东西。有效的学习需要对整个环境进行巧妙的管理,将学术学习时间提升到接近总可用时间的水平,这样才能充分利用时间,而不是让它像夹在两指之间的沙子一样溜走。

整体学习的理念并不能自动解决学术学习时间的问题。我们每个人都有可能在并没有做什么的情况下参与整场比赛。我再次想到棒球这个奇怪的运动,大多数球员在大部分时间里都啥也不做。棒球10\%是行动,90\%是等待——等待轮到你击球,在垒上等待有人击球并推动你前进,在外野等待击球来到你的位置,或者在三垒等待球沿着第三垒线下来,或者等待一个跑者从二垒接近。

对棒球来说,等待是比赛节奏的内在要求。尽管棒球是一个极端的例子,让游戏持续运转的一般性问题却是充分发挥整体学习作用的根本。

在考虑学术学习时间时,可以从四个角度出发:节奏(pace)、专注(focus)、展开(stretch)和坚持(stick)。喜欢首字母缩略词的人可以用pfsst。
\begin{enumerate}
    \item 节奏。各学习者是否在大部分时间都积极参与?有适当节奏的时间可避免走神和松懈。
    \item 专注。学习者的活动是否属于我们希望他们变得更好的核心游戏,而不是瞎忙?
    \item 展开。学习者是否受到了最佳挑战?如果学习者觉得一切都很容易,那他们就不可能学到很多东西;如果学习者不断遇到挫折,那他们也不可能学到很多东西。
    \item 坚持。逐步展开的活动模式中,是否有一部分是专门为帮助知识、理解和技能的掌握而设计的?坚持包括刻意练习、反思、总结,以及稍后对想法和做法的重温等要素。
\end{enumerate}

将所有这些结合在一起,我们就称之为游戏动量,即朝着设计方向无缝进行的活力运动。

节奏问题往往发生在缝隙之间,尤其是在课堂中。当学生听讲座或看视频时,他们只是应该听,还是要做一项有助于他们积极处理想法的任务?当教师向一名学生提问时,其他学生的角色是什么?在小组合作中,小组是否足够小,以减少边缘参与者的问题?在教师与全班的互动中,在提出问题后是否有用于学生思考的等待时间?立即点名一方面会减少学生的反思,另一方面会偏向于那些自认为已知道答案的学生。学生在课堂上思考一个问题时,他们是否会被要求写下几个字?当他们必须要写时,就会调动他们的思维走向具体化。换句话说,好的节奏,就是要以促进大多数学习者在大多数时间里积极参与的方式来组织微妙的教学。

即使节奏很好,但当学习者发现自己所玩游戏的部分过于边缘化而无法产生预期的学习效果时,就会出现不专注问题。例如,为了学习如何处理金钱和基本的经济学知识,学生在教室里开设了一家模拟商店,结果却发现大部分时间都花在了商店的杂事上,比方说家具和装饰。又例如,大学生在教学设计课上,决定开发基于计算机的课程作为课程作业,但实施中把大部分时间都花在了编程语言上,而不是打磨学习方法上。

一般而言,任何学习活动都有需要关注的次要方面。一定程度的关注可以丰富学习内容,但有时次要内容会吞噬大部分学习时间。因为次要方面本身就很吸引人,以至于有时人们几乎不会注意到。装饰商店可能比经营商店更有趣!但是,学习安排是什么?对活动的定义和结构做出好的选择,可以确保大部分时间用于核心内容。

关于展开,也许最棘手的问题是不同的学习者可能处在不同的位置。一个学习者觉得太难了,无法进行有成效的学习,而另一个学习者可能又觉得太容易了。有时,这要求教师进行非正式或正式的判断。如果可以的话,更好的办法是让学习者自己找出适当的挑战水平:如果接下来的两个问题看起来很容易,那么就跳过十个问题。你最难解决的是哪类问题,你在哪里可以找到更多此类问题和处理技巧?这些是第3章以及第7章的内容。

关于坚持,也许最棘手的问题是在正式的学习中把一些东西用后即抛的倾向。我们学完了工业革命、线性方程组或申命记(Deuteronomy)后,就不会期望在随后一段时间内再看到这个主题。我们没有系统地重温的机制,没有一种可以整合来自多个方向的想法和理解、将诸多事物汇集成一个更大规模的事业的机制。在这里,"整体学习"会提供有益助力,因为它就是游戏的名字。

\section*{面向理解的游戏}

想象一场太空中的雪仗:十几名宇航员在地球上空自由落体,大致上围成一个圈。在他们太空服口袋里装着雪球,因为把东西送入轨道的每克成本非常高,这雪球非常贵。不过这只是想象,所以我们用大富翁的钱来买单。

通讯器里响起了信号,每位宇航员从袋中拿出一个雪球,投向圆圈另一侧的宇航员。问题:假设他们都是地球上的神枪手,他们有可能击中对方吗?一个更大的问题:若他们试图将雪仗继续下去,会发生什么?

这个问题或许会让人想起中学或大学时学习的牛顿定律。在我们继续之前,你可能需要思考一下答案。

一个会让牛顿爵士高兴的回答可能是这样的:当宇航员开始打雪仗时,他们也将开始互相远离。向前扔雪球的动作也会把宇航员向后推,这是作用力和反作用力原理。不仅如此,扔雪球还会使宇航员旋转,因为扔雪球的动作不在宇航员重心上。宇航员如果想避免这种情况,就必须将雪球从他们身体的中间向前推,这样动作就会发生在从他们的重心向外的矢量上。他们开始相互远离并旋转。即使他们做出来第一次投掷的动作,他们也不太可能撞到任何东西,但附近航天飞机上的工作人员将不得不投入大量时间来回收漂移的宇航员。
这就是一个玩简版的牛顿定律预测和解释游戏的例子,同时也是一种理解测试。如果你理解牛顿运动定律,就应该能够用它们进行推理;如果你不理解,仅凭日常直觉是不可能做出正确预测的。

这也让我们有机会审视正式和非正式学习的最基本目标之一:理解。尽管死记硬套的学习在某些方面能达到很好的结果,但几乎每个人都同意,教育的更大愿景是在理解的基础上进行学习。然而,有两个问题的答案却不那么容易找到:理解意味着什么?理解与玩好整个游戏之间有什么联系?

"理解 "意味着什么呢?暂且以牛顿定律为例。如何才知道某个学生是否理解了牛顿运动定律?许多类证据都不能说服我们。例如,学生可能会背诵定律,可能会写出一些正确的方程式,可能成功地解决了三四个标准的章末问题,但这位学生仍可能会说:如果距离不太远,瞄准得好,打雪仗的宇航员很容易击中对方。

我们理解的真正标准只能是表现。当人们理解某件事情时,他们能利用所学知识灵活地思考和行动,而不仅仅是排演信息和执行常规技能。如果你不能用牛顿定律思考,你就没有真正理解牛顿定律。如果你不能像一个公民那样思考和行动,你就不能真正理解公民的含义。

前面我提到,我和一些同事开发了一个面向理解的教学的框架。该框架的核心是理解的表现视角,即理解需要被视为一种灵活的表现能力。回顾之前的几个例子,Lois Hetland 在讲授殖民地时期的美国的教学中,帮她的学生培养历史思维能力;Joan Soble在帮她的学生成为更具艺术性的作家。

虽然听起来很有道理,但在许多方面,日常用语却在向另一个方向发展。人们通常把理解说成是“懂得了”、“明白了”或“看到了”某件事情的本原。日常提及理解时,会使用拥有、接受和感知等隐喻。这些描述我们对理解的主观体验的方式会误导我们。我们很容易觉得自己“懂”,但实际上并没有真懂。只有当我们的思维和行动都与我们所知道的相一致时,才能确定自己的确理解了某事。

这就引出了我们的第二个问题:理解与玩好整个游戏之间有什么联系?理解的表现视角提供了一个尖锐的答案:真正的玩整个游戏意味着在新的情境中灵活地思考和行动,而不仅仅是千篇一律地重复旧模式。玩整个游戏总要有点创意,如果游戏的每一轮都是一样的,那就算不上是游戏了!

另一种思考理解的方法——心智模型——也很有帮助。当你思考"太空打雪仗"这个问题时,几乎可以肯定的是,你正在操作一个心智模型。你想象着宇航员在轨道上飞行,想象着宇航员扔雪球时会发生什么。同样,在准备求职面试的过程中,你会想象各种可能发生的情况。当坐下来写一封信或一篇文章时,你可能会在脑海中快速列出提纲。研究甚至表明,对篮球罚球等运动进行心理练习,可以提高实际技能。

心智模式是理解和整体学习的重要组成部分。广义地说,心智模式是我们头脑中的图像、想法或结构。心智模型不一定是直观的,可以是语言,或我们的身体运动感,或我们的情感,或我们向自己呈现事物的其他方式。无论以何种形式,它们都能支持灵活的思维和行动,而这正是理解的标志。心智模型为我们提供了用于推理和探索的心理表征,就像算盘或艺术家的素描为我们提供了用于推理和探索的外部表征一样。心智模型是思维的棋盘。

学习往往意味着改变棋盘,而不仅仅是在同一棋盘上用同样的棋子学习更先进的策略。有时,我们一开始使用的棋盘包含了错误、盲点和偏见。例如,设想一下牛顿定律最初的棋盘。日常经验让我们对运动物体的行为有一种有限的感知,以这种感知发论的Aristotle认为,物体会自发地减速并停止,其运动会消散。牛顿减速归因于摩擦,这是一个根本性的转变。牛顿的观点在某种程度上代表了不同游戏中的不同举动。再来看看Gandhi这样的人试图改变游戏规则的方式。Mohandas Gandhi或Martin Luther King的包容性公平世界观对人类来说并非那么自然。通常,在群体关系的初始棋盘上,以国籍、族群或宗教为划分标准的"我方"与"他方"存在鲜明不同。一种尊重他人(但不一定要拥抱他人)的更具包容性的观念是艰苦但重要的学习,也是游戏中的另一种变化。

更复杂的游戏之所以难,原因很简单,那就是人们从未见过它们。日常生活中没有足够的牛顿运动或甘地哲学来培养对游戏的感觉。创造性教学的工作之一,就是把预期的游戏放置于可及的地方,以提供阈值体验。一种非常重要的心智模式是对整个游戏的感觉。回想一下学习游泳和学习阅读的例子:还没有完全学会游泳的孩子已经知道整个游泳过程的大致情况,而不知道如何阅读但父母经常给他们读书的孩子也知道阅读的大致情况。这种高层级的心智模式非常强大,因为它们提供了一个大画面,学习者可以将特定元素融入其中,赋予它们意义和目的。对于游泳和阅读来说,做到这一点并不难。但对甘地来说,要做到这一点相当困难,不过我们肯定应该尝试。

\section*{学习的奇迹:玩整个游戏}

我在思考,如何围绕"整个游戏"来组织学习。我可能需要让学习者参与某种涉及解决问题、解释、论证、证据、策略、技能或工艺的探究或表现。学习者通常会产出一些东西——解决方案、图像、故事、文章、模型。我应该注意的是,探究或表现不仅要吸引学习者,而且要关注我真正希望他们学到的东西。

我在思考,如何判断我是否有一个完整的游戏。它可能不是例行公事,而是需要思考;它不仅是解决问题,而是涉及发现问题;它不仅有正确答案,而是涉及解释和说明;它不是情绪化的,而是激发好奇心、发现力、创造力和友情;它不是在真空中进行的,而是在社会背景下应用学科或其他实践的方法、目的和形式。

我在思考,如何才能让新手开始接触整个游戏。我可以尝试找到一个好的初级版本,也许是一个非常初级的版本。初级版本最好能给学习者提供阈值体验,让他们融入有意义的实践中。

我在思考,怎样才能让游戏持续进行,让学习者一直"玩"下去。我可能会注意"pfsst"——节奏(学习者在大部分时间里单独参与)、专注(学习者深思熟虑地做他们应该做得更好的事情)、展开(最佳挑战)和坚持(回顾、重读、排练和总结)。

如果我在思考这些事情,并为此做一些事情,那么我就是在为理解而教学。当人们能够在新的情境中灵活运用所学知识进行思考和行动,而不仅仅是排演信息和执行常规技能时,他们就理解了某件事情。
