\chapter{让游戏值得玩}
有些对话很贴近人们所关心的学习问题,我非常喜欢听他们聊。一方面,此类聊天揭示了一些学习过程,另一方面,告诉了我们很多烦恼。我遇到过很多这样的偶聊,涉及棒球、业余天文学、航海或当代小说。这些对话并不是直接关于学习,但在旁边我一直想:有趣!它揭示了人们已经达成的个人理解以及他们如何做到这一点!
有时这些对话可以更具结构性,就像对人们的学习历史进行系统考古挖掘一样。这里有一个简单但颇有启示性的挖掘计划:
\begin{enumerate}
    \item 你真正理解的一件事是什么?
    \item 你是怎么做才理解它的?
    \item 你如何知道你理解了它?
\end{enumerate}

其实,这些问题一直是前面提到的理解性教学工作中的一部分,目的是揭示人们对“理解性学习是什么样”的非正式感知。

在回答你非常了解的某事时(问题1),学术性主题当然,但非学术的也并非不可,如园艺或育儿。在我们继续之前,你可以花一分钟思考一下你对这3个问题的回答。它们也是很好的反思性问题,可用于询问他人,成人、青少年、幼童都适用。它们聚焦于人们在学校和课外学习中的被认为真正有意义的个人经历。

我乐于听人们对这些问题的回答。对于\textit{你真正理解的一件事是什么},一些经常出现的想法包括:开车、园艺、经营小企业、烹饪、航海、养育孩子、经历离婚、单板滑雪;有时是学术话题——代数、大萧条史、西班牙语。

对于\textit{你是如何理解它的},人们会说:“我做了很多”“我坚持下去了”“我与真正了解情况的人合作”“我得到了很多帮助和反馈”“我思考我正在做什么并加以改进”“我解决遇到的问题并尝试解决它们”“我教别人如何去做”。

对于\textit{你如何知道你理解了它},人们的回答大同小异:我知道我理解它,因为“我能做到”,“当问题出现时,我可以解决它们”,“我可以做出正确的决定”,“我大部分时间都能获得好结果”,“我可以解释我为什么要做我正在做的事情”,“我教过别人如何去做”。

这个普遍的结果肯定了整体学习的重要性。人们的回答几乎从不提及对某个部分的理解,比如毕达哥拉斯定理、使用登山钉的方法、《哈姆雷特》的第五幕或者深盘苹果派需要使用多少肉桂。他们摆在桌子上的几乎总是各类的整体游戏。

答案的第二个强烈模式是肯定了理解表现观点,这是在第一章末尾强调的。当人们谈论他们是如何理解的以及他们如何知道自己理解的时,总是以行动为中心——不断练习、解决问题、得到反馈、坚持不懈、教给他人。

诸多案例中闪耀出来的第三个特征是人们对所学内容的真正投入感。直接的标志是花费了多少时间和坚持了多久,但同样重要的是语气。人们谈论他们的活动时,就像回家、穿着到达完美舒适度的鞋子、熟悉的恋人一样,无论某些主题看起来多么普通,对于他们的捍卫者来说,它们就真正值得学习、理解和实践。

在人们说出他们最喜欢的理解性学习的例子时,他们所忽略的内容与他们提及的内容一样有趣:人们很少选择学术领域作为例子。我想,这并不意味着学术领域缺乏吸引力,之所以得到较少关注有几个其他原因。首先,生活中最生动和最重要的学习路径往往与家庭、出生、死亡、战争、职业和爱好有关,并不是特别学术化。其次,学校教育中缺乏整体学习的情况意味着人们没有可能会对学术追求产生兴趣的其他方式。真可惜!

这将我们带到整体学习七原则中的第二个:\textit{让游戏值得玩}。人们对学习某内容的兴趣与他的实用性想法并不恰好甚至主要相关。心理学家谈论\textit{内在动机},即不考虑其他外在的激励(如报酬、成绩或特权)条件下对一个主题或活动本身的动机。内在动机并不是理所当然的。对各个年龄段孩子对学术科目内在动机的研究揭示了一个令人沮丧的情景:孩子们一开始对学习充满热情,但随着年龄增长兴趣逐渐减少。

在Mark Lepper, Jennifer Corpus和Sheena Iyengar做的一项涉及近800名三年级到八年级学生的考察的结果表明,学术学习的内在动机从三年级到八年级持续下降。这项研究分别测量了学生的内在动机和外在动机,发现外在动机在各个年级基本保持不变。正如预期的那样,内在动机意味着更高的成就;而外在动机,特别是对轻松工作和迎合教师的渴望,与成就负相关。对于内在动机下降的一个自然假设是社会认可和内在对学术的兴趣可能带来负担。然而,研究测量了学生对社会认可的需求,结果发现它并不是罪魁祸首。作者们推测可能导致这一情况的原因之一是“学习变得越来越脱离背景,以至于学生在日常生活中越来越难找到直接相关或有用的内容。”

整体学习的第一原则---\textit{玩整个游戏}---在某种程度上已提供了对这一问题的答案。打棒球比练习击球更有趣,演奏乐曲比练习音阶更有趣,参与一些初级版的历史或数学探究比背诵日期或做算术题更有趣。以节奏、专注、展开和坚持这四点为主要特质的围绕整体游戏的活泼学习,很可能激发学习者的目标感、进步感和回报感。

还有更多值得说的。在内容方面,挑战是选择挑框架内容,以确保其真正有价值,且这个价值是透明的。在过程方面,有多种方法可以利用好入门、理解、期望和选择。每一种都可以有助于使游戏值得玩下去。

\section*{学值得学的}

人们常常回忆起他们最后一次打棒球、见到一个老朋友、约会或者痛饮一番的经历。不久前,我也思考了一个问题:我上一次解二次方程是什么时候?

这个问题有些怪客,但我对数学非常感兴趣,我真的喜欢解二次方程。所以,当我问自己这个问题时,我发现虽然我获得了数学博士学位,虽然我从事认知心理学和教育的技术职业,虽然我偶尔会使用技术统计,但我已经几十年没解过一个二次方程了。

我的高中数学老师是一位非常出色的教师,他花了几周的时间与我们一起学习二次方程。我想大多数人并不特别喜欢它,但我是真的喜欢!几乎我认识的每个人都曾有过解二次方程的考验。然而,他们中几乎没有人后来再探索过二次方程的奥秘,甚至大多数人可能已经忘记了大部分技巧。

关于持续学习的研究揭示了一个沮丧的普遍现象:多数学生只是简单地忘记了他们学的大部分内容。他们记住的内容通常理解得不好,而那些理解并牢记的知识又很少得到积极运用。学习科学家根据Alfred North Whitehead的观点为其命名:\textit{惰性知识}---即那种在测验中能记起来,但却与他们生活的实际情境没有联系的知识。

忘记、误解和惰性知识怎么会成为一个问题呢?虽说有很多因素,但很大一部分责任要归咎于学生在学校所学内容的根本性孤立脱节。代表性课程之间缺乏连接潜力,其内容与实际应用、个人洞察力或其他任何事物都没有联系。我们在学校玩的游戏与我们在外面需要玩以及想玩的游戏不够相似。为了用一个部分来表征整体\footnote{一个文学修辞手法,称为“提喻法”( synecdoche),适合喜欢博学词汇的人。},我们正面临着严重的二次方程教育问题。

我们需要有联系而不是孤立的课程,一个能与未来的洞察力和应用紧密而恰当联系的知识。美国伟大的哲学家和教育家John Dewey在论及主题为中心的教学时,设想了类似的情景,这些主题充满了可能性和联系,可以称之为\textit{生成性知识}。他希望教育能够丰富学习者生活的各个方面,能够增长与生活的许多重要方面有广泛、切实关联的知识,但不能变成充满实际应用的手册。

生成性知识是什么样子的?以数学分支中的概率与统计为例。大学前课程一般很少深入涉及这些内容,然而统计信息在报纸、杂志、新闻广播中随处可见。概率性考量在许多日常生活领域中都会出现,例如在做出有关治疗的明智决策时。统计和概率推理是一种能够产生联系的“游戏”,美国数学教师协会在其标准中鼓励给予概率和统计更多关注。如果我必须选择,我会增加概率和统计的内容并减少二次方程的内容。

再比如,考虑为什么从北爱尔兰到波斯尼亚再到南非的各个民族群体经常而又持久地互相仇恨的心理学和社会学这个黑暗主题。我们已经对种族仇恨的原因和动态有了相当深入的了解。如果我教社会学,我可能会教有关种族仇恨的根源,而不是法国大革命。或者我可以从种族仇恨的根源这个路径来教法国大革命。因为它是有联系的知识!

\section*{把值得学的挑出来}

那么,这种连接性的课程内容的构思可能来自哪里?一个优质来源是教师。前面提到的“理解性教学”模型鼓励教师将教学聚焦于\textit{生成性主题},即那些属于学科或实践中心部分的、能引起学习者以及教师的兴趣和关注的主题,这些主题使获得洞察和应用成为可能。

教师基于自己的经验和他们对所教内容的理解创建生成性主题,这并不存在官方清单,虽然能在一些著作中找到很多实例,比如Martha Stone Wiske编著的《Teaching for Understanding》中就有很多原创项目。只要将生成性主题牢记心间,无论什么级别的教育工作者都可以问自己:我可以教什么新的主题?我可以如何对已经教的主题进行新解,使其具有实在的生成性?

我在与教师一起探索这些问题时,产生了一些很好的想法。以下是一些例子:
\begin{itemize}
    \item 将文学中的正义(例如,To Kill a Mockingbird)与青少年对正义的关注联系起来,与文学作为社会评论联系起来,与近期的正义事件(如Rodney King案)联系起来,与不断出现的正义问题联系起来。
    \item 什么是生命体?病毒是活的吗?计算机病毒呢(一些人认为它们是活的)?晶体呢?
    \item 比例和比率的世界。研究表明,许多学生对这个像统计和概率一样频繁出现的核心概念理解不够。枯燥?不一定。提出这个主题的教师指出,比例和比率在多的惊人的情形中出现——乐谱、饮食、运动统计。
    \item 谁的历史?有人说,历史是由胜利者书写的。这个主题直接探讨了历史记录是如何被那些编写它的人所塑造的——胜利者、异议者和其他特定利益集团。
\end{itemize}

不要将生成性知识与仅仅是有趣或实用的知识混在一起是重要的。我们可以认为生成性的知识是一种\textit{广角的理解},一种概念的体系和一种能在许多情形下产生洞察和影响的整体性例子。回顾之前列出的主题。可认为它们是特定的学科知识,但每个主题也都是一个强大的概念体系。概率和统计学为我们提供了一个概率和趋势的窗口;种族仇恨的根源揭示了邻里、国家和其他水平上的竞争和偏见的动态;正义的模式在人类各项事务中不断出现;生命的本质在这个时代的试管婴儿和重组DNA工程中变得越来越重要;比例和比率是基本的描述模式;“谁的历史?”主题与群体认同和观念等人类的中心现象打交道。

为教师的匠心添翼的是学者关于什么值得学的见解。例如,Neil Postman在他的《The End of Education》中批评了教育心理学痴迷于手段而忽视对目的的关注。他抱怨“学习工程的常常夸大其词,获得了配不上的重要性”。与此相反,Postman将真正的问题视为一个“形而上学的问题”,一个关于基本价值的问题。他建议,为了使教育变得有意义,需要围绕能将一切联系起来的正确的“神”或“大叙事”来组织教育。Postman不太喜欢当下选择的一些“神”,如经济实用主义、消费主义或塑造了课程的技术等。他认为并不是这些“神”没能提供宏大叙事,而是他们连丰富的叙事都没提供。他们没能告诉我们足够关于我们是谁的知识,没能提供强有力的和富有成果的道德指导,没能解释世界的深刻之迷——三条他视为根本的标准。

Postman青睐于能更好完成这些任务的大叙事,如,Spaceship Earth(地球飞船)或the Fallen Angel(堕落天使)。地球飞船提醒我们,所有人都在同一生态、政治和经济船上。堕落天使关注在伟大和弱小之间的人类,努力理解人类状况的黑暗面和盲点。Postman提供了诸如此类的大叙事主题示例,以说明主题的深度和广度可以年复一年地为教育服务。无论人们如何看待,地球飞船或堕落天使最突出的是其构思的广度,我们所面对的是范围广泛的理解和以无数方式联系在一起的课程。

我的同事Howard Gardner在他的《The Disciplined Mind》一书中提出的教育愿景也值得一说。Gardner建议围绕三个总体主题组织教育\textit{:真、善和美}。这样的教育既能深刻而真实地反映当今世界的错综复杂,又能反映各学科的情况。对于真,Gardner提到了达尔文的自然选择理论。对于美,他提到了莫扎特歌剧的一首咏叹调。对于善,他提到了善的对立暗面,纳粹决定实施最终解决方案。同样,和Postman一样,非常突出概念的广度。达尔文、莫扎特和最终解决方案是否是您个人选择的焦点并不重要,重要的是实现广泛的理解以及一个连接性的课程。

纵观 Postman、Gardner 和其他来源,我发现有意义的教育愿景似乎都在讲三个基本议题:\textit{启蒙、赋权和责任}。例如,对地球飞船主题的深入探讨将启蒙我们关于这个星球的位置,赋予我们采取重要行动的能力,并培养我们的责任感。同样,对最终解决方案和相关种族灭绝事件的考察将启蒙我们认识人类本性中令人不快的强大方面(Postman的堕落天使主题),赋予我们要警惕什么以及要做什么的想法,并培养我们应对黑暗面的责任感。

如果我们所教的内容强调了广泛的理解,并将启蒙、赋权和责任放在愿景中,那么有充分的理由认为,年轻人将获得更多、理解更多、并使用更多他们所学到的内容。我在1992年的《Smart Schools》一书中提出了一句格言,在过去的几年里,我更加坚信它,它可以归结为一句话:\textit{我们的最重要选择是我们试图教什么}。当然,如何教也是一个重要的选择。确实,无论如何教,许多学生至少暂时性的都会学到一些我们教的内容。然而,他们会继续记住这些内容吗?他们有想要记住的理由吗?广泛的理解可以给他们一个理由。

那么,二次方程的命运如何?它是新世界中过时的东西吗?如果我们扩大议程,它也许就不是。我还记得几个月前,我与一组教育工作者分享了我的二次方程教育的担忧。其中一人回应说:
\begin{quotation}
    \textit{为什么不重新构思一下?给这个主题更广的视角——增长模型。在世界上,我们看到线性增长、二次增长、指数增长等模式。细胞、销售、身体、人口、经济、晶体和宇宙本身的增长是一个我们反复遇到的基本现象。二次方程可能在这个更大的视野中变得更有意义。}  
\end{quotation}

二次方程还有其他角色,但增长模型是一个好角色、一个有游戏性的角色!当然,增长模型将改变二次方程的处理方式,会放宽一些求解技术,并使其重要性更加突出;但这样做显然会更好。

\section*{善用开篇}

“这是最好的时代,这是最坏的时代;这是智慧的时代,这是愚蠢的时代;这是信仰的时代,这是怀疑的时代……”你或许认得这段话,它是Charles Dickens《A Tale of Two Cities》的开篇。这正是作家们常说的“叙事钩子”的经典例子。它立刻激发了你的好奇心。最好的时代和最坏的时代如何同时存在?智慧和愚蠢如何并肩而立?你迫切地想读下去。

作家希望读者读下去,教师希望学习者学下去。因此,我们不妨从小说家的笔记本中汲取一些启示。开头很重要!

这确实是许多老师深知的一点。我的同事Ron Ritchhart在《Intellectual Character》中描述了优秀教师在第一次面对一群新学生时是如何努力营造一种吸引人的氛围的。他讲述了一位代数老师是如何用一种令人放松的点名方式开始第一节课:这位老师强调了记住所有新面孔的困难,故意表现出自己的弱点。然后,他在黑板上写了一个来自当天报纸的谜题。这位老师解释说,是一个学生把它带到课堂上来的。然后,这位老师提到了他对小问题的喜爱,并邀请学生在一年中随时随地带来类似的谜题。接下来,这位老师在黑板上写下了Norton Juster的幽默而受欢迎的儿童读物《The Phantom Tollbooth》中一个令人望而生畏的复杂算术计算。这位老师让学生们试着算出答案,自言自语地说他最好自己算一下,再次表现出自己并非一直都掌握所有事情。当然,学生们给出了各种不同的答案。这位老师记下了自己的答案,引来了一片叹息声,但马上说他怀疑自己是否算对了。不过,他对所有提出的答案的态度很明确。“数学不是民主”,答案最受不受欢迎不重要,我们需要证明。

不难看出,Charles Dickens和这位老师在做着同样的事情。他们都在向各自受众抛出叙事钩子,希望能吸引尽可能多的人。

当然,有人可能会对自己说,“有些人会喜欢,有些人不喜欢,凡是都是这样。” 这种说法当然有一定的道理——人们确实有自己的偏好——但这个比例绝不是100\%。这让我想起人们在谈论艺术时常说的一句话:“我对艺术不太了解,但我清楚自己喜欢什么。” 事实是,人们往往\textit{不}知道自己喜欢什么,直到他们积累了相当多的经验。

为了积累这种经验,人们需要稍微投入一些。还记得前几章提到的“阈值体验”的概念吗?即使是初级版本,游戏也能连贯地进行。让游戏值得玩,无论是什么游戏,第一个挑战就是让人们足够深入其中,给游戏一个机会。

当然,吸引人的过程不会止步于此。无论主题本身的价值和魅力如何,我们每天都可以做很多事将其最佳的一面展示出来。

\section*{充分理解}

让我们继续这个文学主题。就比如,你正在上一堂课,学习爱尔兰诗人William Butler Yeats的诗《Sailing to Byzantium》。随着课程的进行,学生们发现它出现了令人惊讶的转折:更大的主题不是Yeats的诗,而是不朽的诗。《Sailing to Byzantium》前往一个神话中的拜占庭,在那里诗人想象自己变成了一个金色的雕像,以此来表达对不朽的渴望:“such a form as Grecian goldsmiths make/Of hammered gold and gold enameling/To keep a drowsy emperor awake(像希腊金匠制作的/锤炼的黄金和金色的珐琅/让昏昏欲睡的皇帝清醒)” 同样是阅读、品味和思考不朽主题的,还有Keats的《Ode to a Grecian Urn》,其中艺术达到了永恒的静止状态,而莎士比亚的十四行诗《Shall I compare thee to a summer’s day?》则在结尾让它所致敬的女士通过诗人永垂不朽的文字获得不朽。

揭示了不朽诗歌的主题后,为帮助集中和激发学生的注意力,老师还宣布了理解方面的具体目标。老师还邀请学生添加一两个理解目标,他们热情地做到了。这些目标要求学生们理解不朽的魅力以及诗人们表达这种魅力的不同方式。还有一些个人和小组活动,旨在鼓励积极参与和个人责任。一项活动让学生寻找一首具有类似主题的二十世纪诗歌,在小组讨论中,学生将比较和对比这首现代诗歌与这些更经典的作品。另一项活动要求学生从个人联系出发解读一首诗歌并解释这些联系。

在学生们探索这些诗歌时,会收到大量关于他们工作的反馈,有时来自老师,有时来自其他人,有时来自他们自己。这些反馈几乎没有打分的意味,而是以加深学生自己的学习过程为中心。从更大的视角看,所有这一切都是学习体验的一部分,重点是理解诗歌——主题意义、美学意义、技巧意义以及许多不同类型的意义。不朽诗歌的主题就位于这个全景之中。

这里有一个我前面提到的理解性教学框架的例子。所有理解教学框架的特征都出现在上面:
\begin{itemize}
    \item \textbf{一个生成性主题},不朽的诗歌
    \item \textbf{理解目标},例如理解不朽的魅力以及诗人如何表达它
    \item \textbf{理解表现},学习者为灵活地思考和运用他们的知识而进行的活动
    \item \textbf{持续评估},及早且频繁的反馈,并非主要针对分数,而是为了扩展理解
\end{itemize}
	
有兴趣了解理解性教学细节的读者可以在Martha Stone Wiske编著的《Teaching for Understanding》、Tina Blythe的《The Teaching for Understanding Guide》以及Martha Stone Wiske等人合著的《Teaching for Understanding with Technology》中找到更多信息。WIDE World 是哈佛大学教育研究生院开发的一个在线专业发展项目,也致力于促进理解性教学。

理解性教学是一种让游戏值得玩的方式。你看,理解不仅仅是一种成就,也是一种动力。当我们理解时,我们就会更加投入,当我们发现自己正在建立理解时,我们就会更加着迷。这不仅是常识,也是研究结果。在最初提出该框架的研究中,参加理解性教学课程的学生报告说,他们认为这种学习方式的各种特征是有价值的。

要开启理解的磁力,仅使用上述四部分框架是不够的。那些锻炼和扩展理解力的思考和行动,很容易被相对空洞的乐趣所取代。假设本月的主题是生态学,这无疑是一个生成性主题。我们希望学习者对生态学的几个方面有广泛的理解。一个诱人的活动可能是要求学生制作一张巨大的墙报,展示一个池塘复杂的生态系统。这当然可以是一个引人入胜的合作性项目,并且可以为以后的课程制作一个有用的展示。但它是否是一种理解的表现?不完全是。它本身并不涉及太多思考,学生只需要按照指示将各个部分拼凑在一起。回想一下上一章中关于节奏、聚焦、展开和坚持的原则,它缺乏专注和展开。

另一个活动可能是做一些使用生态学词汇的填字游戏。如果这个填字游戏设计得很好,会需要做很多思考。问题是,这种思考与理解生态系统关系不大,而更多地与将单词拼凑在一起有关。填字游戏可能出于其他原因而有价值,但它不是关于生态学的理解表现,它缺乏正确的聚焦。

那么这个呢:要学生参观当地池塘,尽可能识别各种大小的生物,并绘制池塘的生态图。这听起来比填字游戏更好,它有点像墙报,但现在学生必须自己制作图表。

这当然好多了,但仍有一个坑需要避免。一个小组可能会识别出许多生物并制作一个大型图表来显示它们的位置,但没能说明它们之间的关系。学生们很可能会称之为“生态图”,但它没有显示食物网和类似的依赖关系,这再次是聚焦和展开的问题。
我们在这里看到的这个坑可能会在许多我们认为是好的理解表现的活动中出现:写一篇论文、写一个故事、制作一个图表、进行讨论、创作戏剧等等。所有这些都很有好,因为它们提供了进行理解表现的充分机会。然而,它们通常都不需要大量的针对性思考。学生可以轻松地完成这些任务,用关于主题的象征性信息来填充论文、故事、图表、讨论或戏剧。

那么如何填平这个坑呢?用一种需要真正理解表现的方式来定义生态图绘制活动并不难:
\begin{quotation}
    \textit{用箭头显示依赖关系:X 吃 Y,X 栖息于 Y,等等;标明它们是什么;并在图表上写下你从观察或文本中得到的关于每种依赖关系真实性的证据。}
\end{quotation}

要以这种形式玩这个游戏,你必须用你所知道的和你能找到的东西来思考,聚焦和展开清晰可见。

\section*{用好期望}

你是个好的学习者么?学习研究有个引人注目的发现,就是学习者的水平很大程度上取决于他人和自己的期望。我们通常认为能力是学习者水平的决定因素,但事实上,学习者的态度、自信和参与度也是非常重要。

也许最著名的结果出自Robert Rosenthal和Lenore Jacobson在1968年出版的《Pygmalion in the Classroom》。罗马诗人Ovid的故事说,雕塑家Pygmalion创造了一座迷人的女性雕像,然后爱上了它,并祈祷Venus使雕像变成活的……成功了。Rosenthal和Jacobson的数据表明,教师的期望具有Pygmalion式的效果:当老师被告知某些学生将出现发展飞跃时,几个月后的测试结果会显示这些儿童的智商确实比其他儿童提高了——尽管这些儿童实际上是通过欺骗性操作随机选择的。作者解释说,这意味着教师以帮助这些学生绽放的方式对待他们。

虽然Pygmalion效应普遍流传,但也遭到广泛质疑。问题 1:只有低年级的学生表现出明显的改善。问题 2:试图重复该研究的工作通常都没能证实它。然而期望效应的概念幸存了下来,又被称为自我实现预言——在智商方面可能不是,但在内容领域和其他增长维度的成就方面肯定存在。

教师可能会通过什么样的行为来传递对学习者的期望信号呢?不难想象。是教师总看向那个学生,在其他人都不知道答案的时候?还是总不看向那个学生,暗示他或她从不知道答案?教师是否等待学生的回答,好像确信他或她心里有答案?或者快速转移到另一个学生?当教师将学生分配小组以便互相帮助时,那个学生是否总是被安排与“聪明”的学生一起,似乎是被选来帮助的?学生是否因为解决了相对简单的问题而获得许多赞扬,从而表明对他的或她的期望标准有多低?教师和同伴可以通过无数种方式传递智力期望,从而创造出他们所谓的现实。

在这种思路下,Rosenthal后来提出了教师期望效应起作用的四因素视角:\textit{气氛}(为预期成绩更高的学生创造一个更加温暖、更受欢迎的社会和情感氛围)、\textit{反馈}(更为周详、细致的反馈)、\textit{投入}(尝试教更多、更难的材料)和\textit{产出}(提供更多的回应机会)。注意,这四个因素并不一定会给教师期望值更高的学生带来优势。“逆期望”效应也是可能的。一些教师选择关注似乎较弱的学生的困境,为他们的发展创造一个更好的气氛、反馈、投入和产出。

当然,自我实现预言不仅仅关乎教师对学习者的期望,还有学习者对自己的期望。斯坦福大学心理学家Carol Dweck和她的同事们提出了一个有趣的观点。多年来,她的团队一直研究年轻人如何看待自己的智力能力以及这种看法如何影响他们在学习中的努力投入。Dweck区分了\textit{增量学习者}和\textit{实体学习者}\footnote{incremental leaner and entity leaner}。实体学习者认为智力是固定的,认为“要么会,要么不会”。增量学习者则认为智力是可以发展和提高的。这两种哲学观点转化为学习行为上的明显对比。当实体学习者遇到难以理解或解决的问题时,他们很容易放弃。相比之下,增量学习者会像雕刻石头一样,逐渐使挑战变得可解决。

人们可能会认为,实体学习者是那些不太聪明的人,但事情并不是那么简单。Dweck和她的同事们的研究表明,许多实体学习者都是最优秀和最聪明的人。他们自己的智力设下了一种陷阱,因为他们一生中在教育中遇到的各种障碍都能轻松跨越。当他们遇到不太容易跨越的障碍时,他们倾向于认为这次太高了,超出了他们的能力。

一个阴性结果非常令人不安:初始困难可能会使实体学习者在后续学习中失败,即使初始困难与后续学习无关。Barbara Licht和Carol Dweck通过自我报告问卷确定了一些五年级学生的掌握或增量心态。后来,学生们收到了介绍“一个有趣的新主题”的小册子,实际上是一些基本的心理学原则,都有五部分。第一部分是介绍性的,一些小册子的第二和第三部分写得非常清楚而在另一些中则写得有些混乱。学生们学习后,参加一个简单的多选题测试,评估的却是对第四和第五部分、而不是第二和第三部分的掌握情况。对没有得满分的学生,给他们鼓励、让他们复习,然后进行另一个简单的测试,直到时间耗尽。

结果揭示了实体心态的毁灭性影响。收到第二和第三部分写得非常清楚的小册子的学生,表现得相当好,并且在获得进一步学习机会后表现得非常好,无论他们是增量或实体心态。这表明,两组学生在智力能力上是相同的。转而看那些收到混乱版本小册子的学生,具有增量心态的学生并没有被吓倒:他们在第四和第五部分的表现仍然相同。然而,对于具有实体心态的学生,混乱版本的影响是戏剧性的。他们在第一次测试中表现非常糟糕,即使在获得复习机会后,只有35\%的学生得到了满分,而其他学生,包括那些没有遇到早期混乱的实体学习者,则有约70\%的学生得到了满分。

Dweck为故事添加了一个Pygmalion式的转折,揭示了教师的行为如何塑造学习者的实体期望。Dweck和她的同事们仔细观察了课堂互动,揭示了某些教师如何将实体态度传递给学生。例如,教师可能会说这样的话:“我知道这里每个人都发现数学是一个困难的科目,但是……”或“Sandy,那是个非常好的尝试”,暗示她不需要再尝试其他方法。还有一个熟悉的做法,即选择前三四个举手的学生之一,暗示只有马上想到的、而不是需要思考的答案才是值得听到的答案。

让游戏值得玩的好策略与教练在棒球、足球或其他运动中所做的事如出一辙。好的教练不仅通过传授技能,还通过培养学生的性格来达到目的。他们传递高期望,让每个团队成员自信和投入。这并不意味着人都是相同的,显然不是。但是,天赋不同于态度,而态度——无论是教师还是学生的——最终证明非常重要!

\section*{善用选择}

我妻子是个大歌剧迷,我是个小歌剧迷。我们俩都喜欢各种古典音乐和其他音乐,但倾向于不同流派。在我妻子看来的一个伟大的歌剧体验,对我来说只是一个不错的体验:我很高兴能够在那里。在她看来还算愉快的歌剧体验,对我来说只是非常缓慢的三个小时。不管学习者年龄或类型如何,哪怕在相当大的基本兴趣领域中,也不会同等程度的喜欢摆在他们面前的一切。这很明显,也很重要:人们有偏好、倾向和怪癖。

即便如此,人们在学习某件事时的参与度不应仅被视为他们对这件事的喜好程度。事实证明,内在动机不仅仅是个人偏好的问题,还与整体学习方式有关。一段良好的学习体验可能无法让我变成一个热情的歌剧爱好者,但它能产生显著的影响。

激发热情的一个有效方法是给予选择。作为一个经验法则,当学习者感到他们可以选择关注的内容和学习的方式时,他们更可能表现出内在动机,从而实现更广泛和更深入的学习。

让我们先从故事的阴暗面开始,即选择的对立面——强制。在1966年,J. W. Brehm提出了一种非常有趣的动机模式,称为\textit{抗拒}\footnote{reactance},指的是人们在感知到自己的自由受到限制时的反应。这些自由的范围可以很广,从宗教和政治选择等基本问题到诸如禁烟政策或乱扔垃圾法等小限制。人们往往会对限制产生负面反应,表现方式从被动抵抗,到小规模破坏,甚至到彻底的大规模反叛。

学者在大学的厕所隔间的墙上张贴了不同的标语,有些是“不要在墙上涂写”,而其他的则在句子前加上了“请”字。有些标语声称来自大学警察局局长,而另一些则标明只是来自大学警察局的地面委员会。你认为哪个厕所隔间收集到的涂鸦最多?你可能猜到了,是那些没有“请”字的和来自高权威人物的。

这与正式教育的联系显而易见。课堂上涉及大量的约束——你必须掌握什么、作业的截止时间、需要遵守的行为规范。毫无疑问,结构是必要的,但如果学习者只感受到结构而没有灵活性,抗拒心理很可能会出现,从而削弱内在动机。有一点自由空间可以走更远。

不仅是选择本身,还有选择的理由。大量研究表明,外在动机可能会削弱内在动机。当强烈的外在动机(如金钱奖励、成绩、社会地位)占据学习者的思维时,可能会模糊内在价值。在一项示例性研究中,一些孩子首先参与了艺术活动,然后收到了奖励证书;而另一些孩子则没有获得证书。后来给他们机会再次使用艺术材料时,获得证书的儿童中较少的人想这样做。在另一项大学层面的研究中,学生被要求对参与创意写作的理由进行排序。一些学生收到了一份偏向外在理由(如公众认可)的列表,而另一些则收到了一份偏向内在理由(如自我表达)的列表。每个人随后都立即写了一首俳句。最终,外在组的俳句在评审中获得的质量评分较低。因此,我们再次强调,这不仅仅是有无选择的问题,还有选择的因素——内在还是外在?

甚至知识和想法的呈现方式也可以创造出更多或更少的选择感。通常,呈现方式是非常帝国主义的——事情就是这样。相信它!在90年代末,哈佛心理学家Ellen Langer和同事们引入了\textit{条件式教学}的概念,其中语言包括较少的绝对语句和更多的短语,如“可能是”和“可能不是”。系统研究表明,信息的保留率与之前相同,而信息在创造力和问题解决方面的应用却更好。关键不是一切都应该被抛弃,而是许多教学设置具有绝对主义的特点,剥夺了学习者的有意义的选择权和减少了他们的内在动机和参与度。

甚至知识和理念的呈现方式也能影响选择感的强弱。通常,呈现方式是绝对的——事情就是这样。相信它!在90年代末,哈佛心理学家Ellen Langer及其同事提出了“条件性教学 ”的概念。这种教学中,使用更少的绝对措辞,而更多地采用了“可能是”和“可以是”等表达。这些说法邀请学习者参与内容,并自行做出判断。系统研究表明,信息的保留率和原方法同样良好,而在创造性和问题解决中的信息使用效果更佳。关键不是一切都应该被抛弃,而是许多教学环境具有绝对主义的特征。这剥夺了学习者有意义的选择,降低了他们的内在动机和参与感。

选择是一个好的经验法则,但这个法则有几个意想不到的复杂关节,其中一个涉及文化变量。某些文化背景的人似乎比其他文化背景的人更不容易受到抗拒心理的影响。在一项研究中,亚裔美国儿童由于其更为相互依存的自我观念,显示出较少的个人选择优势,而英裔美国儿童则显示出更多的优势。事实上,S. Iyengar 和 M. Lepper 在1999年的一项研究中报告发现,当受信任的权威人物或同龄人为他们做选择时,亚裔美国人的内在动机最为强烈。

在2000年进行的一项补充研究中,Iyengar 和 Lepper发现,过多的选择可能会削弱内在动机。一些研究表明,少量的选择比没有选择能带来更多的参与和更深的学习。然而,当选择的数量增加到二十或三十个时,内在动机却下降。在不同的实验中,拥有众多选择的参与者(与只有六个选择的参与者相比)对购买美味果酱的兴趣较低,对撰写额外学分论文的兴趣也较低;对于选择撰写论文的参与者,其论文质量也较差;在巧克力选择任务中,他们选择巧克力而非金钱作为补偿的倾向也较低。在这些研究中,拥有许多选择的参与者实际上更享受决策过程,但他们也感到对自己的决定负有更多责任,并且对过程感到更加困惑。背后的道理是:当人们从一个较大的预设选项集内进行选择时,可能会出现Iyengar和Lepper所称的\textit{选择过载}效应 。同样,就如同内在原因与外在原因一眼,问题不仅在于是否有选择,而在于选择的类型。

在单个决策点之上,是整个学习设置的文化基调,这种基调可能比精心设计的特定选择时刻更重要。我们是否拥有需求文化还是机会文化?作为学习者,一个人是否感到被情况的需求所压迫——规则、断言,甚至需要处理大量选择的要求——或者一个人体验到一系列机会?在今天的许多课堂中,总的基调似乎更像是一种需求文化,而不是机会文化。机会文化帮助使游戏变得值得玩耍。

在单个决策点之上,是着整个学习环境的文化基调,这种基调可能比对特定选择时刻的精心设计更为重要。我们是处于一种需求文化还是机会文化?作为学习者,是否感到受到情境要求的压迫——规则、主张,甚至需要应对大量选择的要求——或者体验到一系列机会?在许多当今的课堂上,总的基调似乎更倾向于需求文化,而非机会文化。机会文化更有助于让游戏变得值得参与。

\section*{挑战,想象和其他}

对于如何让学习变得更有意义,我们总是有更多的话要说。为此,我简要地提几个进一步的想法。在充分利用期望和选择的同时,我们也可以充分利用挑战。挑战的水平是激发动力的强大因素。

这个概念的一个著名表达是Mihaly Csikszentmihalyi提出的“心流”。他巧妙地选择了这个词,唤起了我们在挑战性活动顺利进行时所体验到的那种投入和动力的感觉。Csikszentmihalyi展示了“心流”是如何反映能力和挑战之间的理想平衡的:当能力超过挑战时,我们会感到无聊;相反,我们会感到沮丧。两者之间是最佳挑战的动机甜蜜点。

如何找到这个甜蜜点呢?第一步与上一章的观点相呼应:找到初级版本。初级版本的核心在于让学习者参与到一个有意义的阈值体验中,从而避开元素过多和信息过载的问题。

到目前为止,一切顺利。但是对于一个初始兴趣、能力和抱负参差不齐的群体,如何让每个学习者都接近他们各自的最佳挑战水平呢?这里的诀窍在于构建学习情境,以便在初级版本中学习者可以专注于他们自己的最佳挑战水平。这是视频游戏最强大的特征之一,它们通常按难度级别组织,允许玩家从一个级别进阶到另一个级别,从而始终面对基于前一个级别所发展技能的可控挑战。事实上,基于技术的游戏是组织学习的一种强大资源,David Schaffer在《How Computer Games Help Children Learn》中深入探讨了这一主题。更一般的观点不是说课程设计应像视频游戏那样安排,而是说任务应该以便于不同学习者可以尝试逐渐变得更难的目标这种形式组织起来。例如,教师可以引入探究和设计项目,为一些学习者提供选择更具挑战性的任务的自由,或者提供分层问题集,让学习者尝试特定难度的题目,并在做得好的情况下继续前进。

让我们再添加另一个主题:充分发挥想象力。Kieran Egan在他的《An Imaginative Approach to Teaching》一书中专题讨论了这一问题。Egan精心挑选了一组激发想象力的认知工具。主题枯燥?那就去寻找其情感意义。识别出能够集中注意力和激发学习者的二元对立,比如好与坏、地与天、勇气与懦弱。寻找英雄人物,利用笑话和幽默,动用隐喻,让学习者参与到一种八卦中,并提出关于什么是现实和极端现实的问题等等。

伊根在他的书中提供了许多例子。例如,他展示了如何让工业革命这个沉闷的主题重获生机。Egan建议,与其展示一系列历史事实,不如为一位英雄举办一场盛大的游行。他建议的英雄之一是Isambard Kingdom Brunel)(1806-1859)。这位英国工程师通过大胆的创新和巧妙的设计,创造了令人瞩目的成就,包括吊桥和远洋客轮,其规模远远超过以前建造的任何东西,充满了技术创新,这些创新至今仍在影响着今天的工程学。

当然,Egan并不打算美化工业革命的黑暗面。他也在那里为想象力找到了空间:来自诗人William Blake和Robert Burns,以及Mary Shelley的《Frankenstein》中的意象。他用如此高调的框架来构建主题,邀请学习者参与一系列富有想象力的活动:讲故事、戏剧化、调查等等。

让学习变得更有意义,就像那些层层递进的视频游戏一样,是一个需要不断精进的任务。我最关心的不是要走多远,而是要有一个良好的开端。回到本章的第一个主题,我们首先要关注学习的内容本身。第一步是选择一个\textit{值得}学习的游戏。再次强调:我们最重要的选择是我们试图教什么。

\section*{学习的奇迹:让游戏值得玩}

我该如何教授值得学习的东西呢?可以利用对要处理的主题和框架方式的选择。可以优先考虑具有生成性的主题和广泛的理解,这些主题可以阐明人类本质、社会、伦理、知识的本质等基本问题。

我该如何充分利用开端呢?要避免一开始就堆积大量的规则和流程。建立一种开放的好奇精神,并找到一种方法让学习者尽快进入游戏的初级版本。

我该如何充分利用理解力这一强大的动力?或许可以借助理解教学框架,以生成性主题、理解目标、理解表现和持续评估为组织学习。

我该如何充分利用期望?首先,要避免发出低期望的微妙信号。要培养自信主动的心态,不是通过宣传,而是通过配置环境,让学生能够逐步成功,相信自己可以提高能力。

我该如何充分利用选择?要提醒自己,并非所有人都必须做完全相同的事情。学习者可以在个体选择中找到能量。当然,有一些要求;但希望创造一种机会文化,而不是一种需求文化。

我该如何充分利用挑战?可以从找到游戏的易于上手的初级版本开始,并配置活动,使不同的学生能够找到他们自己的最佳挑战水平。

我该如何充分利用想象力?可以借助激发想象力的认知工具,如故事、隐喻、二元对比、英雄、现实及其极端。
